\documentclass[11pt,letterpaper]{article}
\usepackage[utf8]{inputenc}
\usepackage[T1]{fontenc}
\usepackage[spanish]{babel}
\usepackage{amsmath,amssymb}
\usepackage{amsfonts}
\usepackage{amssymb}
\usepackage{graphicx}
\usepackage{lmodern}
\usepackage{xspace}
\usepackage{multicol}
\usepackage{hyperref}
\usepackage{float}
\usepackage{hyperref}
\usepackage{color}
\usepackage{framed}


%\usepackage[colorlinks=true, linkcolor=black, urlcolor=blue, pdfborder={0 0 0}]{hyperref}

\usepackage[left=2cm,right=2cm,top=2cm,bottom=2cm]{geometry}

\newcommand{\X}{\mathbb{X}}
\newcommand{\x}{\mathbf{x}}
\newcommand{\Y}{\mathbf{Y}}
\newcommand{\y}{\mathbf{y}}
\newcommand{\xbarn}{\bar{x}_n}
\newcommand{\ybarn}{\bar{y}_n}
\newcommand{\paren}[1]{\left( #1 \right)}
\newcommand{\llaves}[1]{\left\lbrace #1 \right\rbrace}
\newcommand{\barra}{\,\vert\,}
\newcommand{\mP}{\mathbb{P}}
\newcommand{\mE}{\mathbb{E}}

\newcommand{\abs}[1]{\left\vert #1 \right\vert}
\newcommand{\muv}{\boldsymbol{\mu}}
\newcommand{\mcov}{\boldsymbol{\Sigma}}
\newcommand{\vbet}{\boldsymbol{\beta}}
\newcommand{\veps}{\boldsymbol{\epsilon}}
\newcommand{\mC}{\mathbf{C}}
\newcommand{\ceros}{\boldsymbol{0}}
\newcommand{\mH}{\mathbf{H}}
\newcommand{\ve}{\mathbf{e}}
\newcommand{\avec}{\mathbf{a}}
\newcommand{\res}{\textbf{RESPUESTA}\\}

\newcommand{\fin}{$\blacksquare.$}
\newcommand{\finf}{\blacksquare.}

\newtheorem{thm}{Teorema:}


\begin{document}
\begin{table}[ht]
\centering
\begin{tabular}{c}
\textbf{Maestría en Computo Estadístico}\\
\textbf{Inferencia Estadística} \\
\textbf{Tarea 3}\\
\today \\
\emph{Enrique Santibáñez Cortés}\\
Repositorio de Git: \href{https://github.com/Enriquesec/Inferencia_Estad-stica/tree/master/Tareas/Tarea_4}{Tarea 4, IE}.
\end{tabular}
\end{table}
Escriba de manera concisa y clara sus resultados, justificando los pasos necesarios. Serán descontados puntos de los ejercicios mal escritos y que contenga ecuaciones sin una estructura gramatical adecuada. Las conclusiones deben escribirse en el contexto del problema. Todos los programas y
simulaciones tienen que realizarse en R.
%Problema 1
%------------------------------------------------------------------------------------------------------%
%------------------------------------------------------------------------------------------------------%
%------------------------------------------------------------------------------------------------------%
1. Sea $X$ una v.a continua cuya función de distribución $F$ es estrictamente creciente. El Teorema
de la Transformación Integral nos dice que $Y=F(X)$ tiene distribución Uniforme(0, 1).\\

a) Sea $U\sim Uniforme(0, 1)$ y $X'=F ^{-1}(U)$. Muestre que $X' \sim F$.\\

b) Escriba un programa que tome variables aleatorias Uniforme(0, 1) y que las utilice para generar variables aleatorias de una distribución Exp$(\beta)$. El problema debe recibir el
tamaño de la muestra que se desea generar, denotado por $m$, y al parámetro $\beta$. Deberá regresar una muestra de tamaño $m$.\\

c) Simule $m=100$ muestras Exp(1/2). Con esta muestra contruya un QQ plot exponencial y una gráfica que compare el histograma de la muestra con la función de densidad de
Exp(1/2). Comente.\\

%Problema 2
%------------------------------------------------------------------------------------------------------%
%------------------------------------------------------------------------------------------------------%
%------------------------------------------------------------------------------------------------------%
2. Sea A el triángulo de vértices (0, 0), (0, 1), (1, 0) y suponga que $X, Y$ tiene una densidad conjunta uniforme en el triángulo. (a) Halle las distribuciones marginales de $X, Y \ \text{y}\ Z =X + Y$ . (b) ¿Son $X$ y $Y$ independientes? ¿Por qué?\\

%Problema 3
%------------------------------------------------------------------------------------------------------%
%------------------------------------------------------------------------------------------------------%
%------------------------------------------------------------------------------------------------------%
3. Halle la densidad condicional de $X|Y=y$ si $(X,Y)$ tiene densidad conjunta \begin{align*}
f_{X,Y}(x,y)=\frac{1}{y}\exp \left(-\frac{x}{y}-y \right) \ \text{para} \ x,y>0.
\end{align*} 
También calcule $\mE(X|Y=y)$.\\

%Problema 4
%------------------------------------------------------------------------------------------------------%
%------------------------------------------------------------------------------------------------------%
%------------------------------------------------------------------------------------------------------%
4. Sea $Y\sim \exp(\theta)$ y dado $Y=y$, X tiene distribución de Poisson de media $y$. Encuentre la ley de $X$.\\


%Problema 5
%------------------------------------------------------------------------------------------------------%
%------------------------------------------------------------------------------------------------------%
%------------------------------------------------------------------------------------------------------%
5. Sea $(X, Y)$ un vector aleatorio con la siguiente densidad
\begin{align*}
f(x,y)=\left\{\begin{matrix}
\pi^{-1} & x^2+y^2\leq 1\\
0 & x^2+y^2>1
\end{matrix} \right.
\end{align*} 
Demuestre $X$ y $Y$ no están correlacionadas, pero que no son independientes.\\

%Problema 6
%------------------------------------------------------------------------------------------------------%
%------------------------------------------------------------------------------------------------------%
%------------------------------------------------------------------------------------------------------%
6.Sea $X_1\ \text{y} \ X_2$ v.a.i que tienen una distribución normal estándar. Obtenga la densidad conjunta de $(Y_1, Y_2)$, donde $Y_1=\sqrt{X_1^2+X_2^2}$ y $Y_2=X_1/X_2$. ¿Son $Y_1$ y $Y_2$ independientes? \\

%Problema 7
%------------------------------------------------------------------------------------------------------%
%------------------------------------------------------------------------------------------------------%
%------------------------------------------------------------------------------------------------------%
7. Ejercicio 6 de la tarea anterior.\\

%Problema 8
%------------------------------------------------------------------------------------------------------%
%------------------------------------------------------------------------------------------------------%
%------------------------------------------------------------------------------------------------------%
8. Ejercicio 3 del examen.\\


\textbf{Honours problems}

1. Sea $X$ una v.a continua con segundo momento finito. Demuestra que la mediana $M(X)$ de $X$ satisface

\begin{align*}
|M(X)-E(X)|\leq (Var(X))^{1/2}.
\end{align*}

Hint: Use los honour problems de la tarea anterior.





\end{document}