\documentclass[11pt,letterpaper]{article}
\usepackage[utf8]{inputenc}
\usepackage[T1]{fontenc}
\usepackage[spanish]{babel}
\usepackage{amsmath}
\usepackage{amsfonts}
\usepackage{amssymb}
\usepackage{graphicx}
\usepackage{lmodern}
\usepackage{xspace}
\usepackage{multicol}
\usepackage{hyperref}
\usepackage{float}
\usepackage{hyperref}
\usepackage{color}
\usepackage{framed}



\usepackage[left=2cm,right=2cm,top=2cm,bottom=2cm]{geometry}

\newcommand{\X}{\mathbb{X}}
\newcommand{\x}{\mathbf{x}}
\newcommand{\Y}{\mathbf{Y}}
\newcommand{\y}{\mathbf{y}}
\newcommand{\xbarn}{\bar{x}_n}
\newcommand{\ybarn}{\bar{y}_n}
\newcommand{\paren}[1]{\left( #1 \right)}
\newcommand{\llaves}[1]{\left\lbrace #1 \right\rbrace}
\newcommand{\barra}{\,\vert\,}
\newcommand{\mP}{\mathbb{P}}
\newcommand{\mE}{\mathbb{E}}
\newcommand{\mR}{\mathbb{R}}
\newcommand{\mJ}{\mathbf{J}}
\newcommand{\mX}{\mathbf{X}}
\newcommand{\mS}{\mathbf{S}}
\newcommand{\mA}{\mathbf{A}}
\newcommand{\unos}{\boldsymbol{1}}
\newcommand{\xbarnv}{\bar{\mathbf{x}}_n}
\newcommand{\abs}[1]{\left\vert #1 \right\vert}
\newcommand{\muv}{\boldsymbol{\mu}}
\newcommand{\mcov}{\boldsymbol{\Sigma}}
\newcommand{\vbet}{\boldsymbol{\beta}}
\newcommand{\veps}{\boldsymbol{\epsilon}}
\newcommand{\mcC}{\mathcal{C}}
\newcommand{\mcR}{\mathcal{R}}
\newcommand{\mcN}{\mathcal{N}}

\newcommand{\ceros}{\boldsymbol{0}}
\newcommand{\mH}{\mathbf{H}}
\newcommand{\ve}{\mathbf{e}}
\newcommand{\avec}{\mathbf{a}}
\newcommand{\res}{\textbf{RESPUESTA}\\}

\newcommand{\defi}[3]{\textbf{Definición:#3}}
\newcommand{\fin}{$\blacksquare.$}
\newcommand{\finf}{\blacksquare.}
\newcommand{\tr}{\text{tr}}
\newcommand*{\temp}{\multicolumn{1}{r|}{}}

\newcommand{\grstep}[2][\relax]{%
   \ensuremath{\mathrel{
       {\mathop{\longrightarrow}\limits^{#2\mathstrut}_{
                                     \begin{subarray}{l} #1 \end{subarray}}}}}}
\newcommand{\swap}{\leftrightarrow}

\newcommand{\gen}{\text{gen}}
\newtheorem{thmt}{Teorema:}
\newtheorem{thmd}{Definición:}
\newtheorem{thml}{Lema:}
\newcommand{\muh}{\hat{\mu}}
\newcommand{\thh}{\hat{\theta}}
\newcommand{\xb}{\bar{x}}
\newcommand{\yb}{\bar{y}}
\newcommand{\Xb}{\bar{X}}
\newcommand{\Yb}{\bar{Y}}
\newcommand{\zal}{$z_{\alpha/2}$}
\begin{document}
\begin{table}[ht]
\centering
\begin{tabular}{c}
\textbf{Maestría en Computo Estadístico}\\
\textbf{Inferencia Estadística} \\
\textbf{Tarea 1}\\
\today \\
\emph{Enrique Santibáñez Cortés}\\
Repositorio de Git: \href{https://github.com/Enriquesec/Inferencia_Estad-stica/tree/master/Tareas/Tarea_8}{Tarea 8, IE}.
\end{tabular}
\end{table}

1. Sean $X_1,\cdots,X_n\sim Uniforme(0,\theta)$. Sea $f(\theta)\propto 1/\theta$. Calcule la densidad posterior.

\res \begin{framed}
    \begin{thmd} \label{d_priori}
	Sea $X_1,\cdots, X_n$ con distribución a priori $f(\theta)$ entones la distribución posteriori cumple que 
	\begin{align*}
	f(\theta|x) \propto L(\theta) f(\theta),
	\end{align*}
	donde $L(\theta)$ representa la función de verosimilitud.
    \end{thmd}
\end{framed}
Ocupando la definición (\ref{d_priori}) como sabemos que la función de verosimilitud para el caso en que la m.a es Uniforme es
\begin{align*}
L(\theta) = \prod_{i=1}^n f(x_i) = \frac{1}{\theta^n}_{\{\theta \leq \max(x_1,\cdots, x_n)  \}} = \frac{1}{\theta^n}_{\{\theta \leq  x_{(n)} \}}.
\end{align*}
Entonces
\begin{align*}
	f(\theta|x)& \propto L(\theta) f(\theta)=\frac{1}{\theta^n}_{\{\theta \leq  x_{(n)}\}} \frac{1}{\theta}\\
	&\propto \frac{1}{\theta^{n+1}}_{\{\theta \leq  x_{(n)} \}}. \ \ \finf
\end{align*}
Ahora calculamos la constante de normalización,
\begin{align*}
\int_{x_(n)}^\infty  \frac{1}{\theta^{n+1}} =\left. -\frac{1}{n\theta^n} \right|_{x_{(n)}}^\infty = \frac{1}{nx_{(n)}^n}
\end{align*}
Por lo tanto, la distribución a posteriori es
\begin{align*}
f(\theta|x)= \frac{nx_{(n)}^n}{\theta^{n+1}}1_{\{\theta \leq  x_{(n)} \}}.
\end{align*}
\end{document}