\documentclass[11pt,letterpaper]{article}
\usepackage[utf8]{inputenc}
\usepackage[T1]{fontenc}
\usepackage[spanish]{babel}
\usepackage{amsmath}
\usepackage{amsfonts}
\usepackage{amssymb}
\usepackage{graphicx}
\usepackage{lmodern}
\usepackage{xspace}
\usepackage{multicol}
\usepackage{hyperref}
\usepackage{float}
\usepackage{hyperref}
\usepackage{color}

\newcommand{\azul}[1]{\textcolor{MaterialBlue900}{#1}}
\usepackage{array}

\hypersetup{colorlinks=true,   linkcolor=MaterialBlue900}
%\usepackage[colorlinks=true, linkcolor=black, urlcolor=blue, pdfborder={0 0 0}]{hyperref}

\usepackage[left=2cm,right=2cm,top=2cm,bottom=2cm]{geometry}
\title{Modelos no paramétricos y de regresión 2018-1}
\author{Tarea examen: pruebas binomiales y tablas de contingencia}
\date{Fecha de entrega: 08/01/2017}
\setlength{\parindent}{0in}
\spanishdecimal{.}


\newcommand{\X}{\mathbb{X}}
\newcommand{\x}{\mathbf{x}}
\newcommand{\Y}{\mathbf{Y}}
\newcommand{\y}{\mathbf{y}}
\newcommand{\xbarn}{\bar{x}_n}
\newcommand{\ybarn}{\bar{y}_n}
\newcommand{\paren}[1]{\left( #1 \right)}
\newcommand{\llaves}[1]{\left\lbrace #1 \right\rbrace}
\newcommand{\barra}{\,\vert\,}
\newcommand{\mP}{\mathbb{P}}
\newcommand{\mE}{\mathbb{E}}
\newcommand{\mI}{\mathbf{I}}
\newcommand{\mJ}{\mathbf{J}}
\newcommand{\mX}{\mathbf{X}}
\newcommand{\mS}{\mathbf{S}}
\newcommand{\mA}{\mathbf{A}}
\newcommand{\unos}{\boldsymbol{1}}
\newcommand{\xbarnv}{\bar{\mathbf{x}}_n}
\newcommand{\abs}[1]{\left\vert #1 \right\vert}
\newcommand{\muv}{\boldsymbol{\mu}}
\newcommand{\mcov}{\boldsymbol{\Sigma}}
\newcommand{\vbet}{\boldsymbol{\beta}}
\newcommand{\veps}{\boldsymbol{\epsilon}}
\newcommand{\mC}{\mathbf{C}}
\newcommand{\ceros}{\boldsymbol{0}}
\newcommand{\mH}{\mathbf{H}}
\newcommand{\ve}{\mathbf{e}}
\newcommand{\avec}{\mathbf{a}}
\newcommand{\res}{\textbf{RESPUESTA}\\}
\newcommand{\rojo}[1]{\textcolor{MaterialRed900}{#1}}

\newcommand{\defi}[3]{\textbf{Definición:#3}}
\newcommand{\fin}{$\blacksquare.$}
\newcommand{\finf}{\blacksquare.}
\newcommand{\tr}{\text{tr}}
\begin{document}
\begin{table}[ht]
\centering
\begin{tabular}{c}
\textbf{Maestría en Computo Estadístico}\\
\textbf{Inferencia Estadística} \\
\textbf{Tarea 2}\\
\today \\
\emph{Enrique Santibáñez Cortés}\\
Repositorio de Git: \href{https://github.com/Enriquesec/Inferencia_Estad-stica/tree/master/Tareas/Tarea_2}{Tarea 2, IE}.
\end{tabular}
\end{table}

\begin{itemize}
\item[1.] Cuando una máquina no se ajusta adecuadamente tiene una probabilidad 0.15 de producir un artículo defectuoso. Diariamente, la máquina trabaja hasta que se producen 3 artículos defectuosos. Se detiene la máquina y se revisa para ajustarla. ¿Cuál es la probabilidad de que una máquina mal ajustada produzca 5 o más artículos antes de que sea detenida? ¿Cuál es el número promedio de artículos que la máquina producirá antes de ser detenida?

\res
Sea $X$ el número de artículos producidos antes de que se produzcan 3 artículos defectuosos, entonces podemos decir que $X\sim BN(3,0.15).$ Por lo tanto, \textbf{la probabilidad de que una máquina mal ajustada produzca 5 o más artículos antes de que sea detenida} es
$$\mP(X\geq 5) =1-\mP(X\leq 4).$$
Por como se distribuye $Y$ podemos decir que \textbf{el número promedio de artículos que la máquina producirá antes de ser detenida} es
$$\mE(X)=\frac{r}{p}=\frac{3}{0.15}. \finf$$

\item[2.] Los empleados de una compañía de aislantes son sometidos a pruebas para detectar residuos de asbesto en sus pulmones. Se le ha pedido a la compañía que envíe a tres empleados, cuyas pruebas resulten positivas, a un centro médico para realizarles más análisis. Si se sospecha que el 40 \% de los empleados tienen residuos de asbesto en sus pulmones, encuentre la probabilidad de que deban ser analizados 10 trabajadores para poder encontrar a 3 con resultado positivo.

\res

\item[5.] Considera $X$ una v.a. con función de distribución $F$ y función de densidad $f$, y sea $A$ un intervalo de la línea real $\mathbb{R}$. Definamos la función indicadora $1_{A}(x):$
\begin{equation*}
1_{A}(x) = \left\{\begin{array}{ccr}
1 & \text{si} \ x\in A\\
0 & \text{en otro caso}
\end{array}\right.
\end{equation*}
Sea $Y=1_{A}(x).$ Encuentre una expresión para la distribución acumulada y el valor esperado de $Y$. 

\item[6.] Las calificaciones de un estudiante de primer semestre en un examen de química se describen
por la densidad de probabilidad
$$f_y(y)=6y(1-y)\ \ \ \ 0\geq y \geq 1,$$
donde $y$ representa la proporción de preguntas que el estudiante contesta correctamente. Cualquier calificación menor a 0.4 es reprobatoria. Responda lo siguiente:
\begin{itemize}
\item[a)] ¿Cuál es la probabilidad de que un estudiante repruebe?
\item[b)] Si 6 estudiantes toman el examen, ¿cuál es la probabilidad de exactamente 2 reprueben?
\end{itemize}

\item[8.] En una oficina de correo los paquetes llegan según un proceso de Poisson de intensidad $\lambda$. Hay un costo de almacenamiento de c pesos por paquete y por unidad de tiempo. Los paquetes se acumulan en el local y se despachan en grupos cada $T$ unidades de tiempo (es decir, se despachan en $T , 2T , 3T , \cdots)$. Hay un costo por despacho fijo de K pesos (es decir, el costo es independiente del número de paquetes que se despachen). (a) ¿Cuál es el costo promedio por paquete por almacenamiento en el primer ciclo $[0, T ]$? (b) ¿Cuál es el costo promedio por paquete por almacenamiento y despacho en el primer ciclo? (c) ¿Cuál es el valor de $T$ que minimiza este costo promedio?


\item[9.] Considere la siguiente función
\begin{equation*}
F(x)=\left\{ \begin{array}{cl}
0 & \text{para} \ x<0\\
0.1 & \text{para} \ x=0\\
0.1+0.8x & \text{para} \ 0<x<1\\
1 & \text{para} 3/4\geq x
\end{array} \right.
\end{equation*}
¿Es una función de distribución? Si es una función de distribución, ¿corresponde a una variable
aleatoria discreta o continua?

\end{itemize}


\textbf{Honour problems (no es obligatorio entregarlos, pero dan crédito extra)}\\

\begin{itemize}
\item[1.] Cambiando las hipótesis 2 y 3 que se usaron para contruir los procesos de Poisson homogéneos
a la forma indicada en la diapositiva 135, deduzca la distribución del número de eventos que
ocurren durante el intervalo [t 1 , t 2 ].

\item[2.] Sea $N(t), \ t\leq 0$ un proceso de Poisson con parámetro $\lambda >0$. Para $0<\mu <t$ y $0\geq k\geq n,$ calcule la probabilidad $P(N(u)=k|N(y)=n)).$ Interpreta los resultados.
\end{itemize}


\end{document}