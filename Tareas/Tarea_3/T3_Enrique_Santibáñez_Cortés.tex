\documentclass[11pt,letterpaper]{article}
\usepackage[utf8]{inputenc}
\usepackage[T1]{fontenc}
\usepackage[spanish]{babel}
\usepackage{amsmath,amssymb}
\usepackage{amsfonts}
\usepackage{amssymb}
\usepackage{graphicx}
\usepackage{lmodern}
\usepackage{xspace}
\usepackage{multicol}
\usepackage{hyperref}
\usepackage{float}
\usepackage{hyperref}
\usepackage{color}
\usepackage{framed}


%\usepackage[colorlinks=true, linkcolor=black, urlcolor=blue, pdfborder={0 0 0}]{hyperref}

\usepackage[left=2cm,right=2cm,top=2cm,bottom=2cm]{geometry}

\newcommand{\X}{\mathbb{X}}
\newcommand{\x}{\mathbf{x}}
\newcommand{\Y}{\mathbf{Y}}
\newcommand{\y}{\mathbf{y}}
\newcommand{\xbarn}{\bar{x}_n}
\newcommand{\ybarn}{\bar{y}_n}
\newcommand{\paren}[1]{\left( #1 \right)}
\newcommand{\llaves}[1]{\left\lbrace #1 \right\rbrace}
\newcommand{\barra}{\,\vert\,}
\newcommand{\mP}{\mathbb{P}}
\newcommand{\mE}{\mathbb{E}}

\newcommand{\abs}[1]{\left\vert #1 \right\vert}
\newcommand{\muv}{\boldsymbol{\mu}}
\newcommand{\mcov}{\boldsymbol{\Sigma}}
\newcommand{\vbet}{\boldsymbol{\beta}}
\newcommand{\veps}{\boldsymbol{\epsilon}}
\newcommand{\mC}{\mathbf{C}}
\newcommand{\ceros}{\boldsymbol{0}}
\newcommand{\mH}{\mathbf{H}}
\newcommand{\ve}{\mathbf{e}}
\newcommand{\avec}{\mathbf{a}}
\newcommand{\res}{\textbf{RESPUESTA}\\}

\newcommand{\fin}{$\blacksquare.$}
\newcommand{\finf}{\blacksquare.}

\newtheorem{thm}{Teorema:}


\begin{document}
\begin{table}[ht]
\centering
\begin{tabular}{c}
\textbf{Maestría en Computo Estadístico}\\
\textbf{Inferencia Estadística} \\
\textbf{Tarea 3}\\
\today \\
\emph{Enrique Santibáñez Cortés}\\
Repositorio de Git: \href{https://github.com/Enriquesec/Inferencia_Estad-stica/tree/master/Tareas/Tarea_3}{Tarea 3, IE}.
\end{tabular}
\end{table}
Escriba de manera concisa y clara sus resultados, justificando los pasos necesarios. Serán descontados puntos de los ejercicios mal escritos y que contenga ecuaciones sin una estructura gramatical adecuada. Las conclusiones deben escribirse en el contexto del problema. Todos los programas y
simulaciones tienen que realizarse en R.\\

% Problema 1.
%-------------------------------------------------------------------------------------
%-------------------------------------------------------------------------------------
%-------------------------------------------------------------------------------------
1. Resuelva lo siguiente:

a) Sea $X\sim Exponencial(\beta)$. Encuentre $\mP(|X-\mu_X |\geq k\sigma_X)$ para $k>1$. Compare esta probabilidad con la que obtiene de la desigualdad de Chebyshev. 

\res  Cómo $X\sim Exponencial(\beta)$ entonces $\mu_X=\beta$ y $\sigma_X=\sqrt{\beta^2}=\beta.$ Por lo que:
\begin{align*}
\mP(|X-\mu_X |\geq k\sigma_X)&=\mP(|X-\beta|\geq k\beta)=\mP(X-\beta\leq -k\beta)+\mP(X-\beta\geq k\beta)\\ \\
&=\mP(X\leq -k\beta+\beta)+\mP(X\geq k\beta+\beta),
\end{align*}
ahora como $k>1$ eso implica que $-k\beta +\beta<0$, como $X$ es una variable con distribución exponencial podemos concluir que $\mP(X\leq -k\beta+\beta)=0$. Continuando simplificando y sabiendo que $X$ se distribuye exponencial:
\begin{align*}
\mP(|X-\mu_X |\geq k\sigma_X)&=\mP(X\geq k\beta+\beta)\\ \\
&=F(k\beta+\beta) \ \ \ \ \ \ \  \text{(fda exponencial)}\\ \\
&=e^{-\frac{k\beta+\beta}{\beta}}\\ \\
&=e^{-k-1}.
\end{align*}
Ahora, recordemos la desigualdad de Chebyshev.
\begin{framed}
    \begin{thm} \label{chebyshev}
    (Desigualdad de Chebyshev) Sea $X$ una v.a.,$\mu=\mE(X)$ y $\sigma^2=V(X)$. Entonces, si $t>0$
        \begin{align}
            \mP(|X-\mu|\geq t) \leq \frac{\sigma^2}{t^2}
        \end{align}
    \end{thm}
\end{framed} 
Entonces, considerando el teorema anterior y haciendo a $t=k\sigma_X$ (note que se sigue cumpliendo que $t>0$) tenemos:
$$\mP(|X-\mu_X|\geq k\sigma_X) \leq \frac{\sigma_X^2}{(k\sigma_X)^2}=\frac{1}{k^2}.$$
Comparando la probabilidad obtenida con la cota de la desigualdad de Chebyshev:
$$e^{-k-1}\leq \frac{1}{k^2}\ \ \ \finf$$

b) Sean $X_1,\cdots,X_n\sim Bernoulli(p)$ independientes y $\bar{X}=n^{-1}\sum_{i=1}^n X_i$. Usando las desigualdades de Chebyshev y Hoeffding, acote $\mP(|\bar{X}-p|>\epsilon)$. Demuestre que para $n$ grande la cota de Hoeffding es más pequeña que la cota de Chebyshev. ¿En qué beneficia esto?

\res Para calcula la cota de Chebyshev, primero calculemos $E[\bar{X}]$ y $Var[\bar{X}]$. Como las $X_i$ son independientes y debido a que $X_i\sin Bernoulli(p)$:
\begin{align*}
E[\bar{X}]=\frac{1}{n}E[\sum_{i=1}^nX_i]=\frac{1}{n}\sum_{i=1}^nE[X]=\frac{np}{n}=p, \ \ \text{y}
\end{align*}
\begin{align*}
Var[\bar{X}]=\frac{1}{n^2}Var[\sum_{i=1}^nX_i]=\frac{1}{n^2}\sum_{i=1}^nVar[X]=\frac{np(1-p)}{n^2}=\frac{p(1-p)}{n}.
\end{align*}
Utilizando (\ref{chebyshev}) y haciendo $\epsilon=t$ tenemos:
\begin{align*}
\mP(|\bar{X}-p|>\epsilon) \leq \mP(|\bar{X}-p|\geq \epsilon) \leq \frac{\frac{p(1-p)}{n}}{\epsilon^2}=\frac{p(1-p)}{n\epsilon^2 },
\end{align*}
y como $0\leq p \leq 1$ podemos ver que $p(1-p)\leq \frac{1}{4}$ y por lo tanto la cota de Chebyshev es: 
\begin{align}\label{cota_chebyschev}
\mP(|\bar{X}-p|>\epsilon) \leq \frac{p(1-p)}{n\epsilon^2 } \leq \frac{1}{4n\epsilon^2 }.
\end{align}
Ahora, recordemos la desigualdad de Hoeffding.
\begin{framed}
    \begin{thm} (Desigualdad de Hoeffding) Sea $Y_1, \ Y_2\ ,\cdots,\Y_n$ v.a independientes tales que $E(Y_i)=0,$ $a_i\leq Y_i \leq b_i$. Sea $\epsilon>0$. Entonces, para cualquier $t>0$
        \begin{align}
            \mP\left(\sum_{i=1}^nY_i\geq \epsilon \right) \leq e^{-t\epsilon}\prod_{i=1}^ne^{t^2(b_i-a_i)^2/8}.
        \end{align}
    \end{thm}
\end{framed} 
Para determinar las cota de la probabilidad solicitada observemos que
\begin{align}
\label{valor_absoluto}
|\bar{X}-p|>\epsilon=(\bar{X}-p>\epsilon)\cup(p-\bar{X}>\epsilon).
\end{align}
Además,
\begin{equation}
\label{igualdad}
\bar{X}-p=\frac{1}{n}\sum_{i=1}^n X_i-\frac{1}{n}\sum_{i=1}^n p=\sum_{i=1}^n\frac{X_i-p}{n}.
\end{equation}
Entonces se reduce a encontrar la cota de Hoeffiding para:
$$\mP\left( (\bar{X}-p>\epsilon)\cup(p-\bar{X}>\epsilon)\right)=\mP(\bar{X}-p>\epsilon)\cup\mP(p-\bar{X}>\epsilon).$$

Primero encontremos la cota para  $\mP(\bar{X}-p>\epsilon)$. Para ello denotemos a $Y_i=\frac{X_i-p}{n}, \ i=1,2,\cdots,n.$ Ahora, como $X_i\sim Bernoulli(p)$ esto implica que
$$\mE(Y_i)=\mE\left(\frac{X_i-p}{n} \right)=\frac{\mE(X_i)-p}{n}=\frac{p-p}{n}=0.$$
Notemos que $Y_i$ tiene una cota inferior haciendo $X=0$ y una cota superior $X=1$ las cuales son:
$$-\frac{p}{n}\leq Y_i \leq \frac{1-p}{n}.$$
Cómo ya cumple todos los supuestos, podemos ocupar la desigualdad de Hoeffding en las $Y_i$. Sea $\epsilon>0$, y para cualquier $t>0$ 
\begin{align*}
\mP\left(\sum_{i=1}^nY_i>\epsilon\right)=\mP\left(\sum_{i=1}^n\frac{X_i-p}{n}>\epsilon\right)\leq \mP\left(\sum_{i=1}^n\frac{X_i-p}{n}\geq\epsilon\right) &\leq e^{-t\epsilon}\prod_{i=1}^n e^{t^2(\frac{1-p}{n}+\frac{p}{n})^2/8}\\
&=e^{-t\epsilon}\prod_{i=1}^n e^{\frac{t^2}{8n^2}}\\
&=e^{-t\epsilon} e^{\frac{nt^2}{8n^2}}\\
&=e^{-t\epsilon+\frac{t^2}{8n}}.
\end{align*}
Ya encontramos una cota utilizando la desigualdad de Hoeffding, pero no es una cota mínima. Para encontrar la mínima encontremos el valor de $t$ que minimiza el exponente de $e$, es decir, encontremos un mínimo para $f(t)=-t\epsilon+\frac{t^2}{8n}.$ Para ello utilicemos el criterio de primera y segunda derivada, derivamos $f(t)$:
$$f'(t)=-\epsilon +\frac{t}{4n}.$$
Igualamos a cero y despejamos $t$:
\begin{align*}
-\epsilon +\frac{t}{4n} &= 0\\
t&=4n\epsilon.
\end{align*}
Calculemos la segunda derivada de $f(t)$:
$$f''(t)=\frac{1}{4n}.$$
Como $f''(t)>0$ implica que $t=4n\epsilon$ sea un mínimo para $f(t)$, y la cota mínima de Hoeffding para este problema es cuanto $t=4n\epsilon$. Sustituyendo en la desigualdad calculada:
\begin{align*}
\mP\left(\sum_{i=1}^n\frac{X_i-p}{n}>\epsilon\right)&\leq e^{-(4n\epsilon)\epsilon+\frac{(4n\epsilon)^2}{8n}}\\
&= e^{-4n\epsilon^2+\frac{16n^2\epsilon^2}{8n}}\\
&= e^{-2n\epsilon^2}.
\end{align*}
Y por lo tanto (\ref{igualdad}) podemos concluir que: 
\begin{align*}
\mP\left(\bar{X}-p>\epsilon\right)&\leq e^{-2n\epsilon^2}.
\end{align*}
Realizando un razonamiento análogo determinemos la cota para $\mP\left(p-\bar{X}>\epsilon\right)$. Para ello denotemos $Z_i=\frac{p-X_i}{n}, \ i=1,2,\cdots,n.$ Ahora, como $X_i\sim Bernoulli(p)$ esto implica que
$$\mE(Z_i)=\mE\left(\frac{p-X_i}{n} \right)=\frac{p-\mE(X_i)}{n}=\frac{p-p}{n}=0.$$
Notemos que $Z_i$ tiene una cota inferior haciendo $X=1$ y una cota superior $X=0$ las cuales son:
$$\frac{p-1}{n}\leq Z_i \leq \frac{p}{n}.$$
Cómo ya cumple todos los supuestos, podemos ocupar la desigualdad de Hoeffding en las $Z_i$. Sea $\epsilon>0$, y para cualquier $t>0$ 
\begin{align*}
\mP\left(\sum_{i=1}^n Z_i>\epsilon\right)=\mP\left(\sum_{i=1}^n\frac{p-X_i}{n}>\epsilon\right)\leq \mP\left(\sum_{i=1}^n\frac{p-X_i}{n}\geq\epsilon\right) &\leq e^{-t\epsilon}\prod_{i=1}^n e^{t^2(\frac{p}{n}-\frac{p-1}{n})^2/8}\\
&=e^{-t\epsilon}\prod_{i=1}^n e^{\frac{t^2}{8n^2}}\\
&=e^{-t\epsilon} e^{\frac{nt^2}{8n^2}}\\
&=e^{-t\epsilon+\frac{t^2}{8n}}.
\end{align*}
Ya encontramos una cota utilizando la desigualdad de Hoeffding, pero no es una cota mínima. Para encontrar la mínima encontremos el valor de $t$ que minimiza el exponente de $e$, es decir, encontremos un mínimo para $f(t)=-t\epsilon+\frac{t^2}{8n}.$ Para ello utilicemos el criterio de primera y segunda derivada, derivamos $f(t)$:
$$f'(t)=-\epsilon +\frac{t}{4n}.$$
Igualamos a cero y despejamos $t$:
\begin{align*}
-\epsilon +\frac{t}{4n} &= 0\\
t&=4n\epsilon.
\end{align*}
Calculemos la segunda derivada de $f(t)$:
$$f''(t)=\frac{1}{4n}.$$
Como $f''(t)>0$ implica que $t=4n\epsilon$ sea un mínimo para $f(t)$, y la cota mínima de Hoeffding para este problema es cuanto $t=4n\epsilon$. Sustituyendo en la desigualdad calculada:
\begin{align*}
\mP\left(\sum_{i=1}^n\frac{p-X_i}{n}>\epsilon\right)&\leq e^{-(4n\epsilon)\epsilon+\frac{(4n\epsilon)^2}{8n}}\\
&= e^{-4n\epsilon^2+\frac{16n^2\epsilon^2}{8n}}\\
&= e^{-2n\epsilon^2}.
\end{align*}
Entonces: 
\begin{align*}
\mP\left(p-\bar{X}>\epsilon\right)&\leq e^{-2n\epsilon^2}.
\end{align*}
Por lo tanto, ocupando (\ref{valor_absoluto}) podemos concluir que la cota de Hoeffding es:
\begin{align*}
\mP\left(|\bar{X}-p|>\epsilon\right)\leq 2e^{-2n\epsilon^2}.
\end{align*}
Entonces para mostrar que cuando $n$ es grande la cota de Hoeffing es más pequeña que la cota de Chebyshev (\ref{chebyshev}), veamos la primera derivada de cada cota con respecto a $n$. Sea $f(n)$ el denominador de la cota de Hoeffing y $g(n)$ el denominador la cota de Chebyshev:
\begin{align*}
f'(n)&=2\epsilon^2 e^{2n\epsilon^2}/2=\epsilon^2e^{2n\epsilon^2}.\\
g'(n)&=4\epsilon^2
\end{align*} 
De aquí podemos observar que el crecimiento de la cota de Chebyshev es lineal no depende de $n$ y en la cota de Hoeffing depende de $n$. Entonces de lo anterior implica que para que 
$$g'(n)<f'(n)$$
se tiene que cumplir que $4<e^{2n\epsilon^2}$, entonces es sencillo ver que para cualquier $\epsilon>0$ se puede encontrar un $n$ lo suficientemente grande para que se cumpla la igualdad. Por lo tanto, para  algún $n$ grande cumple que el dominador del cota de Hoeffindg es más grande que la cota de Chebyshev y por lo tanto que la cota de Hoeffing es más pequeña que la cota de Chebyshev para algún $n$ grande. \ \ \ \fin 

% Problema 2.
%-------------------------------------------------------------------------------------
%-------------------------------------------------------------------------------------
%-------------------------------------------------------------------------------------%

2. Sean  $X_1,\cdots,X_n\sim Bernoulli(p)$.

a) Sea $\alpha >0 $ fijo y defina 
$$\epsilon_n = \sqrt{\frac{1}{2n} \log\left(\frac{2}{\alpha}\right)}.$$
Sea $\hat{p}=n^{-1}\sum_{i=1}^n X_i.$ Defina $C_n=(\hat{p}_n-\epsilon_n,\hat{p}_n+\epsilon_n)$. Use la desigualdad de Hoeffding para demostrar que $$\mP(C_n \text{contiene a } p)\geq 1-\alpha$$.
Diremos que $C_n$ es un $(1-\alpha)-$intervalo de confianza para $p$. En la practica, se trunca el intervalo de tal manera de que no vaya debajo del 0 o arriba del 1.

\res Observemos que el evento de que $C_n$ contiene a $p$ es igual al evento que:
$$\{p \not \in C\}=|\hat{p}-p|>\epsilon_n.$$

Entonces, como $X_i\sim Bernoulli(p)$ independientes y utilizando la desigualdad de Hoeffding encontrada en el ejercicio anterior tenemos que:
\begin{align*}
\mP(p\not \in C)=\mP(|\hat{p}-p|>\epsilon_n)\leq 2e^{-2n\epsilon_n^2},
\end{align*}
sustituyendo el valor de $\epsilon_n$:
\begin{align*}
\mP(p\not \in C)&\leq 2e^{-2n\left(\sqrt{\frac{1}{2n} \log\left(\frac{2}{\alpha}\right)}\right)^2}\\
&= 2e^{-2n\left(\frac{1}{2n} \log\left(\frac{2}{\alpha}\right)\right)}\\
&=2e^{-\log\left(\frac{2}{\alpha}\right)}\\
&=\frac{2\alpha}{2}=\alpha.
\end{align*}
Es decir, la probabilidad de que el intervalo $C_n$ no contenga a $p$ es menor que $\alpha$:
$$\mP(p\not \in C)\leq \alpha.$$
Recordando la propiedad de probabilidad $\mP(A)+\mP(A^c)=1.$ Entonces, podemos ver que
\begin{align*}
\mP(p\not \in C)\leq \alpha\\
-\mP(p\not \in C)\geq -\alpha\\
1-\mP(p\not \in C)\geq 1-\alpha\\
\mP(p\in C)\geq 1-\alpha.
\end{align*}
Es decir, queda probado que
$$\mP(C_n \text{contiene a } p)\geq 1-\alpha. \ \ \ \ \finf$$

% Problema 3.
%-------------------------------------------------------------------------------------
%-------------------------------------------------------------------------------------
%-------------------------------------------------------------------------------------
3. Una partícula se encuentra inicialmente en el origen de la recta real y se mueve en saltos de una unidad. Para cada salto, la probabilidad de que la partícula salte una unidad a la izquierda es $p$ y la probabilidad de que salte una unidad a la derecha es $1-p$. Denotemos por $X_n$ a la posición de la partícula después de n unidades. Encuentre $\mE(X_n)$ y Var$(X_n)$. Esto se conoce como una caminata aleatoria en una dimensión. 

\res Este proceso cambia de un estado a otro en dos tiempos consecutivos de acuerdo con probabilidad de transición descritas en el problemas. Estas probabilidades se pueden escribir de la forma siguiente:
\begin{align*}
\mP(X_{n+1}=j|X_n=i)=\left\{\begin{array}{cc}
p & \text{si } j=i+1,\\
1-p & \text{si} j=i-1,\\
0 & \text{otro caso.}
\end{array} \right.
\end{align*}

Esa caminata aleatoria puede también definirse de la siguiente forma: $\xi_1, \ \xi_2,\cdots$ una sucesión de variables aleatorias independientes e idénticamente distribuidas. Por la idéntica distribución denotemos a cualquiera de ellas mediante la letra $\xi$ sin subíndice. Ahora, si suponemos que $\mP(\xi=+1)=p$ y $\mP(\xi=-1)=1-p$. Entonces para $n\geq 1$ se define 
$$X_n:=X_0+\xi_1+\xi_2+\cdots+\xi_n.$$
donde $X_0$ en este caso suponemos que empieza en 0, es decir, el estado inicial de la partícula es cero. Entonces, a partir de la expresión anterior implica que la esperanza es:
\begin{align*}
\mE(X_n)=\mE(X_0)+\sum_{i=1}^n \mE(\xi_i)=n\mE(\xi)=n(p-1+p)=n(2p-1).
\end{align*}
Ahora, como $\mE(\xi^2)=p(1)^2+(1-p)(-1)^2=1$ y $\mE(\xi)=2p-1$, se tiene que $Var(\xi)=1-(2p-1)^2=1-4p^2+4p-1=4p(1-p)$. Y por lo tanto la varianza de $X_n$ es:
\begin{align*}
Var(X_n)=\sum_{i=1}^n Var(\xi_i)=nVar(\xi)=4np(1-p). \ \ \ \finf
\end{align*}
% Problema 7.
%-------------------------------------------------------------------------------------
%-------------------------------------------------------------------------------------
%-------------------------------------------------------------------------------------
7. Demuestre que la fórmula de la densidad de la Beta integra 1.

\res Sea $X\sim Beta(\alpha, \beta)$, entonces su función de densidad esta definida como:
$$f(x)=\frac{\Gamma(\alpha+\beta)}{\Gamma(\alpha)\Gamma(\beta)}x^{\alpha-1}(1-x)^{\beta-1},\ \ \ \ \ \ 0<x<1.$$
Entonces, la integral de la densidad sería:
\begin{align}\label{beta}
\int_{-\infty}^{\infty}f(x)dx=\int_{0}^{1}\frac{\Gamma(\alpha+\beta)}{\Gamma(\alpha)\Gamma(\beta)}x^{\alpha-1}(1-x)^{\beta-1}dx=\frac{\Gamma(\alpha+\beta)}{\Gamma(\alpha)\Gamma(\beta)} \int_{0}^{1}x^{\alpha-1}(1-x)^{\beta-1}dx.
\end{align}

Ahora recordemos una identidad con la función Gamma, la cuál es:
\begin{align*}
\int_{0}^{1}x^{\alpha-1}(1-x)^{\beta-1}dx = \frac{\Gamma(\alpha) \Gamma(\beta)}{\Gamma(\alpha+\beta)}.
\end{align*}
Por lo anterior tenemos en (\ref{beta}):
\begin{align*}
\int_{-\infty}^{\infty}f(x)dx = \frac{\Gamma(\alpha+\beta)}{\Gamma(\alpha)\Gamma(\beta)}\cdot \frac{\Gamma(\alpha) \Gamma(\beta)}{\Gamma(\alpha+\beta)}=1.
\end{align*}
Por lo tanto, queda demostrado que la densidad de v.a con distribución Beta integra 1. \fin

% Problema Honor 1.
%-------------------------------------------------------------------------------------
%-------------------------------------------------------------------------------------
%-------------------------------------------------------------------------------------
\textbf{Honors problems} \\
1. \begin{itemize}
\item[a)] Sea $X$ una v.a. discreta con media finita y que toma valores en el conjunto $0,1,2\cdots .$ Demuestre que 
$$\mE(X)=\sum_{k=1}^\infty \mP(X\geq k).$$

\res Definamos la siguiente notación para hacer más entendible la demostración: $$p_x=\mP(X=x), \ \ x=0,1,\cdots,$$ y $$q_x=\mP(X> x)=\sum_{k=x+1}^\infty p_k \ \ \ \ x=0,1,\cdots.$$
Entonces debemos probar que:
$$\mE(x)=\sum_{k=1}^\infty \mP(X\geq k)=\sum_{x=0}^\infty q_x.$$
Como sabemos que se cumple que $\mP(X\leq x) +\mP(x>x)=1$, observemos que:
\begin{align*}
\sum_{x=1}^Nxp_x&=p_1+2p_2+3p_3+\cdots+Np_N\\
&=(p_1+p_2+p_3+\cdots+p_N)+(p_2+p_3+\cdots+p_N)+\cdots+(p_{N-1}+p_N)+p_N\\
&=\sum_{x=0}^{N-1} q_x-Nq_N.
\end{align*}
Es decir,
\begin{align}
\label{esperanza_dis}
\sum_{x=1}^Nxp_x= \sum_{x=0}^{N-1} q_x-Nq_N.
\end{align}

Utilizando que $X$ tiene media finita, es decir, como $E(X)=\sum_{x=0}^\infty xp_x$ podemos decir que la serie $\sum_{x=0}^\infty xp_x$ es convergente. Ahora observemos que se cumple la siguiente desigualdad:
\begin{align}
\label{cota_pn}
0 \leq N\sum_{x=N+1}^\infty p_x \leq \sum_{x=N+1}^\infty x p_x.
\end{align} 
La justificación de la desigualdad, se debe a que $n>N$ por como se definieron los límites de la suma y de lado derecho a que $N$ y $p_x$ son positivos. 

Ahora, ocupemos una propiedad conocida de series convergentes: 
\begin{framed}
    \begin{thm} (Condición necesaria de convergencia) Si la serie $\sum_{n=1}^\infty a_n$ es convergente, entoncess 
        \begin{align*}
        \lim_{n\rightarrow  \infty} a_n=0.
        \end{align*}
    \end{thm}
\end{framed} 
Ocupando la propiedad anterior en (\ref{cota_pn}):
\begin{align*}
\lim_{N\rightarrow\infty } 0 \leq \lim_{N\rightarrow\infty } N \sum_{x=N+1}^\infty p_x &\leq \lim_{N\rightarrow\infty } \sum_{x=N+1}^\infty x p_x.\\
&=\sum_{x=N+1}^\infty \lim_{x\rightarrow\infty }x p_x\\
&=0.
\end{align*} 
Es decir, podemos decir que
$$\lim_{N\rightarrow\infty } N \sum_{x=N+1}^\infty p_x =0.$$
Entonces haciendo un límite en (\ref{esperanza_dis}) tenemos que:

\begin{align*}
\lim_{N\rightarrow \infty}\sum_{x=1}^Nxp_x&=\lim_{N\rightarrow \infty}\left( \sum_{x=0}^{N-1} q_x-Nq_N\right)\\
&= \lim_{N\rightarrow \infty}\left( \sum_{x=0}^{N-1} q_x\right)-\lim_{N\rightarrow \infty}\left(Nq_N\right)\\
&= \lim_{N\rightarrow \infty}\left( \sum_{x=0}^{N-1} q_x\right)-\lim_{N\rightarrow \infty}\left(N\sum_{x=N+1}^\infty p_x \right)\\
&= \sum_{x=0}^{\infty} q_x=\sum_{k=1}^\infty\mP(X\geq k).
\end{align*}
Es decir, queda demostrado que 
\begin{align*}
\mE(X)=\sum_{x=1}^\infty xp_x=\sum_{k=1}^\infty\mP(X\geq k).\ \ \ \ \finf
\end{align*}

\item[b)] Sea $X$ una v.a. continua no-negativa con media finita, función de densidad $f$ y función de distribución $F$. Demuestre que 
$$\mE(X)=\int_0^\infty (1-F(t))dt.$$

\res Utilizaremos algunas definiciones vistas en clase de confiabilidad y su relación con función de densidad. $R(x)$ se le conoce como la confiabilidad de X, la cual se define como: $$R(x)=1-F(x).$$ Es claro que se cumple que:
$$R(0)=1 \ \ \text{y}\ \ \lim_{x\rightarrow \infty} R(x)=0.$$ Derivando de ambos lados observamos que: $$R'(x)=-F'(x)=-f(x).$$
Ocupando lo anterior y la definición de esperanza de una variable continua tenemos que:
\begin{flalign*}
\mE(X)&=\int_{-\infty}^\infty xf(x)dx & \text{definición de esperanza}\\ \\
&=\int_{0}^\infty -xR'(x)dx & \text{relación confiabilidad-densidad}\\ \\
&=\int_{0}^\infty -xR'(x)-R(x)+R(x)dx&\text{sumamos un cero}\\ \\
&=\int_{0}^\infty -(xR(x))'+R(x)dx &\text{definición de derivada}\\ \\
&=-xR(x) \vert_{x=0}^\infty+\int_{0}^\infty R(x)dx\\ \\
&=\int_{0}^\infty R(x)dx\\ \\
& =\int_{0}^\infty (1-Fx(x))dx \ \ \finf
\end{flalign*}



\item[c)] ¿Cómo cambia la fórmula del caso anterior cuando el soporte de $X$ es todo $\mathbb{R}?$

\res Cuando el soporte de $X$ es todo $\mathbb{R}$ no tiene sentido la función de confiabilidad. Por lo que 
\end{itemize}
% Problema Honor 2.
%-------------------------------------------------------------------------------------
%-------------------------------------------------------------------------------------
%-------------------------------------------------------------------------------------
2. Sea $X$ una v.a. continua con primer momento finito. Demuestre que la función $G(c)=E(|X-c|). c\in \mathbb{R},$ se minimiza en $c=M(X)$ para $M(X)$ la mediana de $X$.

\res Sea $f(x)$ la función de densidad de $X$. Por propiedades de la esperanza tenemos que:
\begin{align*}
G(c)&=E(|X-c|)=\int_{-\infty}^\infty|x-c|f(x)dx= \int_{-\infty}^c (c-x)f(x)dx+\int_{c}^\infty (x-c)f(x)dx\\
\\
&=c\int_{-\infty}^c f(x)dx-\int_{-\infty}^c xf(x)dx+\int_{c}^\infty xf(x)dx-c\int_{c}^\infty f(x)dx.
\end{align*}
Ahora diferenciamos con respecto a $c$ e igualamos a cero la expresión anterior:
\begin{align*}
G'(c)&=\frac{d}{dc}\left(c\int_{-\infty}^c f(x)dx-\int_{-\infty}^c xf(x)dx+\int_{c}^\infty xf(x)dx-c\int_{c}^\infty f(x)dx \right)\\ \\
&=\int_{-\infty}^c f(x)dx+c\frac{d}{dc} \int_{-\infty}^c f(x)dx-\frac{d}{dc}\int_{-\infty}^c xf(x)dx+\frac{d}{dc}\int_{c}^\infty xf(x)dx-\int_{c}^\infty f(x)dx -c\frac{d}{dc}\int_{c}^\infty f(x)dx 
\end{align*}
Ahora por el Teorema Fundamental del Calculo. 
\begin{align*}
G'(c)&=\int_{-\infty}^c f(x)dx+cf(x)\vert_{x=c}-xf(x)\vert_{x=c}+xf(x)\vert_{x=c}-\int_{c}^\infty f(x)dx-cf(x)\vert_{x=c}\\
&=\int_{-\infty}^c f(x)dx-\int_{c}^\infty f(x)dx.
\end{align*}
Ahora igualamos a cero:
\begin{align*}
G'(c)=\int_{-\infty}^c f(x)dx-\int_{c}^\infty f(x)dx&=0\\
\int_{-\infty}^c f(x)dx&=\int_{c}^\infty f(x)dx.
\end{align*}
Es decir,
\begin{align*}
\mP(X\leq c)=\mP(X>c).
\end{align*}
Por definición de probabilidad sabemos que se cumple que $\mP(X\leq c)+\mP(X>c)=1$. Lo que implica que:
\begin{align*}
\mP(X\leq c)=\mP(X>c)=\frac{1}{2}.
\end{align*}
Y por lo tanto, cuando $c$ es la mediana (por definición) de $X$ minimiza la función $G(c) \ \ \ \finf$

\end{document}