\documentclass[11pt,letterpaper]{article}
\usepackage[utf8]{inputenc}
\usepackage[T1]{fontenc}
\usepackage[spanish]{babel}
\usepackage{amsmath}
\usepackage{amsfonts}
\usepackage{amssymb}
\usepackage{graphicx}
\usepackage{lmodern}
\usepackage{xspace}
\usepackage{multicol}
\usepackage{hyperref}
\usepackage{float}
\usepackage{hyperref}
\usepackage{color}
\usepackage{framed}



\usepackage[left=2cm,right=2cm,top=2cm,bottom=2cm]{geometry}
\title{Modelos no paramétricos y de regresión 2018-1}
\author{Tarea examen: pruebas binomiales y tablas de contingencia}
\date{Fecha de entrega: 08/01/2017}
\setlength{\parindent}{0in}
\spanishdecimal{.}

\newcommand{\X}{\mathbb{X}}
\newcommand{\x}{\mathbf{x}}
\newcommand{\Y}{\mathbf{Y}}
\newcommand{\y}{\mathbf{y}}
\newcommand{\xbarn}{\bar{x}_n}
\newcommand{\ybarn}{\bar{y}_n}
\newcommand{\paren}[1]{\left( #1 \right)}
\newcommand{\llaves}[1]{\left\lbrace #1 \right\rbrace}
\newcommand{\barra}{\,\vert\,}
\newcommand{\mP}{\mathbb{P}}
\newcommand{\mE}{\mathbb{E}}
\newcommand{\mR}{\mathbb{R}}
\newcommand{\mJ}{\mathbf{J}}
\newcommand{\mX}{\mathbf{X}}
\newcommand{\mS}{\mathbf{S}}
\newcommand{\mA}{\mathbf{A}}
\newcommand{\unos}{\boldsymbol{1}}
\newcommand{\xbarnv}{\bar{\mathbf{x}}_n}
\newcommand{\abs}[1]{\left\vert #1 \right\vert}
\newcommand{\muv}{\boldsymbol{\mu}}
\newcommand{\mcov}{\boldsymbol{\Sigma}}
\newcommand{\vbet}{\boldsymbol{\beta}}
\newcommand{\veps}{\boldsymbol{\epsilon}}
\newcommand{\mcC}{\mathcal{C}}
\newcommand{\mcR}{\mathcal{R}}
\newcommand{\mcN}{\mathcal{N}}

\newcommand{\ceros}{\boldsymbol{0}}
\newcommand{\mH}{\mathbf{H}}
\newcommand{\ve}{\mathbf{e}}
\newcommand{\avec}{\mathbf{a}}
\newcommand{\res}{\textbf{RESPUESTA}\\}

\newcommand{\defi}[3]{\textbf{Definición:#3}}
\newcommand{\fin}{$\blacksquare.$}
\newcommand{\finf}{\blacksquare.}
\newcommand{\tr}{\text{tr}}
\newcommand*{\temp}{\multicolumn{1}{r|}{}}

\newcommand{\grstep}[2][\relax]{%
   \ensuremath{\mathrel{
       {\mathop{\longrightarrow}\limits^{#2\mathstrut}_{
                                     \begin{subarray}{l} #1 \end{subarray}}}}}}
\newcommand{\swap}{\leftrightarrow}

\newcommand{\gen}{\text{gen}}
\newtheorem{thmt}{Teorema:}
\newtheorem{thmd}{Definición:}
\newtheorem{thml}{Lema:}

\begin{document}
\begin{table}[ht]
\centering
\begin{tabular}{c}
\textbf{Maestría en Computo Estadístico}\\
\textbf{Álgebra Matricial} \\
\textbf{Tarea 6}\\
\today \\
\emph{Enrique Santibáñez Cortés}\\
Repositorio de Git: \href{https://github.com/Enriquesec/Algebra_matricial/tree/master/tareas/Tarea_6}{Tarea 6, AM}.
\end{tabular}
\end{table}
Todos los cálculos deben ser a mano.
\begin{enumerate}
%Problema 1
%------------------------------------------------------------------------------------------------------%
%------------------------------------------------------------------------------------------------------%
%------------------------------------------------------------------------------------------------------%
\item Dados los vectores $(1, -1, -2), (-1, -5, -8), (2, 1, 1), (2, -8, -14)$, encuentre la dimensión y una base del espacio generado por ellos.

\res Recordemos el siguiente lema y teorema.
\begin{framed}
    \begin{thml} \label{espacio_generado_igua}
	(Visto en clase) Sea $A=\{a_1,\cdots,a_n\}$ un conjunto de vectores que formar algún subespacio $S\subset \mR^m$. Supongamos que podemos encontrar un vector $a_i$ en $A$ tal que se puede expresar como una combinación lineal de los otros vectores de $A$. Entonces $$\gen(A-\{a_i\})=\gen(A),$$ donde $A-\{a_i\}$ denota todos los elementos de $A$ a excepción de $a_i$.
    \end{thml}
\end{framed}
\begin{framed}
    \begin{thmt} \label{vectores_iid}
	(Visto en clase) Sea $A=\{a_1,\cdots,a_n\}$ un conjunto de vectores que formar algún subespacio $S\subset \mR^m$. Sea $A=[a_1,a_2,\cdots, a_n]$ la matriz de tamaño $m\times n$ formada por los vectores $a_i$ como columnas. Entonces $A$ es linealmente independiente si solo si el sistema homogeno tiene solución única la solución trivial, es decir, si el sistema $Ax=0$ solo tiene la solución trivial $x=0$.
    \end{thmt}
\end{framed}

Denotemos a los vectores como $v_1=(1, -1, -2), v_2=(-1, -5, -8), v_3=(2, 1, 1)$ y $v_4=(2, -8, -14)$. Veamos cual es el espacio generado por los vectores,
\begin{align*}
H=\gen(\{v_1,v_2,v_3,v_4\})=\{ x:x=\alpha_1(1, -1, -2)+\alpha_2(-1, -5, -8)+\alpha_3(2, 1, 1)+\alpha_4(2, -8, -14) \}.
\end{align*}
Considerando el lema \ref{espacio_generado_igua} podemos encontrar una igualdad considerando solo vectores linealmente independientes. Usando el teorema  \ref{vectores_iid} y eliminación gaussiana podemos determinar si son vectores linealmente independiente y de ser el caso encontrar solo los vectores que lo sean.
\begin{align}\label{A_invertible}
\begin{pmatrix}
 1 & -1 & 2 & 2 \\
-1 & -5 & 1 & -8\\
-2 & -8 & 1 & -14
\end{pmatrix}%
\grstep[R_3\rightarrow R_3+2R_1]{R_2 \rightarrow R_2+R_1}
%
\begin{pmatrix}
 1 & -1 & 2 & 2 \\
 0 & -6 & 3 & -6\\
 0 & -10& 5 & -10
\end{pmatrix}%
\grstep[]{R_3 \rightarrow R_3-5R_2/3}
%
\begin{pmatrix}
 1 & -1 & 2 & 2 \\
 0 & -6 & 3 & -6\\
 0 &  0 & 0 & 0
\end{pmatrix}.
\end{align}
Por lo anterior, tenemos que los vectores $v_1$ y  $v_2$ son linealmente independientes, entonces por el lema \ref{espacio_generado_igua}
\begin{align*}
H=\gen(\{v_1,v_2,v_3,v_4\})=\gen(\{v_1,v_2\}).
\end{align*}

\textbf{Por lo que una base para $H$ (el espacio generado por los vectores $\{v_1,v_2,v_3,v_4\}$)  sería considerando los vectores} $v_1=(1, -1, -2), v_2=(-1, -5, -8)$ ya que estos vectores cumple la definición de base \ref{base}: linealmente independiente y cumplen que generan a $H$. Ahora como la definición de la dimensión de un espacio vectorial esta definido como
\begin{framed}
    \begin{thmd} \label{dimension}
	Sea $V$ un espacio vectoria. La dimensión de $V$ es el número de vectores en cualquier base de $V$.
    \end{thmd}
\end{framed}
Entonces ocupando la definición \ref{dimension}, la dimensión del espacio generado por los vectores $\{v_1,v_2,v_3,v_4\}$ es el número de vectores que tienen cualquier base, en este caso como ya encontramos que una base la cual tiene 2 vectores podemos \textbf{concluir que la dimensión es 2}. \  \ \fin
%Problema 2
%------------------------------------------------------------------------------------------------------%
%------------------------------------------------------------------------------------------------------%
%------------------------------------------------------------------------------------------------------%
\item Determine sí los vectores $(3, -1, -1, 1), (1, -6, 3, 0), (0, 5, -1, 2), (1, 0, 1, 0)$ forman una base de $\mR^4$.

\res Recordemos la definición de base.
\begin{framed}
    \begin{thmd} \label{base}
	Sea $V$ un espacio vectorial, $W\subset V$ un subespacio. Una base de $W$ es un subconjunto de $W$ linealmente independiente que genera $W$.
    \end{thmd}
\end{framed}
Veamos si los vectores del problema generan a $\mR^4$. Para ello sea $A$ la matriz generada que tienen como columnas los vectores del problema, es decir,
\begin{align*}
A=\begin{pmatrix}
 3 & 1 & 0 & 1\\
-1 &-6 & 5 & 0\\
-1 & 3 &-1 & 1\\
 1 & 0 & 2 & 0
\end{pmatrix}.
\end{align*}
y sea $r=(r_1,r_2,r_3,r_4)\in \mR^4$. Entonces si el siguiente sistema $Ax=r$ tiene solución única, podemos decir que los vectores generan a $\mR^4.$ Para ver si tiene solución única procedemos a resolver el sistema con reduciendo la matriz aumentada a su forma escalonada.
\begin{align*}
&\left(\begin{array}{cccc|c}
 3 & 1 & 0 & 1&r_1\\
-1 &-6 & 5 & 0&r_2\\
-1 & 3 &-1 & 1&r_3\\
 1 & 0 & 2 & 0&r_4
\end{array} \right)%
\grstep[]{R_1 \leftrightarrow R_4}
%
\left(\begin{array}{cccc|c}
 1 & 0 & 2 & 0&r_4\\
-1 &-6 & 5 & 0&r_2\\
-1 & 3 &-1 & 1&r_3\\
 3 & 1 & 0 & 1&r_1
\end{array} \right)%
\grstep[R_3\rightarrow R_3+R_1]{R_2 \rightarrow R_2+R_1}
%
\left(\begin{array}{cccc|c}
 1 & 0 & 2 & 0&r_4\\
 0 &-6 & 7 & 0&r_2+r_4\\
 0 & 3 & 1 & 1&r_3+r_4\\
 3 & 1 & 0 & 1&r_1
\end{array} \right)\grstep[R_3\rightarrow 2R_3+R_2]{R_4 \rightarrow R_4-3R_1}\\ \\ 
&\left(\begin{array}{cccc|c}
 1 & 0 & 2 & 0&r_4\\
 0 &-6 & 7 & 0&r_2+r_4\\
 0 & 0 & 9 & 2&2(r_3+r_4)+r_2+r_4\\
 0 & 1 &-6 & 1&r_1-r_4
\end{array} \right)%
\grstep[]{R_4 \rightarrow 6R_4+R_2}
%
\left(\begin{array}{cccc|c}
 1 & 0 & 2 & 0&r_4\\
 0 &-6 & 7 & 0&r_2+r_4\\
 0 & 0 & 9 & 2&2r_3+3r_4+r_2\\
 0 & 0 &-29 & 6&6(r_1-r_4)+r_2+r_4
\end{array} \right)%
\grstep[]{R_4 \rightarrow 9R_4+29R_3}\\ \\
&\left(\begin{array}{cccc|c}
 1 & 0 & 2 & 0&r_4\\
 0 &-6 & 7 & 0&r_2+r_4\\
 0 & 0 & 9 & 2&2r_3+3r_4+r_2\\
 0 & 0 & 0 &112&6r_1-5r_4+r_2+29(2r_3+3r_4+r_2).
\end{array} \right)
\end{align*}
Como la forma escalonada de la matriz aumentada tiene $4$ pivotes podemos decir que el sistema tiene solución única, y por lo tanto, los vectores del problema generan a $\mR^4$. 
\begin{framed}
    \begin{thmt} \label{iid}
	(Visto en clase) Sea $S=\{a_1,\cdots, a_n\}\subset\mR^n$. $S$ es linealmente independiente si y solo si la matriz formada con los $a_i$ como columnas es invertirle.
    \end{thmt}
\end{framed}
Es sencillo ver en el procedimiento en dónde se encontro la forma escalonada reducida de la matriz aumentada observar que la forma escalonada de la matriz $A$ tiene $4$ pivotes, entonces que podemos decir que $A$ es invertible y por lo tanto decir que los vectores $(3, -1, -1, 1), (1, -6, 3, 0), (0, 5, -1, 2), (1, 0, 1, 0)$ son linealmente independientes. \\

Y por lo tanto, \textbf{como los vectores $(3, -1, -1, 1), (1, -6, 3, 0), (0, 5, -1, 2), (1, 0, 1, 0)$ son linealmente independientes y además generan a $\mR^4$ podemos concluir (utilizando la definición de base \ref{base}) que los vectores forman una base para $\mR^4$}.\ \ \ \ \fin
%Problema 3
%------------------------------------------------------------------------------------------------------%
%------------------------------------------------------------------------------------------------------%
%------------------------------------------------------------------------------------------------------%
\item Sean $x_1 , x_2 , . . ., x_r$ linealmente independientes en $\mR^n$. Si $A$ es una matriz $n\times n$ invertible, demuestre que $Ax_1 , Ax_2 , . . ., Ax_r$ son linealmente independientes.

\res Para la demostración se ocupara el siguiente lema y la definición del rango.
\begin{framed}
    \begin{thml} \label{ab_nosingular}
	Si $A_{m\times n}$ una matriz no singular y $B_{n\times n}$ una matriz entonces
	$$\rho (AB)=\rho(B)\ \ \text{y}\ \ \mcC(AB)=\mcC(B).$$  
    \end{thml}
\end{framed}
\begin{framed}
    \begin{thmd} \label{rango}
	Sea $A$ una matriz $m\times n$. El rango de $A$ es el número de renglones linealmente independientes de $A$. 
    \end{thmd}
\end{framed}
Con los $x_i's$ podemos formar una matriz $B$ tal que las columnas sean los vectores $x_i$, es decir,
$$B=\begin{pmatrix} x_1& x_2& \cdots & x_r \end{pmatrix}.$$
Ahora, como los vectores $x_i's$ son linealmente independientes podemos decir $\rho(B)=r$ (definición del rango \ref{rango}). Ahora formemos una matriz $C$ tal que sus columnas sean los vectores $Ax_i's,$ es decir,
$$C=\begin{pmatrix} Ax_1& Ax_2& \cdots & Ax_r \end{pmatrix}.$$
Como $A$ es no singular y ocupando el lema \ref{ab_nosingular} tenemos que 
\begin{align*}
\rho(C)=\rho(AB)=\rho(B)=r.
\end{align*}
Como el $\rho(C)=r,$ podemos decir que la matriz $C$ tiene $r$ columnas independientes, pero como las columnas de $C$ son los $Ax_i's$ podemos concluir que $Ax_1 , Ax_2 , . . ., Ax_r$ son linealmente independientes. \fin
%Problema 4
%------------------------------------------------------------------------------------------------------%
%------------------------------------------------------------------------------------------------------%
%------------------------------------------------------------------------------------------------------%
\item Dada la matriz 
\begin{align*}
\begin{pmatrix}
2 & 9 & 4 & -3 & 10\\
1 & 7 & 2 &  3 & -4\\
1 & 6 & 2 &  0 &  2\\
3 & 9 & 6 &  0 & 6
\end{pmatrix}
\end{align*}
encuentre bases para $\mcN(A), \mcC(A)$ y $\mcR(A)$. Encuentra el rango de $A$, la nulidad de $A$ y la dimensión de $\mcR(A)$.

\res Recordemos las definiciones de espacio columna, nulo y renglon.
\begin{framed}
    \begin{thmd} \label{espacio_columna}
    Sea $A$ una matriz de tamaño $m\times n$. El espacio columna de $A$ es 
    $$\mathcal{C}(A)=\{y\in \mR^m|y=Ax\ \ \textbf{para algún}\ x\in\mR^n \}$$
    El rango de una matriz $A$ es la dimensión de $\mcC(A),$ y se denota como $\rho(A)$.
    \end{thmd}
\end{framed} 

\begin{framed}
    \begin{thmd} \label{espacio_renglon}
    Sea $A$ una matriz de tamaño $m\times n$. El espacio renglón de $A$ es 
    $$\mathcal{R}(A)=\{y\in \mR^n|y=A^tx\ \ \textbf{para algún}\ x\in\mR^m \}$$
    \end{thmd}
\end{framed} 

\begin{framed}
    \begin{thmd} \label{espacio_nulo}
    Sea $A$ una matriz de tamaño $m\times n$. El espacio nulo de $A$ es
    $$\mathcal{N}(A)=\{ \textbf{x}\in R^n | A\bf \textbf{x}=\textbf{0}.\}$$
    La nulidad de $A$ es la dimensión de $\mcN(A)$, y se denota $v(A)$.
    \end{thmd}
\end{framed} 
\textbf{El espacio columna de} $A$ por definición \ref{espacio_columna} se puede interpretar como
\begin{align*}
\mcC(A)&=\gen\{a_1, a_2,\cdots , a_n\}, \ \text{donde} \ a_i \text{ son las columas de }A.\\
&=\gen\{\begin{pmatrix}2\\1\\1\\1\end{pmatrix},
\begin{pmatrix} 9\\7\\6\\9\end{pmatrix},
\begin{pmatrix}4\\2\\2\\6\end{pmatrix},
\begin{pmatrix}-3\\3\\0\\0 \end{pmatrix},
\begin{pmatrix}10\\-4\\2\\6 \end{pmatrix} \}.
\end{align*}
Ahora, ocupando el lema \ref{espacio_generado_igua} podemos encontrar vectores linealmente independientes tal que 
$$\mcC(A)=\gen\left\{\begin{pmatrix}2\\1\\1\\1\end{pmatrix},
\begin{pmatrix} 9\\7\\6\\9\end{pmatrix},
\begin{pmatrix}4\\2\\2\\6\end{pmatrix},
\begin{pmatrix}-3\\3\\0\\0 \end{pmatrix},
\begin{pmatrix}10\\-4\\2\\6 \end{pmatrix} \right\}=\gen\{v_i,\cdots,v_j\}$$
donde los $v_i,\cdots,v_j$ son vectores de las columnas de $A$. Entonces ocupando \ref{vectores_iid} y eliminación gaussiana determinaremos cuales columnas de $A$ son linealmente independientes.
\begin{align}\label{matriz_escalonada}
&\begin{pmatrix}
2 & 9 & 4 & -3 & 10\\
1 & 7 & 2 &  3 & -4\\
1 & 6 & 2 &  0 &  2\\
3 & 9 & 6 &  0 & 6
\end{pmatrix}%
\grstep[R_3\rightarrow 2R_3-R_1]{R_2 \rightarrow 2R_2-R_1}
%
\begin{pmatrix}
2 & 9 & 4 & -3 & 10\\
0 & 5 & 0 &  9 & -18\\
0 & 3 & 0 &  3 &  -6\\
3 & 9 & 6 &  0 & 6
\end{pmatrix}%
\grstep[R_4\rightarrow 2R_4/3]{R_3 \rightarrow 5R_3/3}
%
\begin{pmatrix}
2 & 9 & 4 & -3 & 10\\
0 & 5 & 0 &  9 & -18\\
0 & 5 & 0 &  5 &  -10\\
2 & 6 & 4 &  0 & 4
\end{pmatrix}%
\grstep[R_4\rightarrow R_4-R_1]{R_3 \rightarrow R_3-R_2} \nonumber\\ \nonumber\\ 
& \begin{pmatrix}
2 & 9 & 4 & -3 & 10\\
0 & 5 & 0 &  9 & -18\\
0 & 0 & 0 & -4 &  8\\
0 &-3 & 0 &  3 & -6
\end{pmatrix}%
\grstep[]{R_4 \rightarrow 5R_4/3}
%
\begin{pmatrix}
2 & 9 & 4 & -3 & 10\\
0 & 5 & 0 &  9 & -18\\
0 & 0 & 0 & -4 &  8\\
0 &-5 & 0 &  5 & -10
\end{pmatrix}%
\grstep[]{R_4 \rightarrow R_4+R_2}
%
\begin{pmatrix}
2 & 9 & 4 & -3 & 10\\
0 & 5 & 0 &  9 & -18\\
0 & 0 & 0 & -4 &  8\\
0 & 0 & 0 & 14 & -28
\end{pmatrix}%
\grstep[]{R_4 \rightarrow R_4+14R_3/2}\nonumber \\ \nonumber\\
&\begin{pmatrix}
2 & 9 & 4 & -3 & 10\\
0 & 5 & 0 &  9 & -18\\
0 & 0 & 0 & -4 &  8\\
0 & 0 & 0 &  0 & 0
\end{pmatrix}. 
\end{align}
Por lo anterior podemos concluir que las columnas $a_1,a_2$ y $a_4$ son linealmente independientes y también que 
$$\mcC(A)=\gen\left\{\begin{pmatrix}2\\1\\1\\1\end{pmatrix},
\begin{pmatrix} 9\\7\\6\\9\end{pmatrix},
\begin{pmatrix}4\\2\\2\\6\end{pmatrix},
\begin{pmatrix}-3\\3\\0\\0 \end{pmatrix},
\begin{pmatrix}10\\-4\\2\\6 \end{pmatrix} \right\}=\gen\left\{\begin{pmatrix}2\\1\\1\\1\end{pmatrix},
\begin{pmatrix} 9\\7\\6\\9\end{pmatrix},
\begin{pmatrix}-3\\3\\0\\0 \end{pmatrix}\right\}.$$
Entonces como los vectores $\left\{\begin{pmatrix}2\\1\\1\\1\end{pmatrix},
\begin{pmatrix} 9\\7\\6\\9\end{pmatrix},
\begin{pmatrix}-3\\3\\0\\0 \end{pmatrix}\right\}$ son linealmente independiente y generan a $\mcC(A)$ podemos decir que son \textbf{una base para} $\mcC(A)$ (por la definición de base \ref{base}). Y entonces, como el rango de $A$ es la dimensión del espacio columna de $A$ podemos concluir por \ref{dimension} que el \textbf{rango de A es} 4, es decir, $\rho(A)=3.$\\

Ahora, \textbf{el espacio nulo de una matriz} $A$ por definición \ref{espacio_nulo} podemos decir que todas las soluciones del sistema $Ax=0$ son el espació nulo. Entonces resolvemos el sistema utilizando eliminación guassiana y llevan a la $A$ a su forma escalonada reducida (retomamos la matriz en el paso

\begin{align*}
&\begin{pmatrix}
2 & 9 & 4 & -3 & 10\\
1 & 7 & 2 &  3 & -4\\
1 & 6 & 2 &  0 &  2\\
3 & 9 & 6 &  0 & 6
\end{pmatrix}%
\grstep[]{Pasos\ de \ \ref{matriz_escalonada}}
%
\begin{pmatrix}
2 & 9 & 4 & -3 & 10\\
0 & 5 & 0 &  9 & -18\\
0 & 0 & 0 & -4 &  8\\
0 & 0 & 0 &  0 & 0
\end{pmatrix}%
\grstep[]{R_3 \Rightarrow -R_3/4}
%
\begin{pmatrix}
2 & 9 & 4 & -3 & 10\\
0 & 5 & 0 &  9 & -18\\
0 & 0 & 0 &  1 & -2\\
0 & 0 & 0 &  0 & 0
\end{pmatrix}%
\grstep[R_2\Rightarrow R_2-9R_3]{R_1 \Rightarrow R_1+3R_3}\\ \\
&\begin{pmatrix}
2 & 9 & 4 &  0 & 4\\
0 & 5 & 0 &  0 & 0\\
0 & 0 & 0 &  1 &-2\\
0 & 0 & 0 &  0 & 0
\end{pmatrix}%
\grstep[R_2\Rightarrow R_2/2]{R_1 \Rightarrow R_1/2}
%
\begin{pmatrix}
1 & 9/2 & 2 &  0 & 2\\
0 & 1 & 0 &  0 & 0\\
0 & 0 & 0 &  1 &-2\\
0 & 0 & 0 &  0 & 0
\end{pmatrix}%
\grstep[]{R_1 \Rightarrow R_1-9R_2/5}
%
\begin{pmatrix}
1 & 0 & 2 &  0 & 2\\
0 & 1 & 0 &  0 & 0\\
0 & 0 & 0 &  1 &-2\\
0 & 0 & 0 &  0 & 0
\end{pmatrix}
\end{align*}
Por lo tanto, la solución general del sistema $Ax=0$ es $x_1+2x_3+2x_5=0, \ x_2=0, \ x_4-2x_5=0,$ y $x_3, x_5$ libres, esto implica (dando valores a $x_5=0,x_3=1$ y $x_5=1,x_3=0$ para encontrar la solución al sistema general) que 
$$\mcN(A)=\gen\left\{\begin{pmatrix}
-2\\
0\\
0\\
2\\
1
\end{pmatrix},\begin{pmatrix}
-2\\
0\\
1\\
0\\
0
\end{pmatrix} \right\}.$$
Y en conclusión una base para \textbf{el espacio nulo de $A$} es $\left\{\begin{pmatrix}
-2\\
0\\
0\\
2\\
1
\end{pmatrix},\begin{pmatrix}
-2\\
0\\
1\\
0\\
0
\end{pmatrix} \right\}$ y \textbf{la nulidad de $A$} sería la dimensión de $\mcN(A)$ que sería 2.\\

Por último, es espacio renglón de por definición \ref{espacio_renglon} se puede interpretar como
\begin{align*}
\mcR(A)&=\gen(\{a_1^t, a_2^t,\cdots,a_n^t\}) \ \ \text{ donde } a_i^t \text{ son las columnas de} A.\\
&=\gen(\left\{\begin{pmatrix}2\\9\\4\\-3\\10\end{pmatrix}, 
\begin{pmatrix}1\\7\\2\\3\\-4\end{pmatrix},
\begin{pmatrix}1\\6\\2\\0\\2\end{pmatrix},
\begin{pmatrix}3\\9\\6\\0\\6\end{pmatrix}
\right\}.
\end{align*}
Utilizando un razonamiento análogo al que se hizo para encontrar una base del espacio $\mcC(A)$, encontremos los vectores linealmente independientes.
\begin{align*}
\begin{pmatrix}
 2&1 &1&3\\
 9&7 &6&9\\
 4&2 &2&6\\
-3&3 &0&0\\
10&-4&2&6\\
\end{pmatrix}%
\grstep[R_5\Rightarrow R_5-5R_1]{R_3 \Rightarrow R_3-2R_1}
%
\begin{pmatrix}
 2&1 &1&3\\
 9&7 &6&9\\
 0&0 &0&0\\
-3&3 &0&0\\
0&-9&-3&-9\\
\end{pmatrix}%
\grstep[R_4\Rightarrow 2R_4+3R_1]{R_2 \Rightarrow 2R_2-9R_1}
%
\begin{pmatrix}
 2&1 &1&3\\
 0&5 &3&-9\\
 0&0 &0&0\\
 0&9 &3&9\\
0&-9&-3&-9\\
\end{pmatrix}%
\grstep[]{R_2\leftrightarrow R_5}
\end{align*}

\begin{align*}
\begin{pmatrix}
 2&1 &1&3\\
 0&5 &3&-9\\
 0&-9&-3&-9\\
 0&9 &3&9\\
 0&0 &0&0
\end{pmatrix}%
\grstep[R_4\Rightarrow R_4+R_3]{R_3 \Rightarrow 5R_3+9R_2}
%
\begin{pmatrix}
 2&1 &1&3\\
 0&5 &3&-9\\
 0&0&12&-126\\
 0&0 &0&0\\
 0&0 &0&0
\end{pmatrix}.
\end{align*}
Por lo anterior podemos concluir que los vectores $a_1^t, a_2^t$ y $a_3^t$ son linealmente independientes y que 
\begin{align*}
\mcR(A)=\gen(\left\{\begin{pmatrix}2\\9\\4\\-3\\10\end{pmatrix}, 
\begin{pmatrix}1\\7\\2\\3\\-4\end{pmatrix},
\begin{pmatrix}1\\6\\2\\0\\2\end{pmatrix},
\begin{pmatrix}3\\9\\6\\0\\6\end{pmatrix}
\right\}=\gen(\left\{\begin{pmatrix}2\\9\\4\\-3\\10\end{pmatrix}, 
\begin{pmatrix}1\\7\\2\\3\\-4\end{pmatrix},
\begin{pmatrix}1\\6\\2\\0\\2\end{pmatrix}\right\}.
\end{align*}
Entonces como los vectores $\left\{\begin{pmatrix}2\\9\\4\\-3\\10\end{pmatrix}, 
\begin{pmatrix}1\\7\\2\\3\\-4\end{pmatrix},
\begin{pmatrix}1\\6\\2\\0\\2\end{pmatrix}\right\}$ son linealmente independiente y generan a $\mcR(A)$ podemos decir que \textbf{son una base para} $\mcR(A)$ (por la definición de base \ref{base}). Y entonces, la \textbf{dimensión del espacio renglon de $A$ es} 3.\ \ \ \fin

Nota: Se muestra que $\mcC(A)=\mcR(A)$ y que $\mcC(A)+\mcN(A)=n.$
%Problema 5
%------------------------------------------------------------------------------------------------------%
%------------------------------------------------------------------------------------------------------%
%------------------------------------------------------------------------------------------------------%
\item Dadas las bases $\mathcal{B}= \{(3, -1, -1), (1, -6, 3), (0, 5, -1)\}$ y $\mcC = \{(3, 0, 6), (2, 2, -4), (1, -2, 3)\}$ de $\mR^3$ , encuentre la matriz de cambio de base
de $\mathcal{B}$ a $\mcC$ y la matriz de cambio de base de $\mcC$ a $\mathcal{B}$. Encuentre las coordenadas del vector $(-2, 7, 1)$ con respecto a cada una de las bases.

\res Recordemos la definición de la matriz de cambio de base(\textit{vista en clase}).
\begin{framed}
    \begin{thmd} \label{matriz_base}
	Si $(v_i)_{B_2}$ es el vector (en $\mR^n$) de coordenadas de $v_i$ con respecto a la base $B^2$, $v_{B_1}$ es el vector de coordenadas de v con respecto a la base $B_1$ y $v_{B_2}$ es el vector de coordenadas de v con respecto a la base $B_2$, entonces existe una matrix invertible $A$ dada por $A = ((v_1 )_{B_2} \cdots (v_2)_{B_2} )$ ( es decir, tiene los vectores de coordenadas como columnas) tal que $v_{B_2} = Av_{B_1}$.
	$A$ es la matriz de cambio de coordenadas de la base $B_1$ a la base $B_2$.
    \end{thmd}
\end{framed}
\begin{framed}
    \begin{thmt} \label{matriz_base_inversa}
	Si $A$ es la matriz de cambio de coordenadas de la base $\mathcal{B}_1$ a la base $\mathcal{B}_2$ entonces $A^{-1}$ es la matriz de cambio de coordenadas de la base $\mathcal{B}_2$ a la base $\mathcal{B}_1.$
    \end{thmt}
\end{framed}
Con la definición \ref{matriz_base}, podemos encontrar la matriz de cambio de base de $\mathcal{B}$ a la base canónica de $\mR^3$ denotemosla como la matriz $P$ , y la matriz de cambio de la base $\mathcal{C}$ a la base canónica denotermosla como la matriz $Q$. Ocupando \ref{matriz_base_inversa} podemos demostrar sea $x\in\mR^3$, entonces
\begin{align*}
\left.\begin{array}{cc}
x=Px'\\
x=Qx''
\end{array}\right\}\Rightarrow Px'=Qx''\Rightarrow x''=(Q^{-1}P)x'.
\end{align*}
Con lo anterior, ya podemos calcular fácilmente la matriz de cambio de bases:
\begin{align*}
P=\begin{pmatrix}
 3 & 1 & 0\\
-1 &-6 & 5\\
-1 & 3 & -1
\end{pmatrix}, \ \ \ Q=\begin{pmatrix}
 3& 2 & 1\\
 0& 2 &-2\\
 6&-4 & 3
\end{pmatrix}.
\end{align*}
Calculemos la matriz inversa de $Q$, utilizando la matriz aumentada.
\begin{align*}
&\left(\begin{array}{ccc|ccc}
 3& 2 & 1& 1 & 0 & 0\\
 0& 2 &-2& 0 & 1 & 0\\
 6&-4 & 3& 0 & 0 & 1
\end{array}\right) %
\grstep[R_3\Rightarrow R_3-2R_1]{R_1 \Rightarrow R_1-R_2}
%
\left(\begin{array}{ccc|ccc}
 3& 0 & 3& 1 & -1 & 0\\
 0& 2 &-2& 0 & 1 & 0\\
 0&-8 & 1&-2 & 0 & 1
\end{array}\right) %
\grstep[]{R_3 \Rightarrow R_3+3R_2}\\ \\
&\left(\begin{array}{ccc|ccc}
 3& 0 & 3& 1 & -1 & 0\\
 0& 2 &-2& 0 & 1 & 0\\
 0& 0 &-7&-2 &4 & 1
\end{array}\right)%
\grstep[R_2\Rightarrow R_2/2]{R_3 \Rightarrow -R_3/7}
%
\left(\begin{array}{ccc|ccc}
 3& 0 & 3& 1 & -1 & 0\\
 0& 1 &-1& 0 & \frac{1}{2} & 0\\
 0& 0 & 1&\frac{2}{7} &-\frac{4}{7}& -\frac{1}{7}
\end{array}\right)%
\grstep[R_1\Rightarrow R_1/3-R_3]{R_2 \Rightarrow R_2+R_3}\\ \\
&\left(\begin{array}{ccc|ccc}
 1& 0 & 0& \frac{-6+7}{21} & \frac{12-7}{21} &\frac{1}{7}\\
 0& 1 & 0& \frac{2}{7} & \frac{-8+7}{14} & -\frac{1}{7}\\
 0& 0 & 1&\frac{2}{7} &-\frac{4}{7}& -\frac{1}{7}
\end{array}\right)=\left(\begin{array}{ccc|ccc}
 1& 0 & 0& \frac{1}{21} & \frac{5}{21} &\frac{1}{7}\\
 0& 1 & 0& \frac{2}{7} & -\frac{1}{14} & -\frac{1}{7}\\
 0& 0 & 1&\frac{2}{7} &-\frac{4}{7}& -\frac{1}{7}
\end{array}\right)\Rightarrow Q^{-1}=\left(\begin{array}{ccc}
 \frac{1}{21} & \frac{5}{21} &\frac{1}{7}\\
 \frac{2}{7} & -\frac{1}{14} & -\frac{1}{7}\\
 \frac{2}{7} &-\frac{4}{7}& -\frac{1}{7}
\end{array}\right)
\end{align*}.
Por lo tanto, \textbf{la matriz de cambio de base de }$\mathcal{B}$ a $\mathcal{C}$ es 
\begin{align*}
Q^{-1}P=\left(\begin{array}{ccc}
 \frac{1}{21} & \frac{5}{21} &\frac{1}{7}\\
 \frac{2}{7} & -\frac{1}{14} & -\frac{1}{7}\\
 \frac{2}{7} &-\frac{4}{7}& -\frac{1}{7}
\end{array}\right)\begin{pmatrix}
 3 & 1 & 0\\
-1 &-6 & 5\\
-1 & 3 & -1
\end{pmatrix}=\begin{pmatrix}
\frac{3-5-3}{21} & \frac{1-30+9}{21} & \frac{25-3}{21}\\
\frac{12+1+2}{14} & \frac{4+6-6}{14} & \frac{-5+2}{14}\\
\frac{6+4+1}{7} & \frac{2+24-3}{7} & \frac{-20+1}{7}
\end{pmatrix}=\begin{pmatrix}
-\frac{5}{21} & -\frac{20}{21} & \frac{22}{21}\\
\frac{15}{14} & \frac{2}{7} & -\frac{3}{14}\\
\frac{11}{7} & \frac{23}{7} & -\frac{19}{7}
\end{pmatrix}.
\end{align*}
Ahora, utilizando el teorema \ref{matriz_base_inversa} podemos calcular la matriz de cambio de base de $\mathcal{C}$ a $\mathcal{B}$ solamente calculando la inversad de la matriz de cambio de base de $\mathcal{B}$ a $\mathcal{C}$, es decir,
\begin{align*}
\left(\begin{array}{ccc|ccc}
-\frac{5}{21} & -\frac{20}{21} & \frac{22}{21} & 1 & 0 &0\\
\frac{15}{14} & \frac{2}{7} & -\frac{3}{14}&0&1&0 \\
\frac{11}{7} & \frac{23}{7} & -\frac{19}{7}&0&0&1
\end{array}\right)%
\grstep[]{R_1 \Rightarrow -21R_1/5}
%
\left(\begin{array}{ccc|ccc}
           1 & 4 & -\frac{22}{5} & -\frac{21}{5} & 0 &0\\
\frac{15}{14} & \frac{2}{7} & -\frac{3}{14}&0&1&0 \\
\frac{11}{7} & \frac{23}{7} & -\frac{19}{7}&0&0&1
\end{array}\right)%
\grstep[R_3\Rightarrow R_3-11R_1/7]{R_2 \Rightarrow R_2-15R_1/14}\\ \\
\left(\begin{array}{ccc|ccc}
   1 & 4 & -\frac{22}{5} & -\frac{21}{5} & 0 &0\\
   0 &-4 & -\frac{9}{2}  &\frac{9}{2}&1&0 \\
   0 &-3 &  \frac{21}{5} &\frac{33}{5}&0&1
\end{array}\right)%
\grstep[]{R_2 \Rightarrow -R_2/4}
%
\left(\begin{array}{ccc|ccc}
   1 & 4 & -\frac{22}{5} & -\frac{21}{5} & 0 &0\\
   0 & 1 & -\frac{9}{8}  & -\frac{9}{8} &-\frac{1}{4}&0 \\
   0 &-3 &  \frac{21}{5} & \frac{33}{5} &0&1
\end{array}\right)%
\grstep[]{R_3 \Rightarrow R_3+3R_2}\\ \\ 
\left(\begin{array}{ccc|ccc}
   1 & 4 & -\frac{22}{5} & -\frac{21}{5} & 0 &0\\
   0 & 1 & -\frac{9}{8}  & -\frac{9}{8} &-\frac{1}{4}&0 \\
   0 & 0 &  \frac{33}{40} & \frac{129}{40} &-\frac{3}{4}&1
\end{array}\right)%
\grstep[]{R_3 \Rightarrow 40R_3/33}
%
\left(\begin{array}{ccc|ccc}
   1 & 4 & -\frac{22}{5} & -\frac{21}{5} & 0 &0\\
   0 & 1 & -\frac{9}{8}  & -\frac{9}{8} &-\frac{1}{4}&0 \\
   0 & 0 &  1 & \frac{43}{11} &-\frac{10}{11}&\frac{40}{33}
\end{array}\right)%
\grstep[R_2 \Rightarrow R_2+8R_3/9]{R_1 \Rightarrow R_1+22R_3/5}\\ \\
\left(\begin{array}{ccc|ccc}
   1 & 4 &  0 & 13& -4 &\frac{16}{3}\\
   0 & 1 &  0 &\frac{36}{11} &-\frac{14}{11}& \frac{15}{11} \\
   0 & 0 &  1 & \frac{43}{11} &-\frac{10}{11}&\frac{40}{33}
\end{array}\right)%
\grstep[]{R_1 \Rightarrow R_1-4R_2}%
\left(\begin{array}{ccc|ccc}
   1 & 0 &  0 &-\frac{1}{11}&\frac{12}{11} &-\frac{4}{33}\\
   0 & 1 &  0 &\frac{36}{11} &-\frac{14}{11}& \frac{15}{11} \\
   0 & 0 &  1 & \frac{43}{11} &-\frac{10}{11}&\frac{40}{33}
\end{array}\right).
\end{align*}
Es decir, \textbf{la matriz de cambio de base de } $\mathcal{C}$ a $\mathcal{B}$ es
\begin{align*}
\left(\begin{array}{ccc}
-\frac{1}{11}&\frac{12}{11} &-\frac{4}{33}\\
\frac{36}{11} &-\frac{14}{11}& \frac{15}{11} \\
 \frac{43}{11} &-\frac{10}{11}&\frac{40}{33}
\end{array}\right).
\end{align*}
\textbf{Ejemplo númerico.} Para determinar el vector $(-2,7,1)$ en la base $\mathcal{C}$, tenemos que 
\begin{align*}
x=Qx''\Rightarrow Q^{-1}x=x''.
\end{align*}
Entonces, el vector $(-2,7,1)$ en la base $\mathcal{C}$ se puede calcular como
\begin{align*}
(-2,7,1)_{\mathcal{C}}=\left(\begin{array}{ccc}
 \frac{1}{21} & \frac{5}{21} &\frac{1}{7}\\
 \frac{2}{7} & -\frac{1}{14} & -\frac{1}{7}\\
 \frac{2}{7} &-\frac{4}{7}& -\frac{1}{7}
\end{array}\right)\begin{pmatrix}
-2\\
7\\
1
\end{pmatrix}=\begin{pmatrix}
\frac{-2+35+3}{21}\\ \\
\frac{-28-7-2}{14}\\ \\
\frac{-4-28-1}{7}
\end{pmatrix}=\begin{pmatrix}
\frac{12}{7}\\
-\frac{17}{14}\\
-\frac{33}{7}
\end{pmatrix}.
\end{align*}
Por lo tanto, el vector en la base $\mathcal{B}$ es
\begin{align*}
(-2,7,1)_{\mathcal{B}}=\left(\begin{array}{ccc}
-\frac{1}{11}&\frac{12}{11} &-\frac{4}{33}\\
\frac{36}{11} &-\frac{14}{11}& \frac{15}{11} \\
 \frac{43}{11} &-\frac{10}{11}&\frac{40}{33}
\end{array}\right)
\begin{pmatrix}
\frac{12}{7}\\
-\frac{17}{14}\\
-\frac{33}{7}
\end{pmatrix}=\begin{pmatrix}
\frac{-12}{11(7)}+\frac{-12(17)}{11(14)}+\frac{4(33)}{33(7)}\\
\frac{-12(36)}{11(7)}+\frac{-14(17)}{11(14)}+\frac{-15(33)}{11(7)}\\
\frac{-12(43)}{11(7)}+\frac{-10(-17)}{11(14)}+\frac{-40(33)}{33(7)}\\
\end{pmatrix}=\begin{pmatrix}
-\frac{10}{11}\\
\frac{8}{11}\\
\frac{23}{11}
\end{pmatrix}.\ \ \ \finf
\end{align*}
%Problema 6
%------------------------------------------------------------------------------------------------------%
%------------------------------------------------------------------------------------------------------%
%------------------------------------------------------------------------------------------------------%
\item Demuestre que dados cualquier conjunto finito $S$ de vectores en $\mR^n$, existe una matriz $A$ tal que $\mcN(A)=\gen (S)$. De un ejemplo de esto en $\mR^3$ y justifíquelo, es decir, encuentre un método para encontrar $A$.

\res 
\begin{framed}
    \begin{thmt} \label{subespacio_dimension}
	Si $V$ un espacio vectorial y $W\subset V$ un subespacio. Si $dim W=dim V$, entonces $V=W.$
    \end{thmt}
\end{framed}
Sea $S$ el conjunto de vectores $\{s_1,s_2,\cdots ,s_r\},\  \ s_j\in \mR^n,$ y sea $\{v_{1}, v_{2}, \cdots v_k \}$ los vectores linealmente independiente del conjunto $S$, en donde  todos los $v_j's$ pertenecen al conjunto de los vectores $\{s_1,s_2,\cdots ,s_r\}$ y $k\leq r$ (por construcción). Entonces ocupando el teorema  \ref{espacio_generado_igua} tenemos que 
\begin{align}\label{generado}
\gen(S)=\gen(\{v_{1}, v_{2}, \cdots, v_k \}).
\end{align}
Ahora, sea $A$ una matriz de tamaño de $n\times k$ 
$$A=\begin{pmatrix}
(a_1)&(a_2)&\cdots&(a_n)
\end{pmatrix}=\begin{pmatrix}
a_{11}&a_{21}&\cdots&a_{n1}\\
a_{12}&a_{22}&\cdots&a_{n2}\\
\vdots&\vdots&\vdots&\vdots\\
a_{1k}&a_{2k}&\cdots&a_{nk}\\
\end{pmatrix}$$
 tal que los vectores $\{v_{1}, v_{2}, \cdots, v_k \}$ son todas las posibles soluciones (independientes) de 
$$Ax=0,$$
es decir, para cada vector $v_i=\begin{pmatrix}
v_{i1}\\
v_{i2}\\
\vdots\\
v_{in}
\end{pmatrix}$ se cumple que 
\begin{align*}
Av_i=\begin{pmatrix}
a_{11}&a_{21}&\cdots&a_{n1}\\
a_{12}&a_{22}&\cdots&a_{n2}\\
\vdots&\vdots&\vdots&\vdots\\
a_{1k}&a_{2k}&\cdots&a_{nk}\\
\end{pmatrix}\begin{pmatrix}
v_{i1}\\
v_{i2}\\
\vdots\\
v_{in}
\end{pmatrix}=\bf 0.
\end{align*}
Como se cumple para todos los vectores  $\{v_{1}, v_{2}, \cdots, v_k \}$ es equivalente a decir que 
\begin{align*}
&A((v_1)\ (v_2)\ \cdots\ (v_k))=0\Leftrightarrow\\ \\
&(A((v_1)\ (v_2)\ \cdots\ (v_k))^{t}=0^{t}\Leftrightarrow\\ \\
&((v_1)^t\ (v_2)^t\ \cdots\ (v_n)^t)A^{t}=0\Leftrightarrow\\ \\
&\begin{pmatrix}
v_{11}& v_{12}& \cdots&v_{1n}\\
v_{21}&v_{22}& \cdots&v_{2k} \\
\vdots&\vdots&\vdots&\vdots \\
v_{k1}&v_{k2} & \cdots&v_{kn}& 
\end{pmatrix}\begin{pmatrix}
a_{11}&a_{12}& \cdots&a_{1k}\\
a_{21}&a_{22}& \cdots&a_{2k} \\
\vdots&\vdots&\vdots&\vdots \\
a_{n1}&a_{n2} & \cdots&a_{nk}
\end{pmatrix}
=
\begin{pmatrix}
0&0& \cdots&0\\
0&0& \cdots&0\\
\vdots&\vdots&\vdots&\vdots \\
0&0& \cdots&0\\
\end{pmatrix}
\end{align*}
Por lo tanto, $A$ existe y se puede escribir como el conjunto solución del sistema homogéneo formado por los vectores independientes $\{v_{1}, v_{2}, \cdots, v_k \}$. Ahora demostremos que $\mcN(A)=\gen(S),$ para ello observemos que como $\{v_{1}, v_{2}, \cdots, v_k \}$ son soluciones del sistema homogéneo $Ax=0$ esto implica que para cualquier $w=\alpha_1v_{1}+\alpha_2v_2+\cdots \alpha_kv_{k}$ entonces  
\begin{align*}
Aw&=A(\alpha_1v_{1}+\alpha_2v_2+\cdots \alpha_kv_{k})=
A(\alpha_1v_{1})+A(\alpha_2v_{2})+\cdots+A(\alpha_kv_{k})\\
&=\alpha_1(Av_1)+\alpha_2(Av_2)+\cdots+\alpha_k(Av_k)=0.
\end{align*}
Entonces cualquier $w$ que se pueda expresar como combinación lineal de los vectores $\{v_{1}, v_{2}, \cdots, v_k \}$ este cumple que $w\in \mcN(A)$, entonces podemos decir que 
$$\gen(\{v_{1}, v_{2}, \cdots, v_k \})\subset \mcN(A).$$
Y por construcción como $dim(\gen(\{v_{1}, v_{2}, \cdots, v_k \})=k$ y $dim(\mcN(A))=k$, es decir, $dim(\gen(\{v_{1}, v_{2}, \cdots, v_k \})=dim(\mcN(A))$. Ocupando el teorema \ref{subespacio_dimension} podemos concluir que
$$\gen(\{v_{1}, v_{2}, \cdots, v_k \})= \mcN(A).$$
Y por lo tanto, ocupando el resultado anterior y \ref{generado}
$$\mcN(A)=\gen(\{v_{1}, v_{2}, \cdots, v_k \})=\gen(S)\Leftrightarrow \mcN(A)=\gen(S).$$

\textbf{Ejemplo númerico.} Sea $S=\left\{\begin{pmatrix}
2\\
-1\\
1
\end{pmatrix},\begin{pmatrix}
2\\
-3\\
2
\end{pmatrix}\right\}$. Veamos si los vectores son linealmente independiente, para eso ocuparemos el teorema \ref{vectores_iid}
\begin{align*}
\begin{pmatrix}
 2 & 2\\
-1 & -3\\
1 & 2
\end{pmatrix}%
\grstep[R_3 \rightarrow 2R_3-R_1]{R_2 \rightarrow 2R_2+R_1}
%
\begin{pmatrix}
 2 & 2\\
 0 &-4\\
 0 & 4
\end{pmatrix}%
\grstep[]{R_3 \rightarrow R_3+R_2}
%
\begin{pmatrix}
 2 & 2\\
 0 & 1\\
 0 & 0
\end{pmatrix}
\end{align*}
Lo anterior implica que los vectores son linealmente independientes, y por lo tanto \\
$dim\left(\gen\left\{\begin{pmatrix}
2\\
-1\\
1
\end{pmatrix},\begin{pmatrix}
2\\
-3\\
2
\end{pmatrix}\right\}\right)=2$. Ahora, encontremos la matriz $A$ tal que los vectores $\left\{\begin{pmatrix}
2\\
-1\\
1
\end{pmatrix},\begin{pmatrix}
2\\
-3\\
2
\end{pmatrix}\right\}$ son soluciones del sistema homogéneo 
$Ax=0.$ Utilizando la metodología que se describió anteriormente para encontrar las columnas de $A$, tenemos que encontrar las soluciones del sistema
\begin{align*}
\begin{pmatrix}
2 & -1 & 1\\
2 & -3 & 2
\end{pmatrix}\begin{pmatrix}
a_{11}\\
a_{12}\\
a_{13}
\end{pmatrix}=0.
\end{align*}
Ocupando reducción gaussiana resolvemos el sistema anterior
\begin{align*}
\begin{pmatrix}
2 & -1 & 1\\
2 & -3 & 2
\end{pmatrix}%
\grstep[]{R_2 \rightarrow R_2-R_1}
%
\begin{pmatrix}
2 & -1 & 1\\
0 & -2 & 1
\end{pmatrix}%
\grstep[R_1 \rightarrow R_1+R_2]{R_2 \rightarrow -R_2/2}
%
\begin{pmatrix}
2 & 0 & \frac{1}{2}\\
0 & 1 & -\frac{1}{2}
\end{pmatrix}%
\grstep[]{R_1 \rightarrow R_1/2}
%
\begin{pmatrix}
1 & 0 & \frac{1}{4}\\
0 & 1 & -\frac{1}{2}
\end{pmatrix}.
\end{align*}
Por lo tanto, sistema tiene la solución general $a=a_{13}\begin{pmatrix}
-\frac{1}{4}\\
\frac{1}{2}\\
1
\end{pmatrix}$, si hacemos $a_{13}=1,$ entonces una solución es $a=\begin{pmatrix}
-\frac{1}{4}\\
\frac{1}{2}\\
1
\end{pmatrix}$.
Por lo tanto, la matriz $A$ que tiene como solución los vectores $\left\{\begin{pmatrix}
2\\
-1\\
1
\end{pmatrix},\begin{pmatrix}
2\\
-3\\
2
\end{pmatrix}\right\}$ en el sistema homogéneo $Ax=0$  es 
$$A=\begin{pmatrix}
-\frac{1}{4}&\frac{1}{2}&1\\
0&0&0\\
0&0&0
\end{pmatrix}.$$
Observemos que $\dim \left(\mcN(A) \right)=2=dim\left(\gen\left\{\begin{pmatrix}
2\\
-1\\
1
\end{pmatrix},\begin{pmatrix}
2\\
-3\\
2
\end{pmatrix}\right\}\right)$. Ahora, como cualquier vector que sea combinación lineal de los conjunto S, es decir, $w=\alpha \begin{pmatrix}
2\\
-1\\
1
\end{pmatrix}+\beta\begin{pmatrix}
2\\
-3\\
2
\end{pmatrix}$ esto implica que 
\begin{align*}
&\begin{pmatrix}
-\frac{1}{4}&\frac{1}{2}&1\\
0&0&0\\
0&0&0
\end{pmatrix}w=\begin{pmatrix}
-\frac{1}{4}&\frac{1}{2}&1\\
0&0&0\\
0&0&0
\end{pmatrix}\left(\alpha \begin{pmatrix}
2\\
-1\\
1
\end{pmatrix}+\beta\begin{pmatrix}
2\\
-3\\
2
\end{pmatrix} \right)=\alpha\begin{pmatrix}
-\frac{1}{4}&\frac{1}{2}&1\\
0&0&0\\
0&0&0
\end{pmatrix}\begin{pmatrix}
2\\
-1\\
1
\end{pmatrix}+\beta\begin{pmatrix}
-\frac{1}{4}&\frac{1}{2}&1\\
0&0&0\\
0&0&0
\end{pmatrix}\begin{pmatrix}
2\\
-3\\
2
\end{pmatrix}\\ \\
&=0.
\end{align*}
Entonces, cualquier $w\in\gen\left\{\begin{pmatrix}
2\\
-1\\
1
\end{pmatrix},\begin{pmatrix}
2\\
-3\\
2
\end{pmatrix}\right\}\Rightarrow w\in \mcN\left(\begin{pmatrix}
-\frac{1}{4}&\frac{1}{2}&1\\
0&0&0\\
0&0&0
\end{pmatrix}\right)$, y como $\dim \left(\mcN(A) \right)=dim\left(\gen\left\{\begin{pmatrix}
2\\
-1\\
1
\end{pmatrix},\begin{pmatrix}
2\\
-3\\
2
\end{pmatrix}\right\}\right)$ por el teorema \ref{subespacio_dimension} podemos concluir 
$$\gen\left\{\begin{pmatrix}
2\\
-1\\
1
\end{pmatrix},\begin{pmatrix}
2\\
-3\\
2
\end{pmatrix}\right\} =\mcN\left(\begin{pmatrix}
-\frac{1}{4}&\frac{1}{2}&1\\
0&0&0\\
0&0&0
\end{pmatrix}\right).\ \ \ \finf$$
%Problema 7
%------------------------------------------------------------------------------------------------------%
%------------------------------------------------------------------------------------------------------%
%------------------------------------------------------------------------------------------------------%
\item Sea $F$ la matriz por bloques
\begin{align*}
F=\begin{pmatrix}
A & B\\
0 & E
\end{pmatrix}
\end{align*}
Demuestre que $\rho(F)\geq \rho(A)+\rho(E).$ De un ejemplo donde la desigualdad sea estricta.

\res Recordemos los siguientes teoremas vistos en clase. 
\begin{framed}
    \begin{thmt} \label{submatriz}
	Si $A$ es una matriz y $B$ es una submatriz de $A$ entonces$$\rho(B)\leq \rho(A).$$
    \end{thmt}
\end{framed}

\begin{framed}
    \begin{thmt} \label{dimension_2}
	Sea $V$ un espacio vectorial y $U,W$ son subespacios de $V$ entonces 
	$$\dim(U+W)=\dim(U)+\dim(W)-\dim(U\cap W)$$
    \end{thmt}
\end{framed}

\begin{framed}
    \begin{thmt} \label{matrices}
	Sean $A, \ B$ matrices con el mismo número de renglones. Entonces
	$$\mcC((A\ B))=\mcC(A)+\mcC(B).$$
    \end{thmt}
\end{framed}
Ocupando el teorema \ref{matrices} tenemos que
\begin{align*}
\mcC\begin{pmatrix}
A&B\\
0&E
\end{pmatrix}=\mcC\begin{pmatrix}
A\\
0
\end{pmatrix}+\mcC\begin{pmatrix}
B\\
E
\end{pmatrix}.
\end{align*}
Ahora, ocupando el teorema \ref{dimension_2} tenemos que el rango 
\begin{align*}
\dim\left(\mcC\begin{pmatrix}
A&B\\
0&E
\end{pmatrix}\right)&=\dim\left(\mcC\begin{pmatrix}
A\\
0
\end{pmatrix}\right)+\dim\left(\mcC\begin{pmatrix}
B\\
E
\end{pmatrix}\right)-\dim\left(\mcC\begin{pmatrix}
A\\
0
\end{pmatrix}\cap \begin{pmatrix}
B\\
E
\end{pmatrix} \right)\\
\end{align*}
Por definición la dimensión de cualquier subespacio vectorial es mayor o igual a cero, entonces esto implica que lo anterior es
\begin{align}\label{ejer7}
\dim\left(\mcC\begin{pmatrix}
A&B\\
0&E
\end{pmatrix}\right)\geq  \dim\left(\mcC\begin{pmatrix}
A\\
0
\end{pmatrix}\right)+\dim\left(\mcC\begin{pmatrix}
B\\
E
\end{pmatrix}\right)=\rho\left(\begin{pmatrix}
A\\
0
\end{pmatrix}\right)+\rho\left(\begin{pmatrix}
B\\
E
\end{pmatrix}\right)
\end{align}
Ahora, ocupando el teorema \ref{submatriz} y como $A$ es submatriz de $
\begin{pmatrix}
A\\
0
\end{pmatrix}
$ y también como $E$ es submatriz de 
$
\begin{pmatrix}
B\\
E
\end{pmatrix}
$ podemos usar que 
$$\rho\left(\begin{pmatrix}
A\\
0
\end{pmatrix}\right)\geq \rho (A) \ \ \ \ \text{y}\ \ \ \ \rho\left(\begin{pmatrix}
B\\
E
\end{pmatrix}\right)\geq \rho(E).$$
Por lo tanto, sustituyendo en \ref{ejer7} podemos concluir que 
\begin{align*}
\rho(F)=\dim\left(\mcC\begin{pmatrix}
A&B\\
0&E
\end{pmatrix}\right)\geq \rho\left(\begin{pmatrix}
A\\
0
\end{pmatrix}\right)+\rho\left(\begin{pmatrix}
B\\
E
\end{pmatrix}\right)\geq \rho(A)+\rho(E), \Rightarrow \rho(F)\geq \rho(A)+\rho(E).\ \ \ \finf
\end{align*}
\textbf{Ejemplo númerico} Un ejemplo en donde se cumple la desigualdad, sea $A=\begin{pmatrix}
1& 0\\
0&0
\end{pmatrix}, \ B=\begin{pmatrix}
0&0&0\\
1&0&0
\end{pmatrix}$ y $E=\begin{pmatrix}
1&1&2\\
0&1&1
\end{pmatrix}$ entonces por definición del rango de una matriz podemos observar que 
$$\rho(A)=1 \ \ \ \text{y}\ \ \ \rho(E)=2.$$
Ahora calculemos el rango de la matriz $
F=\begin{pmatrix}
A & B\\
0 & E
\end{pmatrix}$, para eso calculemos la forma reducido la de matriz
\begin{align*}
\begin{pmatrix}
A & B\\
0 & E
\end{pmatrix}=\begin{pmatrix}
1&0&0&0&0\\
0&0&1&0&0\\
0&0&1&1&2\\
0&0&0&1&1
\end{pmatrix}%
\grstep[R_4 \rightarrow R_4-R_3]{R_3 \rightarrow R_3-R_2}
%
\begin{pmatrix}
1&0&0&0&0\\
0&0&1&0&0\\
0&0&0&1&2\\
0&0&0&0&-1
\end{pmatrix}.
\end{align*}
Esto implica que $\rho(F)=4$, y por lo que se muestra que $\rho(F)=4>3 =\rho(A)+\rho(E),$ es decir, $\rho(F)>\rho(A)+\rho(E).$
%Problema 8
%------------------------------------------------------------------------------------------------------%
%------------------------------------------------------------------------------------------------------%
%------------------------------------------------------------------------------------------------------%
\item Sea $A$ una matriz cuadrada $n\times n$. Si $\rho(A^k)=\rho(A^{k+1})$ para algún $k\geq 1,$ demuestre que $\rho(A^{k+1})=\rho(A^{k+2})$.

\res Se ocuparan los siguientes teoremas vistos en clase.
\begin{framed}
    \begin{thmt} \label{rho_multiplicacion}
	Sea $A$ una matriz $m\times n$ y $B$ una matriz $n\times p$. Entonces $$\rho(AB)=\rho(B)-dim(\mcN(A)\cap\mcC(B)).$$
    \end{thmt}
\end{framed}
\begin{framed}
    \begin{thmt} \label{rho_columna}
	Sean $A$ y $B$ matrices tales que $AB$ está bien definido. Entonces
	\begin{itemize}
	\item[i)] Si $\rho(AB)=\rho(A)$ entonces $\mcC(AB)=\mcC(A)$.
	\end{itemize}
    \end{thmt}
\end{framed}
Por definición del problema tenemos que $\rho(A^k)=\rho(A^{k+1})$ para algún $k\geq 1,$ por el teorema \ref{rho_columna} podemos decir que $\mcC(A^{k+1})=\mcC(A^k).$ Y ahora ocupando el teorema \ref{rho_multiplicacion} y como sabemos que $\rho(A^k)=\rho(A^{k+1})$ tenemos que
\begin{align}\label{dim_cero}
\rho(A^{k+1})=\rho(AA^{k})=\rho(A^k)-dim(\mcN(A)\cap \mcC(A^k)).\ \ \Rightarrow dim(\mcN(A)\cap \mcC(A^k))=0.
\end{align}
Ahora, ocupando nuevamente el teorema \ref{rho_multiplicacion}  y como sabemos que $\mcC(A^{k+1})=\mcC(A^k)$ tenemos que 
\begin{align*}
\rho(A^{k+2})&=\rho(AA^{k+1})=\rho(A^{k+1})-dim(\mcN(A)\cap \mcC(A^{k+1}))\\
&=\rho(A^{k+1})-dim(\mcN(A)\cap \mcC(A^{k})).
\end{align*}
Por lo tanto, como $dim(\mcN(A)\cap \mcC(A^k))=0$ (por \ref{dim_cero}) podemos concluir que 
$$\rho(A^{k+2})=\rho(A^{k+1}).\ \ \ \finf$$
%Problema 9
%------------------------------------------------------------------------------------------------------%
%------------------------------------------------------------------------------------------------------%
%------------------------------------------------------------------------------------------------------%
\item Sea $A$ una matriz cuadrada $n\times n$. Demuestre que existe un $1\leq p\leq n$ tal que 
\begin{align*}
\rho(A)> \rho (A^2) >\cdots >\rho(A^p)=\rho(A^{p+1})=\cdots
\end{align*}

\res Recordemos el teorema visto en clase:
\begin{framed}
    \begin{thmt} \label{contigencia_nulo}
	Sean $A$ y $B$ matrices tales que $AB$ está bien definido. Entonces $\rho(AB)\leq \rho(A)$.
    \end{thmt}
\end{framed}
\begin{framed}
    \begin{thmt} \label{nulo}
	$\rho(A)=0$ si y solo si $A=0$.
    \end{thmt}
\end{framed}

Ocupando el teorema \ref{contigencia_nulo}, tenemos que
$$\rho(A)\geq \rho(A^2),\ \ \rho(A^2)\geq \rho(A^3), \cdots,\rho(A^p)\geq \rho(A^{p+1}),$$
es decir,

$$\rho(A)\geq \rho(A^2) \geq \rho(A^3)\geq \cdots\geq\rho(A^p)\geq \rho(A^{p+1})\geq \cdots.$$

Entonces, ocupando lo demostrado en el ejercicio 8, podemos decir que cuando existe la igualdad para un $p$ se cumple para todos los siguientes, es decir, 
$$\rho(A)> \rho(A^2) >\rho(A^3)>\cdots>\rho(A^{p})=\rho(A^{p+1})=\cdots.$$
Ahora, demostramos que $p$ (en donde se cumple la primera igualdad) debe de estar en el intervalo $[1,n]$. Si $p=1$ solo se cumple solo cuando $A=0$ (por \ref{nulo}). Ahora, probemos cuando $p>1$. Lo demostraremos por contradicción, supongamos que $p>n$, es decir, la igualdad se cumple en la potencia $p=n+k, \ k\in \mathbb{N}$, es decir,
$$\rho(A)> \rho(A^2) >\rho(A^3)>\cdots>\rho(A^{n})>\cdots=\rho(A^{n+k})=\cdots, \ \text{donde } k\in\mathbb{N}.$$
Por la definición de rango \ref{rango} tenemos que el rango de cualquier matriz $B_{n\times n}$ es menor o igual al número de columnas y mayor o igual que cero, es decir, $n\geq\rho(B)\geq 0$. Ahora como la igualdad se dio en $p=n+k$ significa que ocurrieron estrictamente $n+k$ desigualdades, pero esto implicaría que $\rho(A^{n+k})<0$, ya que se como se cumple estrictamente las desigualdades tendríamos que en cada potencia el rango disminuirá respecto a la potencia anterior, y como el rango esta acotado a $n$ y como ocurren $n+k$ desigualdades estrictas el rango sería negativo, pero por definición esto no es posible (llegamos a una contradicción), por lo tanto, $p\leq n$. Es decir, podemos concluir que existe un $p$ tal que $1\leq p\leq n$ que satisface
$$\rho(A)> \rho(A^2) >\rho(A^3)>\cdots>\rho(A^{p})=\rho(A^{p+1})=\cdots\ \ \finf$$
%Problema 10
%------------------------------------------------------------------------------------------------------%
%------------------------------------------------------------------------------------------------------%
%------------------------------------------------------------------------------------------------------%
\item Dada
\begin{align*}
A=\begin{pmatrix}
 1 & 0 & 2 & 2\\
 2 & 4 &16 & 1\\
-3 & 1 &-3 & 1
\end{pmatrix}
\end{align*}
encuentre su factorización por rango.

\res Recordemos que la factorización por rango.
\begin{framed}
    \begin{thmd} \label{factorizacion_rango}
	El rango de una matriz $A,\ m\times n$, es el mínimo entero para el cual podemos encontrar una matriz $C$ de tamaño $m\times r$ y $R$ de tamaño $r\times n$ tal que $A=CR$. 
    \end{thmd}
\end{framed}
Entonces tenemos que encontrar las matrices $C$ y $R$ de tamaños $3\times r$ y $r\times 4$ respectivamente. Para ello llevemos a la matriz $A$ a su forma escalonada reducida:
\begin{align*}
&\begin{pmatrix}
 1 & 0 & 2 & 2\\
 2 & 4 &16 & 1\\
-3 & 1 &-3 & 1
\end{pmatrix}%
\grstep[R3 \rightarrow R_3+3R_1]{R_2 \rightarrow R_2-2R_1}
%
\begin{pmatrix}
 1 & 0 & 2 & 2\\
 0 & 4 &12 &-3\\
 0 & 1 & 3 & 7
\end{pmatrix}%
\grstep[]{R_2 \rightarrow R_2/4}
%
\begin{pmatrix}
 1 & 0 & 2 & 2\\
 0 & 1 & 3 &-3/4\\
 0 & 1 & 3 & 7
\end{pmatrix}%
\grstep[]{R_2 \rightarrow R_3-R_2}
%
\begin{pmatrix}
 1 & 0 & 2 & 2\\
 0 & 1 & 3 &-3/4\\
 0 & 0 & 0 & 31/4
\end{pmatrix}\\
&\grstep[]{R_3 \rightarrow 4R_3/31}
%
\begin{pmatrix}
 1 & 0 & 2 & 2\\
 0 & 1 & 3 &-3/4\\
 0 & 0 & 0 & 1
\end{pmatrix}%
\grstep[R_2\rightarrow R_2+3R_3/4]{R_1 \rightarrow R_1-2R_3}
%
\begin{pmatrix}
 1 & 0 & 2 & 0\\
 0 & 1 & 3 & 0\\
 0 & 0 & 0 & 1
\end{pmatrix}.
\end{align*}
Por lo anterior tenemos que el rango de la matriz $A$ es $r=3$. Entonces observando la matriz escalonada reducida de $A$ y haciendo una matriz con las columnas de $A$ tal que las nuevas columnas de esta matriz sean linealmente independientes por lo que si consideramos la matriz escalonada reducida nos damos cuenta que las columnas 1,2 y 4 son linealmente independientes, por lo que podemos concluir que 
\begin{align*}
A=\begin{pmatrix}
 1 & 0 & 2 & 2\\
 2 & 4 &16 & 1\\
-3 & 1 &-3 & 1
\end{pmatrix}=\begin{pmatrix}
 1 & 0 & 2\\
 2 & 4 & 1\\
-3 & 1 & 1
\end{pmatrix}\begin{pmatrix}
 1 & 0 & 2 & 0\\
 0 & 1 & 3 & 0\\
 0 & 0 & 0 & 1
\end{pmatrix}=CR.\ \ \ \finf
\end{align*}

\end{enumerate}
\end{document}
