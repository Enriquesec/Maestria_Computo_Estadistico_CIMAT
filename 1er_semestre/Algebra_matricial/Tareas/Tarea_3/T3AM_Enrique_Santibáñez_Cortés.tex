\documentclass[11pt,letterpaper]{article}
\usepackage[utf8]{inputenc}
\usepackage[T1]{fontenc}
\usepackage[spanish]{babel}
\usepackage{amsmath}
\usepackage{amsfonts}
\usepackage{amssymb}
\usepackage{graphicx}
\usepackage{lmodern}
\usepackage{xspace}
\usepackage{multicol}
\usepackage{hyperref}
\usepackage{float}
\usepackage{hyperref}
\usepackage{color}

\hypersetup{colorlinks=true,   linkcolor=MaterialBlue900}
%\usepackage[colorlinks=true, linkcolor=black, urlcolor=blue, pdfborder={0 0 0}]{hyperref}

\usepackage[left=2cm,right=2cm,top=2cm,bottom=2cm]{geometry}
\title{Modelos no paramétricos y de regresión 2018-1}
\author{Tarea examen: pruebas binomiales y tablas de contingencia}
\date{Fecha de entrega: 08/01/2017}
\setlength{\parindent}{0in}
\spanishdecimal{.}

\newcommand{\X}{\mathbb{X}}
\newcommand{\x}{\mathbf{x}}
\newcommand{\Y}{\mathbf{Y}}
\newcommand{\y}{\mathbf{y}}
\newcommand{\xbarn}{\bar{x}_n}
\newcommand{\ybarn}{\bar{y}_n}
\newcommand{\paren}[1]{\left( #1 \right)}
\newcommand{\llaves}[1]{\left\lbrace #1 \right\rbrace}
\newcommand{\barra}{\,\vert\,}
\newcommand{\mP}{\mathbb{P}}
\newcommand{\mE}{\mathbb{E}}
\newcommand{\mI}{\mathbf{I}}
\newcommand{\mJ}{\mathbf{J}}
\newcommand{\mX}{\mathbf{X}}
\newcommand{\mS}{\mathbf{S}}
\newcommand{\mA}{\mathbf{A}}
\newcommand{\unos}{\boldsymbol{1}}
\newcommand{\xbarnv}{\bar{\mathbf{x}}_n}
\newcommand{\abs}[1]{\left\vert #1 \right\vert}
\newcommand{\muv}{\boldsymbol{\mu}}
\newcommand{\mcov}{\boldsymbol{\Sigma}}
\newcommand{\vbet}{\boldsymbol{\beta}}
\newcommand{\veps}{\boldsymbol{\epsilon}}
\newcommand{\mC}{\mathbf{C}}
\newcommand{\ceros}{\boldsymbol{0}}
\newcommand{\mH}{\mathbf{H}}
\newcommand{\ve}{\mathbf{e}}
\newcommand{\avec}{\mathbf{a}}
\newcommand{\res}{\textbf{RESPUESTA}\\}
\newcommand{\rojo}[1]{\textcolor{MaterialRed900}{#1}}

\newcommand{\defi}[3]{\textbf{Definición:#3}}
\newcommand{\fin}{$\blacksquare.$}
\newcommand{\finf}{\blacksquare.}
\newcommand{\tr}{\text{tr}}
\newcommand*{\temp}{\multicolumn{1}{r|}{}}

\newcommand{\grstep}[2][\relax]{%
   \ensuremath{\mathrel{
       {\mathop{\longrightarrow}\limits^{#2\mathstrut}_{
                                     \begin{subarray}{l} #1 \end{subarray}}}}}}
\newcommand{\swap}{\leftrightarrow}

\begin{document}
\begin{table}[ht]
\centering
\begin{tabular}{c}
\textbf{Maestría en Computo Estadístico}\\
\textbf{Álgebra Matricial} \\
\textbf{Tarea 3}\\
\today \\
\emph{Enrique Santibáñez Cortés}\\
Repositorio de Git: \href{https://github.com/Enriquesec/Algebra_matricial/tree/master/tareas/Tarea_3}{Tarea 3, AM}.
\end{tabular}
\end{table}
Todos los cálculos deben ser a mano.

\begin{enumerate}
% Problema 1.
%------------------------------------------------------------------------------
%------------------------------------------------------------------------------
\item Dada la matriz
\begin{equation*}
\left( \begin{array}{rrrr}
-4 & 5 & -6 &7\\
-1 & 1 &  1 &3\\
1  & 2 & -3 &-1
\end{array} \right)
\end{equation*}
encuentre su forma escalonada reducida por renglones. Escriba todas las matrices elementales correspondientes a las operaciones que usó para llevar la matriz a la forma que obtuvo.

\res
Para encontrar la forma escalonada reducida por renglones ocuparemos eliminación de Gauss-Jordan:
\begin{equation*}
\begin{array}{c}
\begin{pmatrix}
-4 & 5 & -6 &7\\
-1 & 1 &  1 &3\\
1  & 2 & -3 &-1
\end{pmatrix}
%
\grstep[]{R1 \Longleftrightarrow R3}
 %
\begin{pmatrix}
1  & 2 & -3 &-1\\
-1 & 1 &  1 &3\\
-4 & 5 & -6 &7
\end{pmatrix}%
\grstep[R3 \rightarrow R_3 +4R_1]{R_2 \rightarrow R_2 +R_1}
%
\begin{pmatrix}
1  & 2 & -3 &-1\\
0 & 3  & -2 & 2\\
0 & 13 & -18 &3
\end{pmatrix}%
\grstep[]{R2 \rightarrow R2/3}
\\
\\
\begin{pmatrix}
1  & 2 & -3 &-1\\
0 &  1 & -2/3 & 2/3\\
0 & 13 & -18 &3
\end{pmatrix}%
\grstep[]{R3 \rightarrow R3-13R2}
%
\begin{pmatrix}
1  & 2 & -3 &-1\\
0 &  1 & -2/3 & 2/3\\
0 &  0 & -28/3 & -17/3
\end{pmatrix}%
\grstep[]{R3 \rightarrow -3R3/28}
%
\begin{pmatrix}
1  & 2 & -3 &-1\\
0 &  1 & -2/3 & 2/3\\
0 &  0 & 1 & 17/28
\end{pmatrix}%
\grstep[R2 \rightarrow R_2+2R_3/3]{R_1 \rightarrow R_1 +R_3}\\ \\
\begin{pmatrix}
1  & 2 & 0 & 23/28\\
0 &  1 & 0 & 15/14\\
0 &  0 & 1 & 17/28
\end{pmatrix}%
\grstep[]{R_1 \rightarrow R_1 -2R_2}%
\begin{pmatrix}
1  & 0 & 0 & -37/28\\
0 &  1 & 0 & 15/14\\
0 &  0 & 1 & 17/28
\end{pmatrix}
\end{array}
\end{equation*}
Las matrices elementales que corresponden a las operaciones que se utilizaron para lleva la matriz a la forma escalonada reducida por renglones son:
$$E_{12}(-2)E_{23}(2/3)E_{13}(3)E_3(-3/28)E_{32}(-13)E_2(1/3)E_{31}(4)E_{21}(1)E_{13}A$$
o
$$\begin{pmatrix}
1 & -2 & 0\\
0 & 1 & 0\\
0 & 0 & 1
\end{pmatrix}
\begin{pmatrix}
1 & 0 & 0\\
0 & 1 & 2/3\\
0 & 0 & 1
\end{pmatrix}
\begin{pmatrix}
1 & 0 & 3\\
0 & 1 & 0\\
0 & 0 & 1
\end{pmatrix}
\begin{pmatrix}
1 & 0 & 0\\
0 & 1 & 0\\
0 & 0 & -\frac{3}{28}
\end{pmatrix} \begin{pmatrix}
1 & 0 & 0\\
0 & 1 & 0\\
0 & -13 & 1\\
\end{pmatrix} \begin{pmatrix}
1 & 0 & 0 \\
0 &\frac{1}{3} & 0 \\
0 & 0 & 1 \\
\end{pmatrix} \begin{pmatrix}
1 & 0 & 0\\
0 & 1 & 0\\
4 & 0 & 1\\
\end{pmatrix} \begin{pmatrix}
1 & 0 & 0\\
1 & 1 & 0\\
0 & 0 & 1\\
\end{pmatrix}$$
$$\cdot \begin{pmatrix}
0 & 0 & 1\\
0 & 1 & 0\\
1 & 0 & 0
\end{pmatrix}\begin{pmatrix}
-4 & 5 & -6 &7\\
-1 & 1 &  1 &3\\
1  & 2 & -3 &-1
\end{pmatrix}\ \ \ \finf$$

% Problema 2.
%------------------------------------------------------------------------------
%------------------------------------------------------------------------------
\item Dada el sistema $Ax=b$, donde 
\begin{equation*}
A=\left( \begin{array}{rrr}
1 & 3 & -1\\
a_1 & -1 & -3\\
1 & 2 &2
\end{array} \right) \ \ \ \text{y} \ \ \ b=\left(\begin{array}{c}
0\\
1\\
a_2
\end{array} \right)
\end{equation*}
encuentre condiciones generales sobre $a_1$ y $a_2$ para que el sistema sea consistente. Si se quiere que la solución sea exactamente $x=(3,-1,2)^t$ , ¿qué valores deben
tener $a_1$ y $a_2$?

\res Recordemos que un sistema es inconsistente cuando no tiene ninguna solución. Para determinar si el sistema es inconsistente o consistente ocuparemos eliminación gaussiana para resolver el sistema. El cual consiste primero en reducir por renglones la matriz aumentada a la forma escalonada por renglones y después sustitución hacia atrás. Reducimos la matriz aumentada a su forma escalonada reducida: 
\begin{equation*}
\begin{array}{c}
\left( \begin{array}{rrr|r}
1 & 3 & -1 & 0\\
a_1 & -1 & -3 & 1 \\
1 & 2 & 2 & a_2
\end{array} \right) %
\grstep[R3 \rightarrow R_3 -R_1]{R_2 \rightarrow R_2 -a_1R_1}
%
\left( \begin{array}{rrr|r}
1 & 3 & -1 & 0\\
0 & -1-3a_1 & -3+a_1 & 1 \\
0 & -1 & 3 & a_2
\end{array} \right) %
\grstep[]{R_2 \Longleftrightarrow R3}
%
\left( \begin{array}{rrr|r}
1 & 3 & -1 & 0\\
0 & -1 & 3 & a_2\\
0 & -1-3a_1 & -3+a_1 & 1 \\
\end{array} \right)\\
\\
%
\grstep[]{R_3 \rightarrow R_3-(1+3a_1)R_1}
%
\left( \begin{array}{rrr|r}
1 &  3 & -1 & 0\\
0 & -1 & 3 & a_2\\
0 &  0 & -6-8a_1 &  1-a_2-3a_2a_1\\
\end{array} \right)
\end{array}
\end{equation*}
El sistema anterior es \textbf{inconsistente o no tiene solución si} $-6-8a_1=0$ y $1-a_2(1+3a_1)\neq 0$, despejando de ambos lados tenemos $a_1=-\frac{3}{4}$ y $ a_2\neq \frac{1}{1+3a_1} \neq \frac{1}{1+3(-3/4)} \neq -\frac{4}{5}$.\\

Ahora, determinemos los valores de $a_1$ y $a_2$ para que la solución del sistema sea $x=(3,-1,2)^t$. Para que $(3,-1,2)^t$ sea solución se tiene que cumplir $A(3,-1,2)^t=b$, es decir,
\begin{equation*}
\left( \begin{array}{rrr}
1 & 3 & -1\\
a_1 & -1 & -3\\
1 & 2 &2
\end{array} \right) \begin{pmatrix}
3\\
-1\\
2
\end{pmatrix}=
\left(\begin{array}{c}
0\\
1\\
a_2
\end{array}\right) \Rightarrow \begin{pmatrix}
3-3-2\\
3a_1+1-6\\
3-2+4
\end{pmatrix} = \begin{pmatrix}
0\\
1\\
a_2
\end{pmatrix} \Rightarrow \begin{pmatrix}
-2\\
3a_1+1-6\\
3-2+4
\end{pmatrix} = \begin{pmatrix}
0\\
1\\
a_2
\end{pmatrix}
\end{equation*}
Pero si observamos el primer renglón de lo anterior, notamos que $x=(3,-1, 2)^t$ no puede ser solución del sistema debido a que no cumple la primera igualdad, independientemente de los valores que tomen $a_1$ y $a_2$.\ \ \ \fin 

% Problema 3.
%------------------------------------------------------------------------------
%------------------------------------------------------------------------------
\item Encuentre la solución general, escribiéndola como combinación lineal de vectores, del sistema homogéneo $Ax=0$ donde

\begin{equation*}
A=\left(\begin{array}{rrrrrr}
1  & -3 & 1 & -1 &  0 & 1\\
-1 &  3 & 0 &  3 &  1 & 3 \\
2  & -6 & 3 &  0 & -1 & 2\\
-1 &  3 & 1 &  5 &  1 & 6
\end{array}\right).
\end{equation*}

\res
Tenemos que $Ax=0$, es decir,
\begin{equation*}
\left(\begin{array}{rrrrrr}
1  & -3 & 1 & -1 &  0 & 1\\
-1 &  3 & 0 &  3 &  1 & 3 \\
2  & -6 & 3 &  0 & -1 & 2\\
-1 &  3 & 1 &  5 &  1 & 6
\end{array}\right) \begin{pmatrix}
x_1\\
x_2\\
x_3\\
x_4\\
x_5\\
x_6
\end{pmatrix}=\begin{pmatrix}
0\\
0\\
0\\
0\\
0\\
0
\end{pmatrix}.
\end{equation*}
Para encontrar la solución general ocupemos reducción gaussiana. Primero reduzcamos la matriz en su forma escalonada reducida:
\begin{equation*}
\begin{array}{c}
\left(\begin{array}{rrrrrr}
1  & -3 & 1 & -1 &  0 & 1\\
-1 &  3 & 0 &  3 &  1 & 3 \\
2  & -6 & 3 &  0 & -1 & 2\\
-1 &  3 & 1 &  5 &  1 & 6
\end{array}\right) %
\grstep[R3 \rightarrow R_3 -2R_1]{R_2 \rightarrow R_2 +R_1}
%
\left(\begin{array}{rrrrrr}
1  & -3 & 1 & -1 &  0 & 1\\
0  &  0 & 1 &  2 &  1 & 4 \\
0  &  0 & 1 &  2 & -1 & 0\\
-1 &  3 & 1 &  5 &  1 & 6
\end{array}\right)%
\grstep[R3 \rightarrow R_3 -R_2]{R_4 \rightarrow R_4 +R_1}\\
\\
\left(\begin{array}{rrrrrr}
1  & -3 & 1 & -1 &  0 & 1\\
0  &  0 & 1 &  2 &  1 & 4 \\
0  &  0 & 0 &  0 & -2 &-4\\
0  &  0 & 2 &  4 &  1 & 7
\end{array}\right)%
\grstep[R_3 \rightarrow -R_3/2]{R_4 \rightarrow R_4-2R_2}
%
\left(\begin{array}{rrrrrr}
1  & -3 & 1 & -1 &  0 & 1\\
0  &  0 & 1 &  2 &  1 & 4 \\
0  &  0 & 0 &  0 &  1 & 2\\
0  &  0 & 0 &  0 & -1 & -1
\end{array}\right)%
\grstep[]{R_4 \rightarrow R_4+R_3}
\end{array}
\end{equation*}

\begin{equation*}
\left(\begin{array}{rrrrrr}
1  & -3 & 1 & -1 &  0 & 1\\
0  &  0 & 1 &  2 &  1 & 4 \\
0  &  0 & 0 &  0 &  1 & 2\\
0  &  0 & 0 &  0 &  0 & -1
\end{array}\right)%
\grstep[]{R_4 \rightarrow -R_4}
\left(\begin{array}{rrrrrr}
1  & -3 & 1 & -1 &  0 & 1\\
0  &  0 & 1 &  2 &  1 & 4 \\
0  &  0 & 0 &  0 &  1 & 2\\
0  &  0 & 0 &  0 &  0 & 1
\end{array}\right)
\end{equation*}
Ahora realizamos sustitución hacia atrás de la matriz anterior:
\begin{equation*}
\begin{array}{c}
\left(\begin{array}{rrrrrr}
1  & -3 & 1 & -1 &  0 & 1\\
0  &  0 & 1 &  2 &  1 & 4 \\
0  &  0 & 0 &  0 &  1 & 2\\
0  &  0 & 0 &  0 &  0 & 1
\end{array}\right) %
\grstep[R_3 \rightarrow R_3-2R_4]{R_2 \rightarrow R_2-4R_4}
%
\left(\begin{array}{rrrrrr}
1  & -3 & 1 & -1 &  0 & 1\\
0  &  0 & 1 &  2 &  1 & 0 \\
0  &  0 & 0 &  0 &  1 & 0\\
0  &  0 & 0 &  0 &  0 & 1
\end{array}\right)%
\grstep[R_2 \rightarrow R_2-R_3]{R_1 \rightarrow R_1-R_4}\\ \\
\left(\begin{array}{rrrrrr}
1  & -3 & 1 & -1 & 0 & 0\\
0  &  0 & 1 &  2 & 0 & 0 \\
0  &  0 & 0 &  0 & 1 & 0\\
0  &  0 & 0 &  0 & 0 & 1
\end{array}\right) %
\grstep[]{R_1 \rightarrow R_1-R_2}
%
\left(\begin{array}{rrrrrr}
1  & -3 & 0 & -3 & 0 & 0\\
0  &  0 & 1 &  2 & 0 & 0 \\
0  &  0 & 0 &  0 & 1 & 0\\
0  &  0 & 0 &  0 & 0 & 1
\end{array}\right) 
\end{array}
\end{equation*}
Por lo anterior encontramos que $x_6=0$ y $x_5=0$ del reglón 3 y 4 respectivamente. Del renglón 2 obtenemos que $x_3=-2x_4$ y del renglón 1 $x_1=3x_2+3x_4.$ Por lo que la solución general es
$$x=\begin{pmatrix}
3x_2+3x_4\\
x_2\\
-2x_4\\
x_4\\
0\\
0
\end{pmatrix},
$$
la cual se puede escribir como combinación lineal de la siguiente forma:
$$x=\left\{ x_2\begin{pmatrix}
3\\
1\\
0\\
0\\
0\\
0
\end{pmatrix}+x_4\begin{pmatrix}
3\\
0\\
-2\\
1\\
0\\
0\\
\end{pmatrix} \right\} \ \ \ \ \ \ \finf$$

% Problema 4.
%------------------------------------------------------------------------------
%------------------------------------------------------------------------------
\item Encuentra la inversa de 
\begin{equation*}
\left(\begin{array}{rrr}
 1 &  0 & -2\\
-3 &  1 &  4\\
 2 & -3 &  4
\end{array}\right).
\end{equation*}

\res
Para encontrar $A^{-1}$, tenemos la ecuación $AX=I$ que puede verse como un conjunto de sistemas de ecuaciones $Ax_i=e_i$ donde $x_i$ son las columnas de $X$. Luego con el método de Gauss, si es posible resolver cada uno de estos sistemas con solución $x_i$, encontramos la inversas de $A$ dada por $X=(x_1, \cdots , x_n)$. Usando Gauss Jordan
$$L(A|I)=(LA|LI)=(I|L)$$
implica que $L=A^{-1}$. 

Ocupando lo anterior tenemos que:
\begin{equation*}
\begin{array}{c}
\left(\begin{array}{rrr|rrr}
 1 &  0 & -2 & 1 & 0 & 0 \\
-3 &  1 &  4 & 0 & 1 & 0\\
 2 & -3 &  4 & 0 & 0 & 1
\end{array}\right)%
\grstep[R3 \rightarrow R_3 -2R_1]{R_2 \rightarrow R_2 -3R_1}
%
\left(\begin{array}{rrr|rrr}
 1 &  0 & -2 &  1 & 0 & 0\\
 0 &  1 & -2 &  3 & 1 & 0\\
 0 & -3 &  8 & -2 & 0 & 1 
\end{array}\right)%
\grstep[]{R_3 \rightarrow R_3+3R2}
%
\left(\begin{array}{rrr|rrr}
 1 &  0 & -2 &  1 & 0 & 0\\
 0 &  1 & -2 &  3 & 1 & 0\\
 0 &  0 &  2 &  7 & 3 & 1 
\end{array}\right)\\
\\
%
\grstep[]{R_3 \rightarrow R_3/2}
% 
\left(\begin{array}{rrr|rrr}
 1 &  0 & -2 &  1 & 0 & 0\\
 0 &  1 & -2 &  3 & 1 & 0\\
 0 &  0 &  1 &  3.5 & 1.5 & 0.5 
\end{array}\right)%
\grstep[R2 \rightarrow R_2+2R_3]{R_1 \rightarrow R_1+2R_3}
%
\left(\begin{array}{rrr|rrr}
 1 &  0 & 0 &  8 & 3 & 1\\
 0 &  1 & 0 &  10 & 4 & 1\\
 0 &  0 & 1 &  3.5 & 1.5 & 0.5 
\end{array}\right).
\end{array}
\end{equation*}
Por lo tanto la inversa es: 
\begin{equation*}
\begin{pmatrix}
8 & 3 & 1 \\
10 & 4 & 1 \\
3.5 & 1.5 & 0.5
\end{pmatrix} \ \ \ \ \finf
\end{equation*}
% Problema 5.
%------------------------------------------------------------------------------
%------------------------------------------------------------------------------
\item Sea 
\begin{equation*}
A=\left(\begin{array}{rrr}
 1 & 1 & 1\\
 0 & 2 & 3\\
 5 & 5 & 1
\end{array}\right).
\end{equation*}
Demuestre que $A$ es no singular y luego escriba $A$ como producto de matrices elementales.

\res
Reduzcamos a $A$ en su forma escalonada:
\begin{equation*}
\left(\begin{array}{rrr}
 1 & 1 & 1\\
 0 & 2 & 3\\
 5 & 5 & 1
\end{array}\right) %
\grstep[]{R_3 \rightarrow R_3-5R_1}
%
\begin{pmatrix}
 1 & 1 & 1\\
 0 & 2 & 3\\
 0 & 0 & -4
\end{pmatrix}.
\end{equation*}
Recordemos que si\\
\textit{Si $A_{n\times n}$ se puede escribir en su forma escalonada con $n$ pivotes si solo sí es no singular (demostrado en clase).} \\
Por lo tanto, como la forma escalonada de $A_{3\times 3}$ tiene 3 pivotes entonces $A$ es no singular. Otra forma de demostrarlo es calculando el determinante de $A$:
$$ \det(A)=\begin{array}{|rrr|}
 1 & 1 & 1\\
 0 & 2 & 3\\
 0 & 0 & -4
\end{array}=(1)(2)(-4)=-8.$$
Por lo tanto, como el determinante de $A$ es distinto de cero eso implica que $A$ es no singular. 

Ahora para escribir a $A$ como producto de matrices elementales, hacemos reducción hacia atrás a la matriz escalonada calculada:
\begin{equation*}
\begin{pmatrix}
 1 & 1 & 1\\
 0 & 2 & 3\\
 0 & 0 & -4
\end{pmatrix}%
\grstep[]{R_3 \rightarrow -R_3/4}
%
\begin{pmatrix}
 1 & 1 & 1\\
 0 & 2 & 3\\
 0 & 0 & 1
\end{pmatrix}%
\grstep[R_2 \rightarrow R_2-3R_3 ]{R_1 \rightarrow R_1-R3}
%
\begin{pmatrix}
 1 & 1 & 0\\
 0 & 2 & 0\\
 0 & 0 & 1
\end{pmatrix}%
\grstep[]{R_2 \rightarrow R_2/2}
%
\begin{pmatrix}
 1 & 1 & 0\\
 0 & 1 & 0\\
 0 & 0 & 1
\end{pmatrix}%
\grstep[]{R_1 \rightarrow R_1-R_2}
%
\begin{pmatrix}
 1 & 0 & 0\\
 0 & 1 & 0\\
 0 & 0 & 1
\end{pmatrix}.
\end{equation*}
Ahora, transformando las operaciones elementales por matrices elementales,tenemos que lo anterior es equivalente a:
\begin{equation*}
\begin{pmatrix}
 1 &-1 & 0\\
 0 & 1 & 0\\
 0 & 0 & 1
\end{pmatrix}
\begin{pmatrix}
 1 & 0 & 0\\
 0 & \frac{1}{2} & 0\\
 0 & 0 & 1
\end{pmatrix}
\begin{pmatrix}
 1 & 0 & -1\\
 0 & 1 & 0\\\
 0 & 0 & 1
\end{pmatrix}
\begin{pmatrix}
 1 & 0 & 0\\
 0 & 1 & -3\\
 0 & 0 & 1
\end{pmatrix}
\begin{pmatrix}
 1 & 0 & 0\\
 0 & 1 & 0\\
 0 & 0 & -\frac{1}{4}
\end{pmatrix}
\begin{pmatrix}
 1 & 0 & 0\\
 0 & 1 & 0\\
-5 & 0 & 1
\end{pmatrix}A=I \Rightarrow
\end{equation*}
Ahora, haciendo uso de las propiedades de las matrices inversas ($(E_1E_2\cdots E_n)^{-1}=E_n^{-1}\cdots E_2^{-1}E_1^{-1}$, demostrada en clase), entonces
\begin{equation*}
A=\begin{pmatrix}
 1 & 0 & 0\\
 0 & 1 & 0\\
-5 & 0 & 1
\end{pmatrix}^{-1}\begin{pmatrix}
 1 & 0 & 0\\
 0 & 1 & 0\\
 0 & 0 & -\frac{1}{4}
\end{pmatrix}^{-1}\begin{pmatrix}
 1 & 0 & 0\\
 0 & 1 & -3\\
 0 & 0 & 1
\end{pmatrix}^{-1}
\begin{pmatrix}
 1 & 0 & -1\\
 0 & 1 & 0\\\
 0 & 0 & 1
\end{pmatrix}^{-1}
\begin{pmatrix}
 1 & 0 & 0\\
 0 & \frac{1}{2} & 0\\
 0 & 0 & 1
\end{pmatrix}^{-1}\begin{pmatrix}
 1 &-1 & 0\\
 0 & 1 & 0\\
 0 & 0 & 1
\end{pmatrix}^{-1}I \Rightarrow 
\end{equation*}
Ahora para calcular las inversas ocupemos lo visto en clase:\\
\textit{Sea $A=I_n+uv^t$ una matriz elemental. Entonces $A$ es invertible con inversa dada por
$$A^{-1}=I-\frac{uv^t}{1+v^tu}.$$ 
Más aún, si $E$ es elemetal del tipi I, II, ó III entonces $E^{-1}$ es elemental del mismo tipo que $E$.}

\begin{equation*}
A=\begin{pmatrix}
 1 & 0 & 0\\
 0 & 1 & 0\\
 5 & 0 & 1
\end{pmatrix}
\begin{pmatrix}
 1 & 0 & 0\\
 0 & 1 & 0\\
 0 & 0 & -4
\end{pmatrix}
\begin{pmatrix}
 1 & 0 & 0\\
 0 & 1 & 3\\
 0 & 0 & 1
\end{pmatrix}
\begin{pmatrix}
 1 & 0 & 1\\
 0 & 1 & 0\\\
 0 & 0 & 1
\end{pmatrix}
\begin{pmatrix}
 1 & 0 & 0\\
 0 & 2 & 0\\
 0 & 0 & 1
\end{pmatrix}
\begin{pmatrix}
 1 & 1 & 0\\
 0 & 1 & 0\\
 0 & 0 & 1
\end{pmatrix}\ \ \ \ \finf
\end{equation*}
Para que quede más especifico realizamos aquí las matrices inversas:
\begin{equation*}
\begin{array}{c}
 \left( \begin{array}{ccc|ccc}
 1 & 0 & 0 & 1 & 0 & 0\\
 0 & 1 & 0 & 0 & 1 & 0 \\
-5 & 0 & 1 & 0 & 0 & 1
\end{array} \right)%
\grstep[]{R_3 \rightarrow R_3+5R_2}
%
\left( \begin{array}{ccc|ccc}
 1 & 0 & 0 & 1 & 0 & 0\\
 0 & 1 & 0 & 0 & 1 & 0 \\
 0 & 0 & 1 & 5 & 0 & 1
\end{array} \right)\Rightarrow \begin{pmatrix}
 1 & 0 & 0\\
 0 & 1 & 0\\
-5 & 0 & 1
\end{pmatrix}^{-1}=\begin{pmatrix}
 1 & 0 & 0\\
 0 & 1 & 0\\
 5 & 0 & 1
\end{pmatrix}, 
\\ \\
\left( \begin{array}{ccc|ccc}
1 & 0 & 0& 1 & 0 & 0\\
0 & 1 & 0& 0 & 1 & 0 \\
0 & 0 & -\frac{1}{4} & 0 & 0 & 1
\end{array} \right)%
\grstep[]{R_3 \rightarrow -4R_3}
%
\left( \begin{array}{ccc|ccc}
1 & 0 & 0& 1 & 0 & 0\\
0 & 1 & 0& 0 & 1 & 0 \\
0 & 0 & 1 & 0 & 0 & -4
\end{array} \right)\Rightarrow
\begin{pmatrix} 
 1 & 0 & 0\\
 0 & 1 & 0\\
 0 & 0 & -\frac{1}{4}
\end{pmatrix}^{-1}=\begin{pmatrix} 
 1 & 0 & 0\\
 0 & 1 & 0\\
 0 & 0 & -4
\end{pmatrix}, \\ \\
\left( \begin{array}{ccc|ccc}
 1 & 0 & 0 & 1 & 0 & 0\\
 0 & 1 &-3 & 0 & 1 & 0\\
 0 & 0 & 1 & 0 & 0 & 1
\end{array}\right)%
\grstep[]{R_2 \rightarrow R_2+3R_3}
%
\left( \begin{array}{ccc|ccc}
 1 & 0 & 0 & 1 & 0 & 0\\
 0 & 1 & 0 & 0 & 1 & 3\\
 0 & 0 & 1 & 0 & 0 & 1
\end{array}\right) \Rightarrow 
\begin{pmatrix}
 1 & 0 & 0\\
 0 & 1 &-3\\
 0 & 0 & 1
\end{pmatrix}^{-1}=\begin{pmatrix}
1 & 0 & 0\\
0 & 1 & 3\\
0 & 0 & 1 
\end{pmatrix} \\ \\
\left( \begin{array}{ccc|ccc}
 1 & 0 &-1& 1 &0&0\\
 0 & 1 & 0&0&1&0\\
 0 & 0 & 1&0&0&1
\end{array} \right)%
\grstep[]{R_1 \rightarrow R_1+R_3}
% 
\left( \begin{array}{ccc|ccc}
 1 & 0 &0& 1 &0&1\\
 0 & 1 & 0&0&1&0\\
 0 & 0 & 1&0&0&1
\end{array} \right) \Rightarrow
\begin{pmatrix}
 1 & 0 & -1\\
 0 & 1 & 0\\\
 0 & 0 & 1
\end{pmatrix}^{-1}=\begin{pmatrix}
1 & 0&1\\
0&1&0\\
0&0&1
\end{pmatrix} \\ \\ 
\left(\begin{array}{ccc|ccc}
 1 & 0 & 0&1&0&0\\
 0 & \frac{1}{2} & 0&0&1&0\\
 0 & 0 & 1&0&0&1
\end{array} \right)%
\grstep[]{R_2 \rightarrow 2R_2}
% 
\left(\begin{array}{ccc|ccc}
 1 & 0 & 0&1&0&0\\
 0 & 1 & 0&0&2&0\\
 0 & 0 & 1&0&0&1
\end{array} \right) \Rightarrow
\begin{pmatrix}
 1 & 0 & 0\\
 0 & \frac{1}{2} & 0\\
 0 & 0 & 1
\end{pmatrix}^{-1}=\begin{pmatrix}
1&0&0\\
0&2&0\\
0&0&1
\end{pmatrix} \ \text{y}\ \\ \\ 
\left(\begin{array}{ccc|ccc}
 1 &-1 & 0&1&0&0\\
 0 & 1 & 0&0&1&0\\
 0 & 0 & 1&0&0&1
\end{array} \right)%
\grstep[]{R_1 \rightarrow R_1+R_2}
% 
\left(\begin{array}{ccc|ccc}
 1 & 0 & 0&1&1&0\\
 0 & 1 & 0&0&1&0\\
 0 & 0 & 1&0&0&1
\end{array} \right) \Rightarrow
\begin{pmatrix}
 1 &-1 & 0\\
 0 & 1 & 0\\
 0 & 0 & 1
\end{pmatrix}^{-1}=\begin{pmatrix}
1&1&0\\
0&1&0\\
0&0&1
\end{pmatrix}.
\end{array}
\end{equation*}


% Problema 6.
%------------------------------------------------------------------------------
%------------------------------------------------------------------------------
\item i) Encuentre dos matrices que sean invertibles pero que su suma no sea invertible. ii) Encuentre dos matrices singulares cuya suma sea invertible. Justifique todas sus aseveraciones.

\res
Recordemos que:\\
\textit{Si $A_{n\times n}$ se puede escribir en su forma escalonada con $n$ pivotes si solo sí es no singular (demostrado en clase).}\\
Sea $A_{3\times 3}$ y $B_{3\times 3}$ matrices triangulares superiores, con elementos en la diagonal distintos de cero, esto implica que $A$ y $B$ son invertibles debido a que estan en su forma escalonada y tienen $3$ pivotes:
$$A=\begin{pmatrix}
a_{11} & a_{12}& a_{13}\\
0 & a_{22} & a_{23}\\
0 &0&a_{33}
\end{pmatrix}\ \ \text{y} \ \ B=\begin{pmatrix}
b_{11} & b_{12}& b_{13}\\
0 &b_{22} & b_{23}\\
0 &0&b_{33}
\end{pmatrix},\ \text{donde} \ a_{ii}\neq 0 \ \text{y} b_{ii}\neq 0.$$
Ahora haciendo que $a_{33}=-b_{33}$ y $a_{11}\neq b_{11}$, $a_{22}\neq b_{22}$, esto implicaría que:
$$A+B=\begin{pmatrix}
a_{11} & a_{12}& a_{13}\\
0 & a_{22} & a_{23}\\
0 &0&-b_{33}
\end{pmatrix}+\begin{pmatrix}
b_{11} & b_{12}& b_{13}\\
0 &b_{22} & b_{23}\\
0 &0&b_{33}
\end{pmatrix}=\begin{pmatrix}
a_{11}+b_{12} & a_{12}+b_{12}& a_{13}+b_{13}\\
0 & a_{22}+b_{22} & a_{23}+b_{23}\\
0 &0&0
\end{pmatrix}.$$
Y por lo tanto, como $A+B$ tiene a lo más 2 pivotes esto implica que $A+B$ no se invertible.  \\

i) Por lo anterior, \textbf{podemos proponer las siguientes dos matrices $A$ y $B$ tal que son invertibles y $A+B$ no sea invertible}:
$$A=\begin{pmatrix}
1&2&3\\
0&4&5&\\
0&0&6
\end{pmatrix} \ \ \text{y}\ \ B=\begin{pmatrix}
2&3&4\\
0&5&7\\
0&0&-6
\end{pmatrix}, \ \Rightarrow A+B=\begin{pmatrix}
3&5&7\\
0&9&12\\
0&0&0
\end{pmatrix}.$$
Sea $A_{3\times 3}$ y $B_{3\times 3}$ matrices triangulares superiores, con un elemento diagonal igual a cero pero en diferente posición en $A$ y $B$ ($a_{33}=0$ y $b_{22}=0$), esto implica que $A$ y $B$ no son invertibles debido a que estan en su forma escalonada y tienen $2$ pivotes:
$$A=\begin{pmatrix}
a_{11} & a_{12}& a_{13}\\
0 & a_{22} & a_{23}\\
0 &0&0
\end{pmatrix}\ \ \text{y} \ \ B=\begin{pmatrix}
b_{11} & b_{12}& b_{13}\\
0 &0 & b_{23}\\
0 &0&b_{33}
\end{pmatrix},\ \text{donde} \ a_{ii}\neq 0 \ \text{y} b_{ii}\neq 0.$$
Ahora considerando a $a_{11}\neq -b_{11}$, entonces:
$$A+B=\begin{pmatrix}
a_{11} & a_{12}& a_{13}\\
0 & a_{22} & a_{23}\\
0 &0&0
\end{pmatrix}+\begin{pmatrix}
b_{11} & b_{12}& b_{13}\\
0 &0 & b_{23}\\
0 &0&b_{33}
\end{pmatrix}=\begin{pmatrix}
a_{11}+b_{11} & a_{12}+b_{12}& a_{13}+b_{13}\\
0 &a_{22} & a_{23}+b_{23}\\
0 &0&b_{33}
\end{pmatrix}\ \text{donde} \ a_{ii}\neq 0 \ \text{y} b_{ii}\neq 0.$$
Y por lo tanto, como $A+B$ tiene 3 pivotes esto implica que $A+B$ es invertible.\\

ii) Por lo anterior, \textbf{podemos proponer las siguientes dos matrices $A$ y $B$ tal que no sean invertibles y $A+B$ sea invertible}:
$$A=\begin{pmatrix}
1&2&3\\
0&4&5&\\
0&0&0
\end{pmatrix} \ \ \text{y}\ \ B=\begin{pmatrix}
2&3&4\\
0&0&7\\
0&0&8
\end{pmatrix}, \ \Rightarrow A+B=\begin{pmatrix}
3&5&7\\
0&4&12\\
0&0&8
\end{pmatrix}.$$
% Problema 7.
%------------------------------------------------------------------------------
%------------------------------------------------------------------------------
\item Encuentre la descomposición LU de la matriz
\begin{equation*}
\left(\begin{array}{rrrr}
 1 &  2 & -1 & 4\\
 0 & -1 &  5 & 8\\
 2 &  3 &  1 & 4\\
 1 & -1 &  6 & 4
\end{array} \right).
\end{equation*}

\res Sea $T_i$  una matriz inferior elemental y si no se requieren intercambios de renglones se pueden hacer $n-1$
multiplicaciones por la izquierda de $A_{n\times n}$ por matrices triangulares inferiores elementales para llevar a A a una forma triangular superior, es decir:
$$T_{n-1} \cdots T_1 A=U$$ 
Luego 
$$A=T^{-1}_1\cdots T^{-1}_{n-1}U=LU,$$
donde $L=T^{-1}_1\cdots T^{-1}_n$.\\

Entonces, usemos eliminación gaussiana para determinar la matriz $U$:
\begin{equation*}
\begin{array}{c}
\left(
\begin{array}{rrrr}
 1 &  2 & -1 & 4\\
 0 & -1 &  5 & 8\\
 2 &  3 &  1 & 4\\
 1 & -1 &  6 & 4
\end{array}
\right)%
\grstep[R_4 \rightarrow R_4-R_1]{R_3 \rightarrow R_3-2R_1}
%
\left(
\begin{array}{rrrr}
 1 &  2 & -1 &  4\\
 0 & -1 &  5 &  8\\
 0 & -1 &  3 & -4\\
 0 & -3 &  7 &  0
\end{array}
\right) %
\grstep[R_4 \rightarrow R_4-3R_2]{R_3 \rightarrow R_3-R_2}
% 
\left(
\begin{array}{rrrr}
 1 &  2 & -1 &   4\\
 0 & -1 &  5 &   8\\
 0 &  0 & -2 & -12\\
 0 &  0 & -8 & -24
\end{array}
\right) %
\grstep[]{R_4 \rightarrow R_4-4R_3}
\\
\\
\left(
\begin{array}{rrrr}
 1 &  2 & -1 &   4\\
 0 & -1 &  5 &   8\\
 0 &  0 & -2 & -12\\
 0 &  0 &  0 & 24
\end{array}
\right).
\end{array} 
\end{equation*}
Por lo tanto, 
\begin{equation*}
U=\left(
\begin{array}{rrrr}
 1 &  2 & -1 &   4\\
 0 & -1 &  5 &   8\\
 0 &  0 & -2 & -12\\
 0 &  0 &  0 & 24
\end{array}
\right).
\end{equation*}
Ahora determinaremos $L$, 

$$E_{43}(-4)E_{42}(-3)E_{32}(-1)E_{41}(-1)E_{31}(-2)A=U $$
o
$$\left(\begin{array}{cccc}
1&0&0&0\\
0&1&0&0\\
0&0&1&0\\
0&0&-4&1
\end{array} \right) \left(\begin{array}{cccc}
1&0&0&0\\
0&1&0&0\\
0&0&1&0\\
0&-3&0&1
\end{array} \right)\left(\begin{array}{cccc}
1&0&0&0\\
0&1&0&0\\
0&-1&1&0\\
0&0&0&1
\end{array} \right)\left(\begin{array}{cccc}
1&0&0&0\\
0&1&0&0\\
0&0&1&0\\
-1&0&0&1
\end{array} \right)\left(\begin{array}{cccc}
1&0&0&0\\
0&1&0&0\\
-2&0&1&0\\
0&0&0&1
\end{array} \right)A=U\Rightarrow A=$$

$$\left(\begin{array}{cccc}
1&0&0&0\\
0&1&0&0\\
-2&0&1&0\\
0&0&0&1
\end{array}\right)^{-1}\left(\begin{array}{cccc}
1&0&0&0\\
0&1&0&0\\
0&0&1&0\\
-1&0&0&1
\end{array} \right)^{-1}\left(\begin{array}{cccc}
1&0&0&0\\
0&1&0&0\\
0&-1&1&0\\
0&0&0&1
\end{array} \right)^{-1}\left(\begin{array}{cccc}
1&0&0&0\\
0&1&0&0\\
0&0&1&0\\
0&-3&0&1
\end{array} \right)^{-1}\left(\begin{array}{cccc}
1&0&0&0\\
0&1&0&0\\
0&0&1&0\\
0&0&-4&1
\end{array} \right)^{-1}U.$$
Y esto implica que:
$$L=\left(\begin{array}{cccc}
1&0&0&0\\
0&1&0&0\\
-2&0&1&0\\
0&0&0&1
\end{array}\right)^{-1}\left(\begin{array}{cccc}
1&0&0&0\\
0&1&0&0\\
0&0&1&0\\
-1&0&0&1
\end{array} \right)^{-1}\left(\begin{array}{cccc}
1&0&0&0\\
0&1&0&0\\
0&-1&1&0\\
0&0&0&1
\end{array} \right)^{-1}\left(\begin{array}{cccc}
1&0&0&0\\
0&1&0&0\\
0&0&1&0\\
0&-3&0&1
\end{array} \right)^{-1}\left(\begin{array}{cccc}
1&0&0&0\\
0&1&0&0\\
0&0&1&0\\
0&0&-4&1
\end{array} \right)^{-1}$$


$$=\left(\begin{array}{cccc}
1&0&0&0\\
0&1&0&0\\
2&0&1&0\\
0&0&0&1
\end{array}\right)\left(\begin{array}{cccc}
1&0&0&0\\
0&1&0&0\\
0&0&1&0\\
1&0&0&1
\end{array} \right)\left(\begin{array}{cccc}
1&0&0&0\\
0&1&0&0\\
0&1&1&0\\
0&0&0&1
\end{array} \right)\left(\begin{array}{cccc}
1&0&0&0\\
0&1&0&0\\
0&0&1&0\\
0&3&0&1
\end{array} \right)\left(\begin{array}{cccc}
1&0&0&0\\
0&1&0&0\\
0&0&1&0\\
0&0&4&1
\end{array} \right)=\left(
\begin{array}{rrrr}
 1 &  0 & 0 &  0\\
 0 &  1 & 0 &  0\\
 2 &  1 & 1 &  0\\
 1 &  3 & 4 &  1
\end{array}\right).$$
Es decir, la descomposición LU de la matriz definida es:
$$
\left(\begin{array}{rrrr}
 1 &  2 & -1 & 4\\
 0 & -1 &  5 & 8\\
 2 &  3 &  1 & 4\\
 1 & -1 &  6 & 4
\end{array} \right)=\left(
\begin{array}{rrrr}
 1 &  0 & 0 &  0\\
 0 &  1 & 0 &  0\\
 2 &  1 & 1 &  0\\
 1 &  3 & 4 &  1
\end{array}\right)\left(
\begin{array}{rrrr}
 1 &  2 & -1 &   4\\
 0 & -1 &  5 &   8\\
 0 &  0 & -2 & -12\\
 0 &  0 &  0 & 24
\end{array}
\right) \ \ \ \finf
$$
Para que quede más especifico que se realizaron las matrices inversas a mano:

\begin{equation*}
 \begin{array}{c}
\cdot \left(\begin{array}{cccc|cccc}
1&0&0&0&1&0&0&0\\
0&1&0&0&0&1&0&0\\
-2&0&1&0&0&0&1&0\\
0&0&0&1&0&0&0&1\\
\end{array}\right) %
\grstep[]{R_3 \rightarrow R_3+2R_1}
%
\left(\begin{array}{cccc|cccc}
1&0&0&0&1&0&0&0\\
0&1&0&0&0&1&0&0\\
0&0&1&0&2&0&1&0\\
0&0&0&1&0&0&0&1\\
\end{array}\right)
\Rightarrow \\ \\
\left(\begin{array}{cccc}
1&0&0&0\\
0&1&0&0\\
-2&0&1&0\\
0&0&0&1
\end{array}\right)^{-1}=\left(\begin{array}{cccc}
1&0&0&0\\
0&1&0&0\\
2&0&1&0\\
0&0&0&1
\end{array}\right).\ \\ \\
\left(\begin{array}{cccc|cccc}
1&0&0&0&1&0&0&0\\
0&1&0&0&0&1&0&0\\
0&0&1&0&0&0&1&0\\
-1&0&0&1&0&0&0&1
\end{array} \right)%
\grstep[]{R_4 \rightarrow R_4+R_1}
% 
\left(\begin{array}{cccc|cccc}
1&0&0&0&1&0&0&0\\
0&1&0&0&0&1&0&0\\
0&0&1&0&0&0&1&0\\
0&0&0&1&1&0&0&1
\end{array} \right)\Rightarrow\\ \\
\left(\begin{array}{cccc}
1&0&0&0\\
0&1&0&0\\
0&0&1&0\\
-1&0&0&1
\end{array} \right)^{-1}=\left(\begin{array}{cccc}
1&0&0&0\\
0&1&0&0\\
0&0&1&0\\
1&0&0&1
\end{array} \right).\\ \\
\left(\begin{array}{cccc|cccc}
1&0&0&0&1&0&0&0\\
0&1&0&0&0&1&0&0\\
0&-1&1&0&0&0&1&0\\
0&0&0&1&0&0&0&1
\end{array} \right)%
\grstep[]{R_3 \rightarrow R_3+R_2}
% 
\left(\begin{array}{cccc|cccc}
1&0&0&0&1&0&0&0\\
0&1&0&0&0&1&0&0\\
0&0&1&0&0&1&1&0\\
0&0&0&1&0&0&0&1
\end{array} \right)\Rightarrow
\end{array}
\end{equation*}

\begin{equation*}
 \begin{array}{c}
\left(\begin{array}{cccc}
1&0&0&0\\
0&1&0&0\\
0&-1&1&0\\
0&0&0&1
\end{array} \right)^{-1}= \left(\begin{array}{cccc}
1&0&0&0\\
0&1&0&0\\
0&1&1&0\\
0&0&0&1
\end{array} \right).\\ \\
\left(\begin{array}{cccc|cccc}
1&0&0&0&1&0&0&0\\
0&1&0&0&0&1&0&0\\
0&0&1&0&0&0&1&0\\
0&-3&0&1&0&0&0&1
\end{array} \right)%
\grstep[]{R_3 \rightarrow R_3+R_2}
% 
\left(\begin{array}{cccc|cccc}
1&0&0&0&1&0&0&0\\
0&1&0&0&0&1&0&0\\
0&0&1&0&0&0&1&0\\
0&0&0&1&0&3&0&1
\end{array} \right)\Rightarrow 
\end{array}
\end{equation*}
\begin{equation*}
\begin{array}{c}
\left(\begin{array}{cccc}
1&0&0&0\\
0&1&0&0\\
0&0&1&0\\
0&-3&0&1
\end{array} \right)^{-1}=\left(\begin{array}{cccc}
1&0&0&0\\
0&1&0&0\\
0&0&1&0\\
0&3&0&1
\end{array} \right)\\ \\
\left(\begin{array}{cccc|cccc}
1&0&0&0&1&0&0&0\\
0&1&0&0&0&1&0&0\\
0&0&1&0&0&0&1&0\\
0&0&-4&1&0&0&0&1
\end{array} \right)%
\grstep[]{R_4 \rightarrow R_4+4R_3}
% 
\left(\begin{array}{cccc|cccc}
1&0&0&0&1&0&0&0\\
0&1&0&0&0&1&0&0\\
0&0&1&0&0&0&1&0\\
0&0&0&1&0&0&4&1
\end{array} \right)\\ \\
\left(\begin{array}{cccc}
1&0&0&0\\
0&1&0&0\\
0&0&1&0\\
0&0&-4&1
\end{array} \right)^{-1}=\left(\begin{array}{cccc}
1&0&0&0\\
0&1&0&0\\
0&0&1&0\\
0&0&4&1
\end{array} \right).
\end{array}
\end{equation*}

% Problema 8.
%------------------------------------------------------------------------------
%------------------------------------------------------------------------------
\item Encuentre la descomposición LU de la matriz
\begin{equation*}
A=\left(\begin{array}{rrrr}
 2 &  3 & -1 & 6\\
 4 &  7 &  2 & 1\\
-2 &  5 & -2 & 0\\
 0 & -4 &  5 & 2
\end{array} \right),
\end{equation*}
y luego úsela para encontrar la solución del sistema $Ax=b$, donde 
\begin{equation*}
b=\left(\begin{array}{c}
1\\
0\\
0\\
4
\end{array} \right).
\end{equation*}
\res 
El problema anterior se resolvió utilizando la metodología vista en clase. Ahora se hará uso de
un método "más sencillo" (Álgebra Lineal, Grossman (2012)). Si $A =LU$, si sabe que $A$ se puede factorizar como:
\begin{equation*}
A=\left(\begin{array}{rrrr}
 2 &  3 & -1 & 6\\
 4 &  7 &  2 & 1\\
-2 &  5 & -2 & 0\\
 0 & -4 &  5 & 2
\end{array} \right) =
\left(\begin{array}{rrrr}
 1 &  0 &  0 & 0\\
 a &  1 &  0 & 0\\
 b &  c &  1 & 0\\
 d &  e &  f & 1
\end{array} \right) \left(\begin{array}{rrrr}
 2 &  3 & -1 & 6\\
 0 &  u &  v & w\\
 0 &  0 &  x & y\\
 0 &  0 &  0 & z
\end{array} \right)=LU
\end{equation*}
Observe que el primer renglón de U es el mismo que el primer renglón de A porque al reducir A a la forma triangular, no hace falta modificar los elementos del primer renglón.\\
Se pueden obtener todos los coeficientes faltantes con tan sólo multiplicar las matrices. La
componente 2, 1 de $A$ es 4. De este modo, el producto escalar del segundo renglón de $L$ y la primera columna de $U$ es igual a 4:
$$4=2a \ \ \ \text{o}  \ \ \ \ a=2$$
Después se tiene:\\
componente $2,2$: $7=6+u \rightarrow \ u=1.$\\
De aquí en adelante se pueden insertar los valores que se encuentran en $L$ y $U$:
\begin{table}[H]
\centering
\begin{tabular}{l|l}
componente $2,3$: $2=-2+v \rightarrow \ v=4.$ & componente $3,4$: $0=-6-88+y \rightarrow \ y=94.$\\ \\
componente $2,4$: $1=12+w \rightarrow \ w=-11.$ & componente $4,1$: $0=2d \rightarrow \ d=0.$\\ \\
componente $3,1$: $-2=2b \rightarrow \ b=-1.$ & componente $4,2$: $-4=e \rightarrow \ e=-4.$\\ \\
componente $3,2$: $5=-3+c \rightarrow \ c=8.$ & componente $4,3$: $5=-16-35f \rightarrow \ f=-3/5.$\\ \\
componente $3,3$: $-2=+1+32+x \rightarrow \ x=-35.$ & componente $4,4$: $2=44-3(94)/5+z \rightarrow \ z=72/5.$\\ 
\end{tabular}
\end{table}
Por lo que:
\begin{equation*}
A=\left(\begin{array}{rrrr}
 2 &  3 & -1 & 6\\
 4 &  7 &  2 & 1\\
-2 &  5 & -2 & 0\\
 0 & -4 &  5 & 2
\end{array} \right) =
\left(\begin{array}{rrrr}
 1 &  0 &  0 & 0\\
 2 &  1 &  0 & 0\\
-1 &  8 &  1 & 0\\
 0 &  -4 & -3/5 & 1
\end{array} \right) \left(\begin{array}{rrrr}
 2 &  3 & -1 & 6\\
 0 &  1 &  4 & -11\\
 0 &  0 &  -35 & 94\\
 0 &  0 &  0 & 72/5
\end{array} \right)=LU
\end{equation*}
Ahora, para resolver un sistema lineal $Ax=b$ cuando $A=LU$ se resuelven dos sistemas triangulares $Ly=b$ y $Ux=y.$ Primero se una sustitución hacia adelante en $Ly=b$, y luego se resuelve el sistema $Ux=y$ usando sustitución hacia atras (esta metodología se presento en clase). Entonces para este problema tenemos que:
\begin{equation*}
\left(\begin{array}{rrrr}
 1 &  0 &  0 & 0\\
 2 &  1 &  0 & 0\\
-1 &  8 &  1 & 0\\
 0 &  -4 & -3/5 & 1
\end{array} \right) \begin{pmatrix}
y_1\\
y_2\\
y_3\\
y_4
\end{pmatrix}= \begin{pmatrix}
1\\
0\\
0\\
4
\end{pmatrix}
\end{equation*}
Esto implica que 
$$y_1=1$$
$$2(1)+y_2=0\Rightarrow y_2=-2$$
$$-(1)+8(-2)+y_3=0\Rightarrow y_3=17$$
$$0-4(-2)-3(17)/5+y_4=4\Rightarrow y_4=31/5$$
Sea acaba de realizar la sustitución hacia adelante. Ahora, de $Ux=y$ se obtiene:
\begin{equation*}
\begin{array}{rcr}
2x_1+3x_2-x_3+6x_4&=&1\\
x_2+4x_3-11x_4&=&-2\\
-35x_3 +94x_4 &=&17\\
72/5x_4&=&31/5
\end{array}
\end{equation*}
o
\begin{equation*}
\begin{array}{lll}
x_4&=&\frac{31}{72}\\ \\
-35x_3 +94(\frac{31}{72}) &=&17,\text{de manera que } x_3=\frac{17-94\left(\frac{31}{72}\right)}{-35}=\frac{17*72-94*31}{-35*72}=\frac{-1690}{-2520}=\frac{169}{252}\\ \\
x_2+4x_3-11x_4&=&-2, \text{de manera que } x_2=-2+11\left(\frac{31}{72}\right) -4\left(\frac{169}{252}\right)=\frac{3}{56}\\ \\
2x_1+3x_2-x_3+6x_4&=&1, \text{de manera que } x_1=\frac{1}{2}-3\left(\frac{31}{72}\right) + \frac{169}{504}-3\left( \frac{3}{112} \right)=-\frac{541}{1008}\\
\end{array}
\end{equation*}

Por lo tanto la solución es:
$$x=\begin{pmatrix}
-\frac{541}{1008}\\ \\
\frac{3}{56}\\ \\ 
\frac{169}{252}\\ \\
\frac{31}{72}
\end{pmatrix} \ \ \ \ \finf
$$
\textit{Cálculos de la simplificación de $x_2$ y $x_1:$}
$$-2+11\left(\frac{31}{72}\right) -4\left(\frac{169}{252}\right)=-2+\frac{341}{72}-\frac{338}{126}=\frac{-18144+42966-24336}{9000}=\frac{486}{72*126}=\frac{3}{56}.$$

$$\frac{1}{2}-3\left(\frac{31}{72}\right) + \frac{169}{504}-3\left( \frac{3}{112} \right)=\frac{1}{2}-\frac{31}{24}+\frac{169}{504}-\frac{9}{112}=\frac{504-1302+338-81}{1008}=-\frac{541}{1008}.$$
% Problema 9.
%------------------------------------------------------------------------------
%------------------------------------------------------------------------------
\item Encuentre la descomposición LU de la matriz 
\begin{equation*}
A=\left(\begin{array}{rrrr}
 1 & -2 & -2 &-3\\
 3 & -9 &  0 &-9\\
-1 &  2 &  4 & 7\\
-3 & -6 & 26 & 2
\end{array} \right),
\end{equation*}
Usando esta misma descomposición como ayuda, encuentre $A^{-1}$.

\res 
Utilizando la metodología que el ejercicio anterior. Tenemos que si $A=LU$, si se sabe que $A$ se puede factorizar como:
\begin{equation*}
A=\left(\begin{array}{rrrr}
 1 & -2 & -2 &-3\\
 3 & -9 &  0 &-9\\
-1 &  2 &  4 & 7\\
-3 & -6 & 26 & 2
\end{array} \right)=
\left(\begin{array}{rrrr}
 1 &  0 &  0 & 0\\
 a &  1 &  0 & 0\\
 b &  c &  1 & 0\\
 d &  e &  f & 1
\end{array} \right) \left(\begin{array}{rrrr}
 1 & -2 & -2 & -3\\
 0 &  u &  v &  w\\
 0 &  0 &  x &  y\\
 0 &  0 &  0 &  z
\end{array} \right)=LU
\end{equation*}
Observe que el primer renglón de U es el mismo que el primer renglón de A porque al reducir A a la forma triangular, no hace falta modificar los elementos del primer renglón.\\
Se pueden obtener todos los coeficientes faltantes con tan sólo multiplicar las matrices. La
componente 2, 1 de $A$ es 3. De este modo, el producto escalar del segundo renglón de $L$ y la primera columna de $U$ es igual a 3:
$$3=a \ \ \ \text{o}  \ \ \ \ a=3$$
Después se tiene:\\
componente $2,2$: $-9=-6+u \rightarrow \ u=-3.$\\
De aquí en adelante se pueden insertar los valores que se encuentran en $L$ y $U$:
\begin{table}[H]
\centering
\begin{tabular}{l|l}
componente $2,3$: $0=-6+v \rightarrow \ v=6.$ & componente $3,4$: $7=3+y \rightarrow \ y=4.$\\ \\
componente $2,4$: $-9=-9+w \rightarrow \ w=0.$ & componente $4,1$: $-3=d \rightarrow \ d=-3.$\\ \\
componente $3,1$: $-1=b \rightarrow \ b=-1.$ & componente $4,2$: $-6=6-3e \rightarrow \ e=4.$\\ \\
componente $3,2$: $2=2+-3c \rightarrow \ c=0.$ & componente $4,3$: $26=6+24+2f \rightarrow \ f=-2.$\\ \\
componente $3,3$: $4=2+x \rightarrow \ x=2.$ & componente $4,4$: $2=9+-8+z \rightarrow \ z=1.$\\ 
\end{tabular}
\end{table}
Por lo que:
\begin{equation*}
A=\left(\begin{array}{rrrr}
 1 & -2 & -2 &-3\\
 3 & -9 &  0 &-9\\
-1 &  2 &  4 & 7\\
-3 & -6 & 26 & 2
\end{array} \right)=
\left(\begin{array}{rrrr}
 1 &  0 &  0 & 0\\
 3 &  1 &  0 & 0\\
-1 &  0 &  1 & 0\\
-3 &  4 & -2 & 1
\end{array} \right) \left(\begin{array}{rrrr}
 1 & -2 & -2 & -3\\
 0 & -3 &  6 &  0\\
 0 &  0 &  2 &  4\\
 0 &  0 &  0 &  1
\end{array} \right)=LU
\end{equation*}
Ahora, considerando la propiedad de la inversa de una matriz (demostrada en clase):
$$(AB)^{-1}=B^{-1}A^{-1}.$$
Entonces tenemos que:
$$A^{-1} = (LU)^{-1}=U^{-1}L^{-1}$$
Calculemos la inversa de $U^{-1}$ y $L^{-1}$:
\begin{equation*}
\begin{array}{c}
\left(\begin{array}{rrrr|rrrr}
 1 &  0 &  0 & 0&1&0&0&0\\
 3 &  1 &  0 & 0&0&1&0&0\\
-1 &  0 &  1 & 0&0&0&1&0\\
-3 &  4 & -2 & 1&0&0&0&1
\end{array} \right)%
\grstep[R_3 \rightarrow R_3+R_1]{R_2 \rightarrow R_2-3R_1}
%
\left(\begin{array}{rrrr|rrrr}
 1 &  0 &  0 & 0&1&0&0&0\\
 0 &  1 &  0 & 0&-3&1&0&0\\
 0 &  0 &  1 & 0&1&0&1&0\\
-3 &  4 & -2 & 1&0&0&0&1
\end{array} \right)%
\grstep[R_4 \rightarrow R_4-4R_2]{R_4 \rightarrow R_4+3R_1}
\\ \\
\left(\begin{array}{rrrr|rrrr}
 1 &  0 &  0 & 0&1&0&0&0\\
 0 &  1 &  0 & 0&-3&1&0&0\\
 0 &  0 &  1 & 0&1&0&1&0\\
 0 &  0 & -2 & 1&15&-4&0&1
\end{array} \right)%
\grstep[]{R_4 \rightarrow R_4+2R_3}
\left(\begin{array}{rrrr|rrrr}
 1 &  0 &  0 & 0&1&0&0&0\\
 0 &  1 &  0 & 0&-3&1&0&0\\
 0 &  0 &  1 & 0&1&0&1&0\\
 0 &  0 &  0 & 1&17&-4&2&1
\end{array} \right)\Rightarrow\\ \\ 
L^{-1} =\begin{pmatrix}
1&0&0&0\\
-3&1&0&0\\
1&0&1&0\\
17&-4&2&1
\end{pmatrix}.
\end{array}
\end{equation*}
\begin{equation*}
\begin{array}{c}
\left(\begin{array}{rrrr|rrrr}
 1 & -2 & -2 &-3&1&0&0&0\\
 0 & -3 &  6 & 0&0&1&0&0\\
 0 &  0 &  2 & 4&0&0&1&0\\
 0 &  0 &  0 & 1&0&0&0&1
\end{array} \right)%
\grstep[R_3 \rightarrow R_3-4R_4]{R_1 \rightarrow R_1+3R_4}
%
\left(\begin{array}{rrrr|rrrr}
 1 & -2 & -2 & 0&1&0&0&3\\
 0 & -3 &  6 & 0&0&1&0&0\\
 0 &  0 &  2 & 0&0&0&1&-4\\
 0 &  0 &  0 & 1&0&0&0&1
\end{array} \right)%
\grstep[]{R_3 \rightarrow R_3/2}
\\ \\
\left(\begin{array}{rrrr|rrrr}
 1 & -2 & -2 & 0&1&0&0&3\\
 0 & -3 &  6 & 0&0&1&0&0\\
 0 &  0 &  1 & 0&0&0&0.5&-2\\
 0 &  0 &  0 & 1&0&0&0&1
\end{array} \right)%
\grstep[R_1 \rightarrow R_1+2R_3]{R_2 \rightarrow R_2-6R_3}
\left(\begin{array}{rrrr|rrrr}
 1 & -2 &  0 & 0&1&0&1  &-1\\
 0 & -3 &  0 & 0&0&1&-3 &12\\
 0 &  0 &  1 & 0&0&0&0.5&-2\\
 0 &  0 &  0 & 1&0&0&0&1
\end{array} \right)%
\grstep[]{R_2 \rightarrow -R_2/3}
\\ \\
\left(\begin{array}{rrrr|rrrr}
 1 & -2 &  0 & 0&1&0&1  &-1\\
 0 &  1 &  0 & 0&0&-1/3&1 &-4\\
 0 &  0 &  1 & 0&0&0&0.5&-2\\
 0 &  0 &  0 & 1&0&0&0&1
\end{array} \right)%
\grstep[]{R_1 \rightarrow R_1+2R_2}
%
\left(\begin{array}{rrrr|rrrr}
 1 &  0 &  0 & 0&1&-2/3&3 &-9\\
 0 &  1 &  0 & 0&0&-1/3&1 &-4\\
 0 &  0 &  1 & 0&0&0&0.5&-2\\
 0 &  0 &  0 & 1&0&0&0&1
\end{array} \right) \Rightarrow
\\ \\

U^{-1} =\begin{pmatrix}
1&-2/3&3 &-9\\
0&-1/3&1 &-4\\
0&0&0.5&-2\\
0&0&0&1
\end{pmatrix}.
\end{array}
\end{equation*}

$$A^{-1}=U^{-1}L^{-1} =\begin{pmatrix}
1&-2/3&3 &-9\\
0&-1/3&1 &-4\\
0&0&0.5&-2\\
0&0&0&1
\end{pmatrix}\begin{pmatrix}
1&0&0&0\\
-3&1&0&0\\
1&0&1&0\\
17&-4&2&1
\end{pmatrix} =\begin{pmatrix}
-147 & \frac{106}{3} & -15 & -9\\
-66 & \frac{47}{3} & -7 & -4\\
\frac{-67}{2} & 8 & \frac{-7}{2} & -2\\
17 & -4 & 2 & 1
\end{pmatrix}\blacksquare
$$
% Problema 10.
%------------------------------------------------------------------------------
%------------------------------------------------------------------------------
\item Encuentre la descomposición LU de la matriz por bandas
\begin{equation*}
A=\left(\begin{array}{cccc}
a_{11} & a_{12} &   0    &    0  \\
a_{21} & a_{22} & a_{23} &    0  \\
    0  & a_{32} & a_{33} & a_{34}\\
    0  &   0    & a_{43} & a_{44}
\end{array} \right),
\end{equation*}
(Para una interesante aplicación de matrices por bandas a problemas de flujo de calor en física y la importancia de obtener su descomposición LU, ver problemas 31 y 32 de Linear Algebra, D. Lay, 4th ed., p. 131 y las explicaciones que ahí se dan.)

\res
Para hacer un poco más facil este problema, definamos las siguientes submatrices de $A$:
$$A_1^*=\begin{pmatrix}
a_{11} & a_{12}\\
a_{21} & a_{22}
\end{pmatrix}, \ \ A_2^*=\begin{pmatrix}
a_{11} & a_{12} & 0\\
a_{21} & a_{22} & a_{23}\\
0 & a_{32} & a_{33}
\end{pmatrix}.\Rightarrow $$
$$\det(A_1^*)=a_{11}a_{22}-a_{12}a_{21}, \ \ y \ \ \det(A_2^*)=a_{33}\det(A_1^*)-a_{32}a_{11}a_{23}.$$
Ahora utilizando (\textit{demostrado en clase}):\\
\textit{$A$ tiene una factorización $LU$ si y solo si todas sus submatrices principales líderes son no singulares.}
\\

Entonces para que $A$ tenga su factorización $LU$ asumimos que $[a_{11}]$, $A_1^*$ y $A_2^*$ son no singulares, es decir, $a_{11} \neq 0$, $\det(A_1^*)\neq 0$ y $\det(A_2^*)\neq 0.$\\
Ahora, considerando la metodología del inciso 9: Si $A =LU$, se sabe que $A$ se puede factorizar como:
\begin{equation*}
A=\left(\begin{array}{cccc}
a_{11} & a_{12} &   0    &    0  \\
a_{21} & a_{22} & a_{23} &    0  \\
    0  & a_{32} & a_{33} & a_{34}\\
    0  &   0    & a_{43} & a_{44}
\end{array} \right)=
\left(\begin{array}{rrrr}
 1 &  0 &  0 & 0\\
 a &  1 &  0 & 0\\
 b &  c &  1 & 0\\
 d &  e &  f & 1
\end{array} \right) \left(\begin{array}{rrrr}
 a_{11} &  a_{12} & 0 & 0\\
 0 &  u &  v & w\\
 0 &  0 &  x & y\\
 0 &  0 &  0 & z
\end{array} \right)=LU
\end{equation*}
Observe que el primer renglón de U es el mismo que el primer renglón de A porque al reducir A a la forma triangular, no hace falta modificar los elementos del primer renglón.\\
Se pueden obtener todos los coeficientes faltantes con tan sólo multiplicar las matrices. La
componente 2, 1 de $A$ es $a_{21}$. De este modo, el producto escalar del segundo renglón de $L$ y la primera columna de $U$ es igual a $a_{21}$:
$$a_{21}=a_{11}\cdots a \ \ \ \text{o}  \ \ \ \ a=\frac{a_{21}}{a_{11}}$$
Después se tiene:\\
componente $2,2$: $a_{22}=a_{12}\cdot \frac{a_{21}}{a_{11}} +u \rightarrow \ u=a_{22}-a_{12}\cdot \left. \frac{a_{21}}{a_{11}}\right.$\\
De aquí en adelante se pueden insertar los valores que se encuentran en $L$ y $U$:
\begin{table}[H]
\centering
\begin{tabular}{l|l}
componente $2,3$: $a_{23}=0+v \rightarrow \ v=a_{23}.$ & componente $3,4$: $a_{34}=0+y \rightarrow \ y=a_{34}.$\\ \\
componente $2,4$: $0=0+w \rightarrow \ w=0.$ & componente $4,1$: $0=a_{11}d \rightarrow \ d=0.$\\ \\
componente $3,1$: $0=a_{11}b \rightarrow \ b=0.$ & componente $4,2$: $0=\left(a_{22}-a_{12}\cdot \frac{a_{21}}{a_{11}}\right) e \rightarrow \ e=0.$\\ \\
componente $3,2$: $a_{32}=(a_{22}-a_{12}\cdot \left. \frac{a_{21}}{a_{11}}\right.)c \rightarrow \ c=\frac{a_{32}a_{11}}{a_{22}a_{11}-a_{12}a_{21}}.$ & componente $4,3$: $a_{43}=(\frac{a_{33}\cdot \det(A_1^*)-a_{23}a_{32}a_{11}}{\det(A_1^*)})f \rightarrow $\\ \\
componente $3,3$: $a_{33}=a_{23}\left(\frac{a_{32}a_{11}}{a_{22}a_{11}-a_{12}a_{21}}\right)+x \rightarrow$ & $f=\frac{a_{43}}{\frac{a_{33}\cdot \det(A_1^*)-a_{23}a_{32}a_{11}}{\det(A_1^*)}}=\frac{a_{43}\cdot \det(A_1^*)}{a_{33}\cdot \det(A_1^*)-a_{23}a_{32}a_{11}}$\\ \\ $x=a_{33}-a_{23}\left(\frac{a_{32}a_{11}}{a_{22}a_{11}-a_{12}a_{21}}\right)=\frac{a_{33}\cdot \det(A_1^*)-a_{23}a_{32}a_{11}}{\det(A_1^*)}$ & componente $4,4$: $a_{44}=a_{34}\frac{a_{43}\cdot \det(A_1^*)}{a_{33}\cdot \det(A_1^*)-a_{23}a_{32}a_{11}}+z \rightarrow $\\ \\
& $z=a_{44}-\frac{a_{34}a_{43}\cdot \det(A_1^*)}{a_{33}\cdot \det(A_1^*)-a_{23}a_{32}a_{11}}$.
\end{tabular}
\end{table}
Por lo tanto:
\begin{equation*}
A=\left(\begin{array}{cccc}
a_{11} & a_{12} &   0    &    0  \\
a_{21} & a_{22} & a_{23} &    0  \\
    0  & a_{32} & a_{33} & a_{34}\\
    0  &   0    & a_{43} & a_{44}
\end{array} \right)=
\end{equation*}
\begin{equation*}
\left(\begin{array}{cccc}
 1 &  0 &  0 & 0\\
 \frac{a_{21}}{a_{11}} &  1 &  0 & 0\\
 0 &  \frac{a_{32}a_{11}}{\det(A_1^*)} &  1 & 0\\
 0 &  0 &  \frac{a_{43}\cdot \det(A_1^*)}{a_{33}\cdot \det(A_1^*)-a_{23}a_{32}a_{11}} & 1
\end{array} \right) \left(\begin{array}{cccc}
 a_{11} &  a_{12} & 0 & 0\\
 0 & \frac{\det(A_1^*)}{a_{11}}&  a_{23} & 0\\
 0 &  0 &  \frac{a_{33}\cdot \det(A_1^*)-a_{23}a_{32}a_{11}}{\det(A_1^*)} & a_{34}\\
 0 &  0 &  0 & a_{44}-\frac{a_{34}a_{43}\cdot \det(A_1^*)}{a_{33}\cdot \det(A_1^*)-a_{23}a_{32}a_{11}}
\end{array} \right)
\end{equation*}

\begin{equation*}
=LU.
\end{equation*}

Ahora, recordando la notación del inicio podemos ver que:
$$\det (A_2^*)=a_{33}\det(A_1^*)-a_{23}a_{32}a_{11}, \ \ y\ \ \det(A)=a_{44}\cdot(a_{33}\cdot \det(A_1^*)-a_{23}a_{32}a_{11})-a_{34}a_{43}\det(A_1^*).$$
Es decir, el resultado anterior se puede simplificar a:
\begin{equation*}
A=\left(\begin{array}{cccc}
a_{11} & a_{12} &   0    &    0  \\
a_{21} & a_{22} & a_{23} &    0  \\
    0  & a_{32} & a_{33} & a_{34}\\
    0  &   0    & a_{43} & a_{44}
\end{array} \right)=
\left(\begin{array}{cccc}
 1 &  0 &  0 & 0\\
 \frac{a_{21}}{a_{11}} &  1 &  0 & 0\\
 0 &  \frac{a_{32}a_{11}}{\det(A_1^*)} &  1 & 0\\
 0 &  0 &  \frac{a_{43}\cdot \det(A_1^*)}{\det(A_2^*)} & 1
\end{array} \right) \left(\begin{array}{cccc}
 a_{11} &  a_{12} & 0 & 0\\
 0 & \frac{\det(A_1^*)}{a_{11}}&  a_{23} & 0\\
 0 &  0 &  \frac{\det(A_2^*)}{\det(A_1^*)} & a_{34}\\
 0 &  0 &  0 & \frac{\det(A)}{\det(A_2^*)}
\end{array} \right)=LU\finf
\end{equation*}
Algo curioso de esas matrices con bandas, es que se puede observar que tanto $L$ y $U$ igual son matrices por bandas.
\end{enumerate}

\end{document}