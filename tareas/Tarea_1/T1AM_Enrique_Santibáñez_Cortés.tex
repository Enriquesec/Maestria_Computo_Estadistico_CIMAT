\documentclass[11pt,letterpaper]{article}
\usepackage[utf8]{inputenc}
\usepackage[T1]{fontenc}
\usepackage[spanish]{babel}
\usepackage{amsmath}
\usepackage{amsfonts}
\usepackage{amssymb}
\usepackage{graphicx}
\usepackage{lmodern}
\usepackage{xspace}
\usepackage{multicol}
\usepackage{hyperref}
\usepackage{float}
\usepackage{hyperref}
\usepackage{color}

\newcommand{\azul}[1]{\textcolor{MaterialBlue900}{#1}}
\usepackage{array}

\hypersetup{colorlinks=true,   linkcolor=MaterialBlue900}
%\usepackage[colorlinks=true, linkcolor=black, urlcolor=blue, pdfborder={0 0 0}]{hyperref}

\usepackage[left=2cm,right=2cm,top=2cm,bottom=2cm]{geometry}
\title{Modelos no paramétricos y de regresión 2018-1}
\author{Tarea examen: pruebas binomiales y tablas de contingencia}
\date{Fecha de entrega: 08/01/2017}
\setlength{\parindent}{0in}
\spanishdecimal{.}


\newcommand{\X}{\mathbb{X}}
\newcommand{\x}{\mathbf{x}}
\newcommand{\Y}{\mathbf{Y}}
\newcommand{\y}{\mathbf{y}}
\newcommand{\xbarn}{\bar{x}_n}
\newcommand{\ybarn}{\bar{y}_n}
\newcommand{\paren}[1]{\left( #1 \right)}
\newcommand{\llaves}[1]{\left\lbrace #1 \right\rbrace}
\newcommand{\barra}{\,\vert\,}
\newcommand{\mP}{\mathbb{P}}
\newcommand{\mE}{\mathbb{E}}
\newcommand{\mI}{\mathbf{I}}
\newcommand{\mJ}{\mathbf{J}}
\newcommand{\mX}{\mathbf{X}}
\newcommand{\mS}{\mathbf{S}}
\newcommand{\mA}{\mathbf{A}}
\newcommand{\unos}{\boldsymbol{1}}
\newcommand{\xbarnv}{\bar{\mathbf{x}}_n}
\newcommand{\abs}[1]{\left\vert #1 \right\vert}
\newcommand{\muv}{\boldsymbol{\mu}}
\newcommand{\mcov}{\boldsymbol{\Sigma}}
\newcommand{\vbet}{\boldsymbol{\beta}}
\newcommand{\veps}{\boldsymbol{\epsilon}}
\newcommand{\mC}{\mathbf{C}}
\newcommand{\ceros}{\boldsymbol{0}}
\newcommand{\mH}{\mathbf{H}}
\newcommand{\ve}{\mathbf{e}}
\newcommand{\avec}{\mathbf{a}}
\newcommand{\res}{\textbf{RESPUESTA}\\}
\newcommand{\rojo}[1]{\textcolor{MaterialRed900}{#1}}

\newcommand{\defi}[3]{\textbf{Definición:#3}}
\newcommand{\fin}{$\blacksquare.$}
\newcommand{\finf}{\blacksquare.}
\newcommand{\tr}{\text{tr}}
\begin{document}
\begin{table}[ht]
\centering
\begin{tabular}{c}
\textbf{Maestría en Computo Estadístico}\\
\textbf{Álgebra Matricial} \\
\textbf{Tarea 1}\\
\today \\
\emph{Enrique Santibáñez Cortés}\\
Repositorio de Git: \href{https://github.com/Enriquesec/Inferencia_Estad-stica/tree/master/Tareas/Tarea_1}{Tarea 1, IE}.
\end{tabular}
\end{table}

\begin{enumerate}
\item Si $A$ es una matriz $m\times n$ dada por bloques de vectores columna como $$ (a_1\ a_2\ \cdots a_n)$$ y $B$ es una matriz $n\times p$ dada por bloques de vectores renglón como
\begin{equation*}
\left(\begin{array}{c}
v_1\\
v_2\\
\vdots \\
v_n
\end{array}
\right)
\end{equation*}
Demuestre que $$AB=\sum_{i=1}^na_iv_i.$$

\item Sean $A$ y $B$ matrices cuadradas del mismo orden. Demuestre que $(A-B)(A+B)=A^2-B^2$ si y solo si $AB=BA.$

\res
$\Rightarrow)$ Si $(A-B)(A+B)=A^2-B^2$ entonces:
\begin{equation*}
\begin{array}{ccc}
A^2-B^2&=&(A-B)(A+B)\\
&&\\
&=&AA-BA+AB-BB\\
&&\\
&=&A^2-BA+AB-B^2 \\
&&\\
BA&=&AB.
\end{array}
\end{equation*}
$\Leftarrow)$ Si $AB=AB$ entonces:
$$(A-B)(A+B)=AA-BA+AB-BB=A^2-B^2.$$
Por lo tanto, queda demostrado que $(A-B)(A+B)=A^2-B^2$ si y solo si $AB=BA.\ \ \finf$

\item Sean $A$ y $B$ matrices $n\times n$, $A\neq 0, \ B\neq 0,$ tales que $AB=BA.$ Demuestre que $A^pB^p=B^pA^p$ para cualesquiera $p,q\in \mathbb{N}.$

\res
Multiplicado por $A^{p-1}$ por la izquierda y $B^{q-1}$ por la derecha tenemos que:

$$A^{p-1}(AB)B^{q-1}=A^{p-1}(BA)B^{q-1} \ \ \ \ \finf$$


\item Se dice que una matriz cuadrada $A$ es antisimétrica si $A=-A^t.$ Demuestre que $A-A^t$ es antisimétrica.

\res
Considerando las propiedades de la transpuesta:
$$-(A-A^t)^t=-(A-A^t)=A-A^t\ \ \ \finf$$

\item Demuestre que dada cualquier matriz cuadrada $A,$ esta se puede escribir como la suma de una matriz simétrica y una matriz antisimétrica.

\item Se dice que una matriz cuadrada $P$ es idempotente si $P^2=P$. Si 
\begin{equation*}
A=\left(\begin{array}{cc}
I& P\\
0&P
\end{array}
\right)
\end{equation*}
y si $P$ es idempotente, encuentre $A^{500}.$

\res
\begin{equation*}
A^2=\left(\begin{array}{cc}
I& P\\
0&P
\end{array}
\right)\left(\begin{array}{cc}
I& P\\
0&P
\end{array}
\right)=
\left(\begin{array}{cc}
I^2+0& IP+P^2\\
0+0&0+P^2
\end{array}
\right)=\left(\begin{array}{cc}
I&2P\\
0&P
\end{array}
\right).
\end{equation*}

\begin{equation*}
A^3=\left(\begin{array}{cc}
I& 2P\\
0&P
\end{array}
\right)\left(\begin{array}{cc}
I& P\\
0&P
\end{array}
\right)=
\left(\begin{array}{cc}
I^2+0& IP+2P^2\\
0+0&0+P^2
\end{array}
\right)=\left(\begin{array}{cc}
I&3P\\
0&P
\end{array}
\right).
\end{equation*}


Por lo tanto 

\begin{equation*}
A^{500}=\left(\begin{array}{cc}
I& 500P\\
0&P
\end{array}
\right) \finf
\end{equation*}

\item Sean $A$ y $B$ matrices de tamaño $m\times n$. Demuestre que tr$(AB^t)=$tr$(A^tB)$. 

\res
Utilizando la propiedad de la traza de una matriz:
$$\tr(A)=\tr(A^t).$$
Y si 
Entonces,
$$\tr(AB^t)=\tr($$
\item Encuentre matrices $A, B \ \text{y} \ C$ tales que tr$(ABC)\neq$tr$(BAC)$.

\res
Por convicción definamos a $A=\left( \begin{array}{cc}
0&1 \\
1&0
\end{array} \right)$, y ahora sea $B=\left( \begin{array}{cc}
b_{11}&b_{12} \\
b_{21}&b_{22}
\end{array} \right)$ y $C=\left( \begin{array}{cc}
c_{11}&c_{12} \\
c_{21}&c_{22}
\end{array} \right)$ por lo que tenemos que:

\begin{equation*}
AB=\left( \begin{array}{cc}
0&1 \\
1&0
\end{array} \right)\left( \begin{array}{cc}
b_{11}&b_{12} \\
b_{21}&b_{22}
\end{array} \right)=\left( \begin{array}{cc}
b_{21}&b_{22} \\
b_{11}&b_{12}
\end{array} \right), \ \ \ BA=\left( \begin{array}{cc}
b_{11}&b_{12} \\
b_{21}&b_{22}
\end{array} \right)\left( \begin{array}{cc}
0&1 \\
1&0
\end{array} \right)=\left( \begin{array}{cc}
b_{12}&b_{11} \\
b_{22}&b_{21}
\end{array} \right).
\end{equation*}
Observemos que la primera multiplicación representa a un cambio de renglones y la segunda a un cambio de columnas. Ahora multipliquemos lo anterior por la matriz $C$ pero solo concentrarnos en los resultados de la diagonal. 
\begin{equation*}
ABC=\left( \begin{array}{cc}
b_{21}&b_{22} \\
b_{11}&b_{12}
\end{array} \right)\left( \begin{array}{cc}
c_{11}&c_{12} \\
c_{21}&c_{22}
\end{array} \right)=
\left( \begin{array}{cc}
b_{21}c_{11}+b_{22}c_{21} & \gamma_1\\
\gamma_2&b_{11}c_{12}+b_{12}c_{22}
\end{array} \right),
\end{equation*}

\begin{equation*}
BAC=\left( \begin{array}{cc}
b_{12}&b_{11} \\
b_{22}&b_{21}
\end{array} \right)\left( \begin{array}{cc}
c_{11}&c_{12} \\
c_{21}&c_{22}
\end{array} \right)=
\left( \begin{array}{cc}
b_{12}c_{11}+b_{11}c_{21}&\gamma_3 \\
\gamma_4& b_{12}c_{12}+b_{21}c_{22}
\end{array} \right),
\end{equation*}
donde $\gamma_i$ no son relevantes para este problema. Ahora calculemos la traza de ese producto de matrices:

\begin{equation*}
\tr (ABC)= b_{21}c_{11}+b_{22}c_{21}+b_{11}c_{12}+b_{12}c_{22}\ \ \text{y}\ \ \tr (BAC)= b_{12}c_{11}+b_{11}c_{21}+ b_{12}c_{12}+b_{21}c_{22}.
\end{equation*}
De lo anterior podemos observar que si tr$(ABC)\neq$tr$(BAC)$ ,
\begin{equation*}
\begin{array}{ccc}
b_{21}c_{11}+b_{22}c_{21}+b_{11}c_{12}+b_{12}c_{22} \neq 
b_{12}c_{11}+b_{11}c_{21}+ b_{12}c_{12}+b_{21}c_{22}\\
b_{21}c_{11}+b_{22}c_{21}+b_{11}c_{12}+b_{12}c_{22}-b_{12}c_{11}-b_{11}c_{21}-b_{12}c_{12}-b_{21}c_{22}\neq 0\\
c_{11}(b_{21}-b_{12})+c_{21}(b_{22}-b_{11})+c_{12}(b_{11}-b_{12})+c_{22}(b_{12}-b_{21})\neq 0\\
(b_{21}-b_{12})(c_{11}-c_{22})+(b_{22}-b_{11})(c_{21}-c_{12})\neq 0.
\end{array}
\end{equation*}
De lo anterior podemos observar que si $c_{ij}>0$ y $b_{ij}>0$ para $i=1,2, \ j=1,2$. Y además considerando las siguientes desigualdades:
\begin{equation*}
\begin{array}{ccc}
c_{11}>c_{22} &,& b_{21}>b_{12}\\
c_{21}>c_{12} &,& b_{22}>b_{11}
\end{array}
\end{equation*} 
Obtenemos un conjunto de matriz que cumplirán que $\tr(ABC)\neq \tr(BAC)$

\item Sea $L$ una matriz triangular inferior $n\times n$. Demuestre que $L=L_1L_2\cdots L_n$ donde $L_i$ es la matriz $n\times n$ que se obtiene reemplazando la $i-$ésima columna de $I_n$ por la $i-$ésima columna de $L$. Demuestre un resultado análogo para matrices triangulares superiores.

\res
Considerando que $L_i$ se puede interpretar como la matriz elemetenal por un escalar.

\item Sea $A=(a_{ij})$ una matriz cuadrada de tamaño $n$, triangular superior tal que $a_{ii}=0$ para $i=1, \cdots , n.$ Demuestre que para $i=1, \cdots , n$ y $j=1, \cdots , \min (n,i+p-1)$ se cumple que $b_{ij}=0$ donde $A^p=(b_{ij})$ y $p$ es un entero positivo. 

\res
Sea $A=(a_{ij})$ una matriz cuadrada de tamaño $n$, triangular superior tal que $a_{ii}=0$ para $i=1, \cdots , n$:
\begin{equation*}
A=\left[ \begin{array}{ccccc}
0& a_{12} & a_{13} & \cdots & a_{1n}\\
0 & 0 & a_{23} & \cdots & a_{2n}\\
0 & 0 & 0 & \cdots & a_{3n}\\
\vdots & \vdots & \vdots & \ddots & \vdots\\
0 & 0 & 0 & \cdots & 0
\end{array}\right]
\end{equation*}
Entonces veamos que 
\begin{equation*}
A^2=\left[ \begin{array}{ccccc}
0& a_{12} & a_{13} & \cdots & a_{1n}\\
0 & 0 & a_{23} & \cdots & a_{2n}\\
0 & 0 & 0 & \cdots & a_{3n}\\
\vdots & \vdots & \vdots & \ddots & \vdots\\
0 & 0 & 0 & \cdots & 0
\end{array}\right]
\left[ \begin{array}{ccccc}
0& a_{12} & a_{13} & \cdots & a_{1n}\\
0 & 0 & a_{23} & \cdots & a_{2n}\\
0 & 0 & 0 & \cdots & a_{3n}\\
\vdots & \vdots & \vdots & \ddots & \vdots\\
0 & 0 & 0 & \cdots & 0
\end{array}\right]= 
\left[ \begin{array}{ccccc}
0 & 0a_{12}+\sum_{j=2}^na_{1j}0 & a_{13} & \cdots & a_{1n}\\
0 & 0 & a_{23} & \cdots & a_{2n}\\
0 & 0 & 0 & \cdots & a_{3n}\\
\vdots & \vdots & \vdots & \ddots & \vdots\\
0 & 0 & 0 & \cdots & 0
\end{array}\right]
\end{equation*}

\end{enumerate}
\end{document}