\documentclass[11pt,letterpaper]{article}
\usepackage[utf8]{inputenc}
\usepackage[T1]{fontenc}
\usepackage[spanish]{babel}
\usepackage{amsmath}
\usepackage{amsfonts}
\usepackage{amssymb}
\usepackage{graphicx}
\usepackage{lmodern}
\usepackage{xspace}
\usepackage{multicol}
\usepackage{hyperref}
\usepackage{float}
\usepackage{hyperref}
\usepackage{color}

\newcommand{\azul}[1]{\textcolor{MaterialBlue900}{#1}}
\usepackage{array}

\hypersetup{colorlinks=true,   linkcolor=MaterialBlue900}
%\usepackage[colorlinks=true, linkcolor=black, urlcolor=blue, pdfborder={0 0 0}]{hyperref}

\usepackage[left=2cm,right=2cm,top=2cm,bottom=2cm]{geometry}
\title{Modelos no paramétricos y de regresión 2018-1}
\author{Tarea examen: pruebas binomiales y tablas de contingencia}
\date{Fecha de entrega: 08/01/2017}
\setlength{\parindent}{0in}
\spanishdecimal{.}


\newcommand{\X}{\mathbb{X}}
\newcommand{\x}{\mathbf{x}}
\newcommand{\Y}{\mathbf{Y}}
\newcommand{\y}{\mathbf{y}}
\newcommand{\xbarn}{\bar{x}_n}
\newcommand{\ybarn}{\bar{y}_n}
\newcommand{\paren}[1]{\left( #1 \right)}
\newcommand{\llaves}[1]{\left\lbrace #1 \right\rbrace}
\newcommand{\barra}{\,\vert\,}
\newcommand{\mP}{\mathbb{P}}
\newcommand{\mE}{\mathbb{E}}
\newcommand{\mI}{\mathbf{I}}
\newcommand{\mJ}{\mathbf{J}}
\newcommand{\mX}{\mathbf{X}}
\newcommand{\mS}{\mathbf{S}}
\newcommand{\mA}{\mathbf{A}}
\newcommand{\unos}{\boldsymbol{1}}
\newcommand{\xbarnv}{\bar{\mathbf{x}}_n}
\newcommand{\abs}[1]{\left\vert #1 \right\vert}
\newcommand{\muv}{\boldsymbol{\mu}}
\newcommand{\mcov}{\boldsymbol{\Sigma}}
\newcommand{\vbet}{\boldsymbol{\beta}}
\newcommand{\veps}{\boldsymbol{\epsilon}}
\newcommand{\mC}{\mathbf{C}}
\newcommand{\ceros}{\boldsymbol{0}}
\newcommand{\mH}{\mathbf{H}}
\newcommand{\ve}{\mathbf{e}}
\newcommand{\avec}{\mathbf{a}}
\newcommand{\res}{\textbf{RESPUESTA}\\}
\newcommand{\rojo}[1]{\textcolor{MaterialRed900}{#1}}

\newcommand{\defi}[3]{\textbf{Definición:#3}}
\newcommand{\fin}{$\blacksquare.$}
\newcommand{\finf}{\blacksquare.}
\newcommand{\tr}{\text{tr}}
\begin{document}
\begin{table}[ht]
\centering
\begin{tabular}{c}
\textbf{Maestría en Computo Estadístico}\\
\textbf{Álgebra Matricial} \\
\textbf{Tarea 1}\\
\today \\
\emph{Enrique Santibáñez Cortés}\\
Repositorio de Git: \href{https://github.com/Enriquesec/Inferencia_Estad-stica/tree/master/Tareas/Tarea_1}{Tarea 1, IE}.
\end{tabular}
\end{table}

\begin{enumerate}
\item Si $A$ es una matriz $m\times n$ dada por bloques de vectores columna como $$ (a_1\ a_2\ \cdots a_n)$$ y $B$ es una matriz $n\times p$ dada por bloques de vectores renglón como
\begin{equation*}
\left(\begin{array}{c}
v_1\\
v_2\\
\vdots \\
v_n
\end{array}
\right)
\end{equation*}
Demuestre que $$AB=\sum_{i=1}^na_iv_i.$$
\res
Como $A$ es una matriz $m\times n$ dada por bloques columnas, como
$$(a_1\ a_2\ \cdots a_n)$$
donde $a_i$ es un vector columna 
$$a_i=\left( \begin{array}{c}
a_{1i}\\
a_{2i}\\
\vdots\\
a_{ni}
\end{array}\right).$$
Del mismo modo podemos decir que $B$ esta constituida por $b_i$ vectores renglón:
$$v_i=(\begin{array}{cccc}
v_{i1} & v_{i2} & \cdots &v_{in}
\end{array} ).$$
Entonces ocupando lo anterior
$$AB=\left( \begin{array}{cccc}
\left( \begin{array}{c}
a_{11}\\
a_{21}\\
\vdots\\
a_{n1}
\end{array}\right) & 
\left( \begin{array}{c}
a_{12}\\
a_{22}\\
\vdots\\
a_{n2}
\end{array}\right)& \cdots &
\left( \begin{array}{c}
a_{1n}\\
a_{2n}\\
\vdots\\
a_{nn}
\end{array}\right)
\end{array} \right) \left( \begin{array}{c}
\left(\begin{array}{cccc}
v_{11} & v_{12} & \cdots &v_{1n}
\end{array} \right) \\
\left(\begin{array}{cccc}
v_{21} & v_{22} & \cdots &v_{2n}
\end{array} \right)\\
\vdots\\
\left(\begin{array}{cccc}
v_{n1} & v_{n2} & \cdots &v_{nn}
\end{array} \right)
\end{array} \right)
$$
$$
=\begin{array}{cccc}
\left( \begin{array}{c}
a_{11}\\
a_{21}\\
\vdots\\
a_{n1}
\end{array}\right) \left(\begin{array}{cccc}
v_{11} & v_{12} & \cdots &v_{1n}
\end{array} \right)+\cdots+\left( \begin{array}{c}
a_{1n}\\
a_{2n}\\
\vdots\\
a_{nn}
\end{array}\right) \left(\begin{array}{cccc}
v_{n1} & v_{n2} & \cdots &v_{nn}
\end{array} \right)
\end{array}$$
$$=\begin{array}{c}
a_1v_1+a_2v_2+\cdots+a_nv_n
\end{array}=\sum_{i=1}^na_iv_i \ \ \finf$$
\item Sean $A$ y $B$ matrices cuadradas del mismo orden. Demuestre que $(A-B)(A+B)=A^2-B^2$ si y solo si $AB=BA.$

\res
$\Rightarrow)$ Si $(A-B)(A+B)=A^2-B^2$ entonces:
\begin{equation*}
\begin{array}{ccc}
A^2-B^2&=&(A-B)(A+B)\\
&&\\
&=&AA-BA+AB-BB\\
&&\\
&=&A^2-BA+AB-B^2 \\
&&\\
BA&=&AB.
\end{array}
\end{equation*}
$\Leftarrow)$ Si $AB=AB$ entonces:
$$(A-B)(A+B)=AA-BA+AB-BB=A^2-B^2.$$
Por lo tanto, queda demostrado que $(A-B)(A+B)=A^2-B^2$ si y solo si $AB=BA.\ \ \finf$

\item Sean $A$ y $B$ matrices $n\times n$, $A\neq 0, \ B\neq 0,$ tales que $AB=BA.$ Demuestre que $A^pB^q=B^qA^p$ para cualesquiera $p,q\in \mathbb{N}.$

\res
Multiplicado por $A^{p-1}$ (donde $p\in \mathbb{N}$) por la izquierda a $AB=BA$ tenemos 
\begin{equation*}
\begin{array}{ccc}
A^{p-1}AB&=&A^{p-1}BA\\
&=&A^{p-2}(AB)A=A^{p-2}(BA)A\\
&=&A^{p-3}(AB)A^2=A^{p-3}(BA)A^2\\
&=&\vdots\\
&=&(AB)A^{p-1}=(BA)A^{p-1}.
\end{array}
\end{equation*}
Simplificando de ambos lados, tenemos que $A^pB=BA^p.$ Ahora multiplicamos al resultado obtenido por la matriz $B^{q-1}$ (donde $q\in \mathbb{N}$) por la derecha tenemos 
\begin{equation*}
\begin{array}{ccc}
A^pBB^{q-1}&=&BA^{p}B^{q-1}\\
&=&BA^{p-1}(AB)B^{q-2}=BA^{p-1}(BA)B^{q-2}\\
&=&BA^{p-2}(AB)AB^{q-2}=BA^{p-2}(BA)AB^{q-2}\\
&=&BA^{p-3}(AB)A^2B^{q-2}=BA^{p-3}(BA)A^2B^{q-2}\\
&\vdots&\\
&=&B(AB)A^{p-1}B^{q-2}=B(BA)A^{p-1}B^{q-2}\\
&=&B^2A^{p}B^{q-2}\\
&=&B^2A^{p-1}(AB)B^{q-3}=B^2A^{p-1}(BA)B^{q-3}\\
&=&B^2A^{p-2}(AB)AB^{q-3}=B^2A^{p-2}(BA)AB^{q-3}\\
&=&B^2A^{p-3}(AB)A^2B^{q-3}=B^2A^{p-3}(BA)A^2B^{q-3}\\
&\vdots&\\
&=&B^2(AB)A^{p-1}B^{q-3}=B^2(BA)A^{p-1}B^{q-3}\\
&=&B^3A^{p}B^{q-3}\\
&\vdots&\\
\end{array}
\end{equation*}

\begin{equation*}
\begin{array}{ccc}
&\vdots&\\
&=&B^{q-1}A^{p-1}(AB)=B^{q-1}A^{p-1}(BA)\\
&=&B^{q-1}A^{p-2}(AB)A=B^{q-1}A^{p-2}(BA)A\\
&=&B^{q-1}A^{p-3}(AB)A^2=B^{q-1}A^{p-3}(BA)A^2\\
&\vdots&\\
&=&B^{q-1}(AB)A^{p-1}=B^{q-1}(BA)A^{p-1}\\
&=&B^qA^{p}.
\end{array}
\end{equation*}
Por lo tanto, queda demostrado que si $AB=BA$ y para cualesquiera $p,q\in \mathbb{N}$ se cumple que $A^pB^q=B^qA^p \ \ \ \finf$ 
\item Se dice que una matriz cuadrada $A$ es antisimétrica si $A=-A^t.$ Demuestre que $A-A^t$ es antisimétrica.

\res
Sea $B=A-A^t$, entonces:
$$B^t=(A-A^t)^t=A^t-A.$$
y $$-B^t=-(A^t-A)=A-A^t.$$
Como $B=-B^t$, podemos concluir que   $A-A^t$ es antisimétrica. \ \ \fin

\item Demuestre que dada cualquier matriz cuadrada $A,$ esta se puede escribir como la suma de una matriz simétrica y una matriz antisimétrica.

\res
Es sencillo
\item Se dice que una matriz cuadrada $P$ es idempotente si $P^2=P$. Si 
\begin{equation*}
A=\left(\begin{array}{cc}
I& P\\
0&P
\end{array}
\right)
\end{equation*}
y si $P$ es idempotente, encuentre $A^{500}.$

\res
Encontremos una formula para encontrar a $A^n$. Primero veamos que pasa cuando $n=2,3$:
\begin{equation*}
A^2=\left(\begin{array}{cc}
I& P\\
0&P
\end{array}
\right)\left(\begin{array}{cc}
I& P\\
0&P
\end{array}
\right)=
\left(\begin{array}{cc}
I^2+0& IP+P^2\\
0+0&0+P^2
\end{array}
\right)=\left(\begin{array}{cc}
I&2P\\
0&P
\end{array}
\right).
\end{equation*}
\begin{equation*}
A^3=\left(\begin{array}{cc}
I& 2P\\
0&P
\end{array}
\right)\left(\begin{array}{cc}
I& P\\
0&P
\end{array}
\right)=
\left(\begin{array}{cc}
I^2+0& IP+2P^2\\
0+0&0+P^2
\end{array}
\right)=\left(\begin{array}{cc}
I&3P\\
0&P
\end{array}
\right).
\end{equation*}
De lo anterior y considerando $n\in \mathbb{N}$ podemos suponer que se cumple que :
\begin{equation*}
A^n= \left(\begin{array}{cc}
I&nP\\
0&P
\end{array}
\right).
\end{equation*}
Demostremos lo anterior de forma inductiva: \\
\textbf{Paso 1.} Mostrar que se cumple para $n=2,3$ o para algún $n$. Por construcción se cumple este paso.\\
\textbf{Paso 2.} Suponer que se cumple para $n$.\\
\textbf{Paso 3.} Demostrar que se cumple para $n+1$. Considerando el paso 2, tenemos que:
\begin{equation*}
A^{n+1}=A^nA=\left(\begin{array}{cc}
I& nP\\
0&P
\end{array}
\right)\left(\begin{array}{cc}
I& P\\
0&P
\end{array}
\right)=\left(\begin{array}{cc}
I^2+0& IP+nP^2\\
0+0&0+P^2
\end{array}
\right)=\left(\begin{array}{cc}
I&(n+1)P\\
0&P
\end{array}
\right).
\end{equation*}
Queda demostrado que para $n\in \mathbb{N}$ se cumple
\begin{equation*}
A^n= \left(\begin{array}{cc}
I&nP\\
0&P
\end{array}
\right).
\end{equation*}
Por lo tanto, utilizando la formula encontrada podemos concluir que 
\begin{equation*}
A^{500}=\left(\begin{array}{cc}
I& 500P\\
0&P
\end{array}
\right)\ \ \  \finf
\end{equation*}

\item Sean $A$ y $B$ matrices de tamaño $m\times n$. Demuestre que tr$(AB^t)=$tr$(A^tB)$. 

\res
Sea $A_{m\times n}, B_{n\times m}$ matrices. Entonces se cumple que (\textit{se demostraron en clase}):
\begin{itemize}
\item $\tr(A)=\tr(A^t).$
\item $\tr(AB)=\tr(BA)$.
\end{itemize}
Ocupando las dos propiedades de la traza anteriores tenemos que:
$$\tr(AB^t)=\tr((AB^t)^t)=\tr(BA^t)=\tr(A^tB). \ \ \finf$$
\item Encuentre matrices $A, B \ \text{y} \ C$ tales que tr$(ABC)\neq$tr$(BAC)$.

\res
Por convicción definamos a $A=\left( \begin{array}{cc}
0&1 \\
1&0
\end{array} \right)$, y ahora sea $B=\left( \begin{array}{cc}
b_{11}&b_{12} \\
b_{21}&b_{22}
\end{array} \right)$ y $C=\left( \begin{array}{cc}
c_{11}&c_{12} \\
c_{21}&c_{22}
\end{array} \right)$. Entonces 
\begin{equation*}
AB=\left( \begin{array}{cc}
0&1 \\
1&0
\end{array} \right)\left( \begin{array}{cc}
b_{11}&b_{12} \\
b_{21}&b_{22}
\end{array} \right)=\left( \begin{array}{cc}
b_{21}&b_{22} \\
b_{11}&b_{12}
\end{array} \right)\ \  \text{y} \ \ \ BA=\left( \begin{array}{cc}
b_{11}&b_{12} \\
b_{21}&b_{22}
\end{array} \right)\left( \begin{array}{cc}
0&1 \\
1&0
\end{array} \right)=\left( \begin{array}{cc}
b_{12}&b_{11} \\
b_{22}&b_{21}
\end{array} \right).
\end{equation*}
Observemos que la primera multiplicación representa a un cambio de renglones y la segunda a un cambio de columnas. Ahora multipliquemos de lado izquierdo lo anterior por la matriz $C$ pero solo concentrarnos en los resultados de la diagonal. 
\begin{equation*}
ABC=\left( \begin{array}{cc}
b_{21}&b_{22} \\
b_{11}&b_{12}
\end{array} \right)\left( \begin{array}{cc}
c_{11}&c_{12} \\
c_{21}&c_{22}
\end{array} \right)=
\left( \begin{array}{cc}
b_{21}c_{11}+b_{22}c_{21} & \gamma_1\\
\gamma_2&b_{11}c_{12}+b_{12}c_{22}
\end{array} \right),
\end{equation*}

\begin{equation*}
BAC=\left( \begin{array}{cc}
b_{12}&b_{11} \\
b_{22}&b_{21}
\end{array} \right)\left( \begin{array}{cc}
c_{11}&c_{12} \\
c_{21}&c_{22}
\end{array} \right)=
\left( \begin{array}{cc}
b_{12}c_{11}+b_{11}c_{21}&\gamma_3 \\
\gamma_4& b_{22}c_{12}+b_{21}c_{22}
\end{array} \right),
\end{equation*}
donde $\gamma_i$ no son relevantes para este problema. Por lo que la traza de esos producto de matrices es:
\begin{equation*}
\tr (ABC)= b_{21}c_{11}+b_{22}c_{21}+b_{11}c_{12}+b_{12}c_{22}\ \ \text{y}\ \ \tr (BAC)= b_{12}c_{11}+b_{11}c_{21}+ b_{22}c_{12}+b_{21}c_{22}.
\end{equation*}
De lo anterior podemos observar que si tr$(ABC)\neq$tr$(BAC)$ ,
\begin{equation*}
\begin{array}{ccc}
b_{21}c_{11}+b_{22}c_{21}+b_{11}c_{12}+b_{12}c_{22} \neq 
b_{12}c_{11}+b_{11}c_{21}+ b_{22}c_{12}+b_{21}c_{22}\\
b_{21}c_{11}+b_{22}c_{21}+b_{11}c_{12}+b_{12}c_{22}-b_{12}c_{11}-b_{11}c_{21}-b_{22}c_{12}-b_{21}c_{22}\neq 0\\
c_{11}(b_{21}-b_{12})+c_{21}(b_{22}-b_{11})+c_{12}(b_{11}-b_{22})+c_{22}(b_{12}-b_{21})\neq 0\\
(b_{21}-b_{12})(c_{11}-c_{22})+(b_{22}-b_{11})(c_{21}-c_{12})\neq 0.
\end{array}
\end{equation*}
Entonces de lo anterior podemos encontrar un conjunto de elementos de las matrices $B$ y $C$:
\begin{equation*}
\begin{array}{ccc}
c_{11}>c_{22} &,& b_{21}>b_{12}\\
c_{21}>c_{12} &,& b_{22}>b_{11}
\end{array}
\end{equation*} 
tal que tr$(ABC)\neq$tr$(BAC), $ donde $A=\left( \begin{array}{cc}
0&1 \\
1&0
\end{array} \right).$\\
Entonces una tripleta de matrices que cumple que tr$(ABC)\neq$tr$(BAC),$ son: $A=\left( \begin{array}{cc}
0&1 \\
1&0
\end{array} \right),$ $B=\left( \begin{array}{cc}
2&3 \\
4&5
\end{array} \right)$ y $C=\left( \begin{array}{cc}
4&2 \\
3&1
\end{array} \right).$ Calculemos la trazas para mostrar que efectivamente se cumple. 
\begin{equation*}
AB=\left( \begin{array}{cc}
0&1 \\
1&0
\end{array} \right)\left( \begin{array}{cc}
2 & 3\\
4 & 5
\end{array} \right)=\left( \begin{array}{cc}
4 & 5\\
2 & 3
\end{array} \right)\ \  \text{y} \ \ \ BA=\left( \begin{array}{cc}
2 & 3\\
4 & 5
\end{array} \right)\left( \begin{array}{cc}
0&1 \\
1&0
\end{array} \right)=\left( \begin{array}{cc}
3 & 2\\
5 & 4
\end{array} \right).
\end{equation*}
Ahora calculemos la multiplicación con la matriz $C$:
\begin{equation*}
ABC=\left( \begin{array}{cc}
4 & 5\\
2 & 3
\end{array} \right)\left( \begin{array}{cc}
4 & 2\\
3 & 1
\end{array} \right)=
\left( \begin{array}{cc}
14+15 & \gamma_1\\
\gamma_2&4+3
\end{array} \right)=
\left( \begin{array}{cc}
29 & \gamma_1\\
\gamma_2&7
\end{array} \right),
\end{equation*}
\begin{equation*}
BAC=\left( \begin{array}{cc}
3 & 2\\
5 & 4
\end{array} \right)\left( \begin{array}{cc}
4 & 2\\
3 & 1
\end{array} \right)=
\left( \begin{array}{cc}
12+6&\gamma_3 \\
\gamma_4& 10+4
\end{array} \right)=
\left( \begin{array}{cc}
18&\gamma_3 \\
\gamma_4& 14
\end{array} \right),
\end{equation*}
donde $\gamma_i$ no son relevantes para este problema. Por lo que la traza de esos producto de matrices es:
\begin{equation*}
\tr (ABC)= 29+7=36\ \ \text{y}\ \ \tr (BAC)= 18+14=32.
\end{equation*}
Por lo que se cumple que tr$(ABC)\neq$tr$(BAC)$ para las matrices propuestas. $\ \ \ \finf$

\item Sea $L$ una matriz triangular inferior $n\times n$. Demuestre que $L=L_1L_2\cdots L_n$ donde $L_i$ es la matriz $n\times n$ que se obtiene reemplazando la $i-$ésima columna de $I_n$ por la $i-$ésima columna de $L$. Demuestre un resultado análogo para matrices triangulares superiores.

\res
Sea $L$ la matriz triangular inferir $n\times n$:
\begin{equation*}
L=\left( \begin{array}{ccccc}
l_{11}&0 & 0 & \cdots &0\\
l_{21}&l_{22}& 0 & \cdots & 0\\
l_{31}&l_{32}& l_{33}& \cdots & 0\\
\vdots & \vdots & \vdots & \ddots & \vdots\\
l_{n1}&l_{n2}&l_{n3}& \cdots & l_{nn}
\end{array}\right).
\end{equation*}	 
Entonces $L_i$ están definidas como:
\begin{equation*}
L_1=\left( \begin{array}{ccccc}
l_{11}&0 & 0 & \cdots &0\\
l_{21}&1& 0 & \cdots & 0\\
l_{31}&0&1& \cdots & 0\\
\vdots & \vdots & \vdots & \ddots & \vdots\\
l_{n1}&0&0& \cdots &1
\end{array}\right), \ L_2=\left( \begin{array}{ccccc}
1&0 & 0 & \cdots &0\\
0&l_{22}& 0 & \cdots & 0\\
0&l_{32}& 1& \cdots & 0\\
\vdots & \vdots & \vdots & \ddots & \vdots\\
0&l_{n2}&0& \cdots & 1
\end{array}\right), \cdots , L_n=\left( \begin{array}{ccccc}
1&0 & 0 & \cdots &0\\
0&1& 0 & \cdots & 0\\
0&0& 1& \cdots & 0\\
\vdots & \vdots & \vdots & \ddots & \vdots\\
0&0&0& \cdots & l_{nn}
\end{array}\right).
\end{equation*}


\item Sea $A=(a_{ij})$ una matriz cuadrada de tamaño $n$, triangular superior tal que $a_{ii}=0$ para $i=1, \cdots , n.$ Demuestre que para $i=1, \cdots , n$ y $j=1, \cdots , \min (n,i+p-1)$ se cumple que $b_{ij}=0$ donde $A^p=(b_{ij})$ y $p$ es un entero positivo. 

\res
Sea $A=(a_{ij})$ una matriz cuadrada de tamaño $n$, triangular superior tal que $a_{ii}=0$ para $i=1, \cdots , n$:
\begin{equation*}
A=\left( \begin{array}{ccccc}
0& a_{12} & a_{13} & \cdots & a_{1n}\\
0 & 0 & a_{23} & \cdots & a_{2n}\\
0 & 0 & 0 & \cdots & a_{3n}\\
\vdots & \vdots & \vdots & \ddots & \vdots\\
0 & 0 & 0 & \cdots & 0
\end{array}\right)
\end{equation*}
Entonces veamos que 
\begin{equation*}
A^2=\left[ \begin{array}{ccccc}
0& a_{12} & a_{13} & \cdots & a_{1n}\\
0 & 0 & a_{23} & \cdots & a_{2n}\\
0 & 0 & 0 & \cdots & a_{3n}\\
\vdots & \vdots & \vdots & \ddots & \vdots\\
0 & 0 & 0 & \cdots & 0
\end{array}\right]
\left[ \begin{array}{ccccc}
0& a_{12} & a_{13} & \cdots & a_{1n}\\
0 & 0 & a_{23} & \cdots & a_{2n}\\
0 & 0 & 0 & \cdots & a_{3n}\\
\vdots & \vdots & \vdots & \ddots & \vdots\\
0 & 0 & 0 & \cdots & 0
\end{array}\right]= 
\left[ \begin{array}{ccccc}
0 & 0a_{12}+\sum_{j=2}^na_{1j}0 & a_{13} & \cdots & a_{1n}\\
0 & 0 & a_{23} & \cdots & a_{2n}\\
0 & 0 & 0 & \cdots & a_{3n}\\
\vdots & \vdots & \vdots & \ddots & \vdots\\
0 & 0 & 0 & \cdots & 0
\end{array}\right]
\end{equation*}

\end{enumerate}
\end{document}