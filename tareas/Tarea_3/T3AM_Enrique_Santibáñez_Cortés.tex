\documentclass[11pt,letterpaper]{article}
\usepackage[utf8]{inputenc}
\usepackage[T1]{fontenc}
\usepackage[spanish]{babel}
\usepackage{amsmath}
\usepackage{amsfonts}
\usepackage{amssymb}
\usepackage{graphicx}
\usepackage{lmodern}
\usepackage{xspace}
\usepackage{multicol}
\usepackage{hyperref}
\usepackage{float}
\usepackage{hyperref}
\usepackage{color}

\newcommand{\azul}[1]{\textcolor{MaterialBlue900}{#1}}
\usepackage{array}

\hypersetup{colorlinks=true,   linkcolor=MaterialBlue900}
%\usepackage[colorlinks=true, linkcolor=black, urlcolor=blue, pdfborder={0 0 0}]{hyperref}

\usepackage[left=2cm,right=2cm,top=2cm,bottom=2cm]{geometry}
\title{Modelos no paramétricos y de regresión 2018-1}
\author{Tarea examen: pruebas binomiales y tablas de contingencia}
\date{Fecha de entrega: 08/01/2017}
\setlength{\parindent}{0in}
\spanishdecimal{.}


\newcommand{\X}{\mathbb{X}}
\newcommand{\x}{\mathbf{x}}
\newcommand{\Y}{\mathbf{Y}}
\newcommand{\y}{\mathbf{y}}
\newcommand{\xbarn}{\bar{x}_n}
\newcommand{\ybarn}{\bar{y}_n}
\newcommand{\paren}[1]{\left( #1 \right)}
\newcommand{\llaves}[1]{\left\lbrace #1 \right\rbrace}
\newcommand{\barra}{\,\vert\,}
\newcommand{\mP}{\mathbb{P}}
\newcommand{\mE}{\mathbb{E}}
\newcommand{\mI}{\mathbf{I}}
\newcommand{\mJ}{\mathbf{J}}
\newcommand{\mX}{\mathbf{X}}
\newcommand{\mS}{\mathbf{S}}
\newcommand{\mA}{\mathbf{A}}
\newcommand{\unos}{\boldsymbol{1}}
\newcommand{\xbarnv}{\bar{\mathbf{x}}_n}
\newcommand{\abs}[1]{\left\vert #1 \right\vert}
\newcommand{\muv}{\boldsymbol{\mu}}
\newcommand{\mcov}{\boldsymbol{\Sigma}}
\newcommand{\vbet}{\boldsymbol{\beta}}
\newcommand{\veps}{\boldsymbol{\epsilon}}
\newcommand{\mC}{\mathbf{C}}
\newcommand{\ceros}{\boldsymbol{0}}
\newcommand{\mH}{\mathbf{H}}
\newcommand{\ve}{\mathbf{e}}
\newcommand{\avec}{\mathbf{a}}
\newcommand{\res}{\textbf{RESPUESTA}\\}
\newcommand{\rojo}[1]{\textcolor{MaterialRed900}{#1}}

\newcommand{\defi}[3]{\textbf{Definición:#3}}
\newcommand{\fin}{$\blacksquare.$}
\newcommand{\finf}{\blacksquare.}
\newcommand{\tr}{\text{tr}}
\newcommand*{\temp}{\multicolumn{1}{r|}{}}

\newcommand{\grstep}[2][\relax]{%
   \ensuremath{\mathrel{
       {\mathop{\longrightarrow}\limits^{#2\mathstrut}_{
                                     \begin{subarray}{l} #1 \end{subarray}}}}}}
\newcommand{\swap}{\leftrightarrow}

\begin{document}
\begin{table}[ht]
\centering
\begin{tabular}{c}
\textbf{Maestría en Computo Estadístico}\\
\textbf{Álgebra Matricial} \\
\textbf{Tarea 3}\\
\today \\
\emph{Enrique Santibáñez Cortés}\\
Repositorio de Git: \href{https://github.com/Enriquesec/Algebra_matricial/tree/feature/tareas/tareas/Tarea_3}{Tarea 3, AM}.
\end{tabular}
\end{table}
Todos los cálculos deben ser a mano.

\begin{enumerate}
% Problema 1.
%------------------------------------------------------------------------------
%------------------------------------------------------------------------------
\item Dada la matriz
\begin{equation*}
\left( \begin{array}{rrrr}
-4 & 5 & -6 &7\\
-1 & 1 &  1 &3\\
1  & 2 & -3 &-1
\end{array} \right)
\end{equation*}
encuentre su forma escalonada reducida por renglones. Escriba todas las matrices elementales correspondientes a las operaciones que usó para llevar la matriz a la forma que obtuvo.

\res
Realizando las operaciones elementales tenemos que 
\begin{equation*}
\begin{array}{c}
\begin{pmatrix}
-4 & 5 & -6 &7\\
-1 & 1 &  1 &3\\
1  & 2 & -3 &-1
\end{pmatrix}
%
\grstep[]{R1 \Longleftrightarrow R3}
 %
\begin{pmatrix}
1  & 2 & -3 &-1\\
-1 & 1 &  1 &3\\
-4 & 5 & -6 &7
\end{pmatrix}%
\grstep[R3 \rightarrow R_3 +4R_1]{R_2 \rightarrow R_2 +R_1}
%
\begin{pmatrix}
1  & 2 & -3 &-1\\
0 & 3  & -2 & 2\\
0 & 13 & -18 &3
\end{pmatrix}%
\grstep[]{R2 \rightarrow R2/3}
\\
\\
\begin{pmatrix}
1  & 2 & -3 &-1\\
0 &  1 & -2/3 & 2/3\\
0 & 13 & -18 &3
\end{pmatrix}%
\grstep[]{R3 \rightarrow R3-13R2}
%
\begin{pmatrix}
1  & 2 & -3 &-1\\
0 &  1 & -2/3 & 2/3\\
0 &  0 & -28/3 & 17/3
\end{pmatrix}%
\grstep[]{R3 \rightarrow -3R3/28}
%
\begin{pmatrix}
1  & 2 & -3 &-1\\
0 &  1 & -2/3 & 2/3\\
0 &  0 & 1 & -17/28
\end{pmatrix}.
\end{array}
\end{equation*}
Lo anterior se puede escribir como las matrices elementales:
$$E_3(-3/28)E_32(-13)E_2(1/3)E_{31}(4)E_{21}(1)E_{13}\ \ \ \ \finf$$
% Problema 2.
%------------------------------------------------------------------------------
%------------------------------------------------------------------------------
\item Dada el sistema $Ax=b$, donde 
\begin{equation*}
A=\left( \begin{array}{rrr}
1 & 3 & -1\\
a_1 & -1 & -3\\
1 & 2 &2
\end{array} \right) \ \ \ \text{y} \ \ \ b=\left(\begin{array}{c}
0\\
1\\
a_2
\end{array} \right)
\end{equation*}
encuentre condiciones generales sobre $a_1$ y $a_2$ para que el sistema sea consistente. Si se quiere que la solución sea exactamente $x=(3,-1,2)^t$ , ¿qué valores deben
tener $a_1$ y $a_2$?

\res Ocupemos Gauss Jordan para encontrar su forma escalonada:
\begin{equation*}
\begin{array}{c}
\left( \begin{array}{rrr|r}
1 & 3 & -1 & 0\\
a_1 & -1 & -3 & 1 \\
1 & 2 & 2 & a_2
\end{array} \right) %
\grstep[R3 \rightarrow R_3 -R_1]{R_2 \rightarrow R_2 -a_1R_1}
%
\left( \begin{array}{rrr|r}
1 & 3 & -1 & 0\\
0 & -1-3a_1 & -3+a_1 & 1 \\
0 & -1 & 3 & a_2
\end{array} \right) %
\grstep[]{R_2 \Longleftrightarrow R3}
%
\left( \begin{array}{rrr|r}
1 & 3 & -1 & 0\\
0 & -1 & 3 & a_2\\
0 & -1-3a_1 & -3+a_1 & 1 \\
\end{array} \right)\\
\\
%
\grstep[]{R_3 \rightarrow R_3-(1+3a_1)R_1}
%
\left( \begin{array}{rrr|r}
1 &  3 & -1 & 0\\
0 & -1 & 3 & a_2\\
0 &  0 & -6-8a_1 &  1-a2-a_2a_1\\
\end{array} \right)
\end{array}
\end{equation*}
Por lo tanto el sistema es inconsistente cuando no tiene ninguna solución. El sistema anterior no tiene solución si $-6-8a_1=0$ y $1-a_2-a_2a_1\neq 0$.\\

Si $-6-8a_1\neq 0$, tenemos que 
\begin{equation*}
\left( \begin{array}{rrr|r}
1 &  3 & -1 & 0\\
0 & -1 & 3 & a_2\\
0 &  0 & 1 &  \frac{1-a2-a_2a_1}{-6-8a_1}\\
\end{array} \right).
\end{equation*}

% Problema 3.
%------------------------------------------------------------------------------
%------------------------------------------------------------------------------
\item Encuentre la solución general, escribiéndola como combinación lineal de vectores, del sistema homogéneo $Ax=0$ donde

\begin{equation*}
A=\left(\begin{array}{rrrrrr}
1  & -3 & 1 & -1 &  0 & 1\\
-1 &  3 & 0 &  3 &  1 & 3 \\
2  & -6 & 3 &  0 & -1 & 2\\
-1 &  3 & 1 &  5 &  1 & 6
\end{array}\right).
\end{equation*}

% Problema 4.
%------------------------------------------------------------------------------
%------------------------------------------------------------------------------
\item Encuentra la inversa de 
\begin{equation*}
\left(\begin{array}{rrr}
 1 &  0 & -2\\
-3 &  1 &  4\\
 2 & -3 &  4
\end{array}\right).
\end{equation*}

\res
Usando Gauss Jordan
\begin{equation*}
\begin{array}{c}
\left(\begin{array}{rrr|rrr}
 1 &  0 & -2 & 1 & 0 & 0 \\
-3 &  1 &  4 & 0 & 1 & 0\\
 2 & -3 &  4 & 0 & 0 & 1
\end{array}\right)%
\grstep[R3 \rightarrow R_3 -2R_1]{R_2 \rightarrow R_2 -3R_1}
%
\left(\begin{array}{rrr|rrr}
 1 &  0 & -2 &  1 & 0 & 0\\
 0 &  1 & -2 &  3 & 1 & 0\\
 0 & -3 &  8 & -2 & 0 & 1 
\end{array}\right)%
\grstep[]{R_3 \rightarrow R_3+3R2}
%
\left(\begin{array}{rrr|rrr}
 1 &  0 & -2 &  1 & 0 & 0\\
 0 &  1 & -2 &  3 & 1 & 0\\
 0 &  0 &  2 &  7 & 3 & 1 
\end{array}\right)\\
\\
%
\grstep[]{R_3 \rightarrow R_3/2}
% 
\left(\begin{array}{rrr|rrr}
 1 &  0 & -2 &  1 & 0 & 0\\
 0 &  1 & -2 &  3 & 1 & 0\\
 0 &  0 &  1 &  3.5 & 1.5 & 0.5 
\end{array}\right)%
\grstep[R2 \rightarrow R_2+2R_3]{R_1 \rightarrow R_1+2R_3}
%
\left(\begin{array}{rrr|rrr}
 1 &  0 & 0 &  8 & 3 & 1\\
 0 &  1 & 0 &  10 & 4 & 1\\
 0 &  0 & 1 &  3.5 & 1.5 & 0.5 
\end{array}\right).
\end{array}
\end{equation*}
Por lo tanto la inversa es: 
\begin{equation*}
\begin{pmatrix}
8 & 3 & 1 \\
10 & 4 & 1 \\
3.5 & 1.5 & 0.5
\end{pmatrix} \ \ \ \ \finf
\end{equation*}
% Problema 5.
%------------------------------------------------------------------------------
%------------------------------------------------------------------------------
\item Sea 
\begin{equation*}
A=\left(\begin{array}{rrr}
 1 & 1 & 1\\
 0 & 2 & 3\\
 5 & 5 & 1
\end{array}\right).
\end{equation*}
Demuestre que $A$ es no singular y luego escriba $A$ como producto de matrices elementales.

\res
Veamos que $A$ es igual a 
\begin{equation*}
\left(\begin{array}{rrr}
 1 & 1 & 1\\
 0 & 2 & 3\\
 5 & 5 & 1
\end{array}\right) %
\grstep[]{R_3 \rightarrow R_3-5R_1}
%
\begin{pmatrix}
 1 & 1 & 1\\
 0 & 2 & 3\\
 0 & 0 & -4
\end{pmatrix}.
\end{equation*}
Calculemos el determinante de $A$:

$$ \det(A)=\begin{array}{|rrr|}
 1 & 1 & 1\\
 0 & 2 & 3\\
 0 & 0 & -4
\end{array}=(1)(2)(-4)=-8.$$
Por lo tanto, como el determinante de $A$ es distinto de cero eso implica que $A$ es no singular. 

Haciendo reducción hacia atras:
\begin{equation*}
\begin{pmatrix}
 1 & 1 & 1\\
 0 & 2 & 3\\
 0 & 0 & -4
\end{pmatrix}%
\grstep[]{R_3 \rightarrow -R_3/4}
%
\begin{pmatrix}
 1 & 1 & 1\\
 0 & 2 & 3\\
 0 & 0 & 1
\end{pmatrix}%
\grstep[R2 \rightarrow ]{R_1 \rightarrow -R_3/4}
%
\end{equation*}

% Problema 6.
%------------------------------------------------------------------------------
%------------------------------------------------------------------------------
\item i) Encuentre dos matrices que sean invertibles pero que su suma no sea invertible. ii) Encuentre dos matrices singulares cuya suma sea invertible. Justifique todas sus aseveraciones.


% Problema 7.
%------------------------------------------------------------------------------
%------------------------------------------------------------------------------
\item Encuentre la descomposición LU de la matriz
\begin{equation*}
\left(\begin{array}{rrrr}
 1 &  2 & -1 & 4\\
 0 & -1 &  5 & 8\\
 2 &  3 &  1 & 4\\
 1 & -1 &  6 & 4
\end{array} \right).
\end{equation*}

\res Usamos la eliminación gaussiana:
\begin{equation*}
\begin{array}{c}
\left(
\begin{array}{rrrr}
 1 &  2 & -1 & 4\\
 0 & -1 &  5 & 8\\
 2 &  3 &  1 & 4\\
 1 & -1 &  6 & 4
\end{array}
\right)%
\grstep[R_4 \rightarrow R_4-R_1]{R_3 \rightarrow R_3-2R_1}
%
\left(
\begin{array}{rrrr}
 1 &  2 & -1 &  4\\
 0 & -1 &  5 &  8\\
 0 & -1 &  3 & -4\\
 0 & -3 &  7 &  0
\end{array}
\right)%
\grstep[]{R_2 \rightarrow -R_2}
%
\left(
\begin{array}{rrrr}
 1 &  2 & -1 &  4\\
 0 &  1 & -5 & -8\\
 0 & -1 &  3 & -4\\
 0 & -3 &  7 &  0
\end{array}
\right)%
\grstep[R_4 \rightarrow R_4+3R_2]{R_3 \rightarrow R_3+R_2}\\
\\
\left(
\begin{array}{rrrr}
 1 &  2 & -1 &   4\\
 0 &  1 & -5 &  -8\\
 0 &  0 & -2 & -12\\
 0 &  0 & -8 & -24
\end{array}
\right)%
\grstep[]{R_3 \rightarrow -R_3/2}
%
\left(
\begin{array}{rrrr}
 1 &  2 & -1 &   4\\
 0 &  1 & -5 &  -8\\
 0 &  0 &  1 &   6\\
 0 &  0 & -8 & -24
\end{array}
\right)%
\grstep[]{R_4 \rightarrow R_4+8R_3}
%
\left(
\begin{array}{rrrr}
 1 &  2 & -1 &   4\\
 0 &  1 & -5 &  -8\\
 0 &  0 &  1 &   6\\
 0 &  0 &  0 &  24
\end{array}
\right)%
\grstep[]{R_4 \rightarrow R_4/24}
\\
\\
\left(
\begin{array}{rrrr}
 1 &  2 & -1 &   4\\
 0 &  1 & -5 &  -8\\
 0 &  0 &  1 &   6\\
 0 &  0 &  0 &  1
\end{array}
\right)= U
\end{array} 
\end{equation*}


% Problema 8.
%------------------------------------------------------------------------------
%------------------------------------------------------------------------------
\item Encuentre la descomposición LU de la matriz
\begin{equation*}
A=\left(\begin{array}{rrrr}
 2 &  3 & -1 & 6\\
 4 &  7 &  2 & 1\\
-2 &  5 & -2 & 0\\
 0 & -4 &  5 & 2
\end{array} \right),
\end{equation*}
y luego úsela para encontrar la solución del sistema $Ax=b$, donde 
\begin{equation*}
b=\left(\begin{array}{c}
1\\
0\\
0\\
4
\end{array} \right).
\end{equation*}
\res 
El problema anterior se resolvió utilizando .... . Ahora se hará uso de
un método más sencillo. Si $A =LU$, se sabe que $A$ se puede factorizar como:
\begin{equation*}
\left(\begin{array}{rrrr}
 2 &  3 & -1 & 6\\
 4 &  7 &  2 & 1\\
-2 &  5 & -2 & 0\\
 0 & -4 &  5 & 2
\end{array} \right) =
\left(\begin{array}{rrrr}
 1 &  0 &  0 & 0\\
 a &  1 &  0 & 0\\
 b &  c &  1 & 0\\
 d &  e &  f & 1
\end{array} \right) \left(\begin{array}{rrrr}
 2 &  3 & -1 & 6\\
 0 &  u &  v & w\\
 0 &  0 &  x & y\\
 0 &  0 &  0 & z
\end{array} \right)=LU
\end{equation*}
Observe que el primer renglón de U es el mismo que el primer renglón de A porque al reducir A a la forma triangular, no hace falta modificar los elementos del primer renglón.\\
Se pueden obtener todos los coeficientes faltantes con tan sólo multiplicar las matrices. La
componente 2, 1 de $A$ es 4. De este modo, el producto escalar del segundo renglón de $L$ y la primera columna de $U$ es igual a 4:
$$4=2a \ \ \ \text{o}  \ \ \ \ a=2$$
Después se tiene:\\
componente $2,2$: $7=6+u \rightarrow \ u=1.$\\
De aquí en adelante se pueden insertar los valores que se encuentran en $L$ y $U$:
\begin{table}[H]
\centering
\begin{tabular}{l|l}
componente $2,3$: $2=-2+v \rightarrow \ v=4.$ & componente $3,4$: $0=-6-88+y \rightarrow \ y=94.$\\ \\
componente $2,4$: $1=12+w \rightarrow \ w=-11.$ & componente $4,1$: $0=2d \rightarrow \ d=0.$\\ \\
componente $3,1$: $-2=2b \rightarrow \ b=-1.$ & componente $4,2$: $-4=e \rightarrow \ e=-4.$\\ \\
componente $3,2$: $5=-3+c \rightarrow \ c=8.$ & componente $4,3$: $5=-16-35f \rightarrow \ f=-3/5.$\\ \\
componente $3,3$: $-2=+1+32+x \rightarrow \ x=-35.$ & componente $4,4$: $2=44-3(94)/5+z \rightarrow \ z=72/5.$\\ 
\end{tabular}
\end{table}
Por lo que:
\begin{equation*}
\left(\begin{array}{rrrr}
 2 &  3 & -1 & 6\\
 4 &  7 &  2 & 1\\
-2 &  5 & -2 & 0\\
 0 & -4 &  5 & 2
\end{array} \right) =
\left(\begin{array}{rrrr}
 1 &  0 &  0 & 0\\
 2 &  1 &  0 & 0\\
-1 &  8 &  1 & 0\\
 0 &  -4 & -3/5 & 1
\end{array} \right) \left(\begin{array}{rrrr}
 2 &  3 & -1 & 6\\
 0 &  1 &  4 & -11\\
 0 &  0 &  -35 & 94\\
 0 &  0 &  0 & 72/5
\end{array} \right)=LU
\end{equation*}
% Problema 9.
%------------------------------------------------------------------------------
%------------------------------------------------------------------------------
\item Encuentre la descomposición LU de la matriz 
\begin{equation*}
A=\left(\begin{array}{rrrr}
 1 & -2 & -2 &-3\\
 3 & -9 &  0 &-9\\
-1 &  2 &  4 & 7\\
-3 & -6 & 26 & 2
\end{array} \right),
\end{equation*}
Usando esta misma descomposición como ayuda, encuentre $A^{-1}$.

\res 
\begin{equation*}
A=\left(\begin{array}{rrrr}
 1 & -2 & -2 &-3\\
 3 & -9 &  0 &-9\\
-1 &  2 &  4 & 7\\
-3 & -6 & 26 & 2
\end{array} \right)=
\left(\begin{array}{rrrr}
 1 &  0 &  0 & 0\\
 3 &  1 &  0 & 0\\
-1 &  0 &  1 & 0\\
-3 &  4 & -2 & 1
\end{array} \right) \left(\begin{array}{rrrr}
 1 & -2 & -2 & -3\\
 0 & -3 &  6 &  0\\
 0 &  0 &  2 &  4\\
 0 &  0 &  0 &  1
\end{array} \right)=LU
\end{equation*}

% Problema 10.
%------------------------------------------------------------------------------
%------------------------------------------------------------------------------
\item Encuentre la descomposición LU de la matriz por bandas
\begin{equation*}
A=\left(\begin{array}{cccc}
a_{11} & a_{12} &   0    &    0  \\
a_{21} & a_{22} & a_{23} &    0  \\
    0  & a_{32} & a_{33} & a_{34}\\
    0  &   0    & a_{43} & a_{44}
\end{array} \right),
\end{equation*}
(Para una interesante aplicación de matrices por bandas a problemas de flujo de calor en física y la importancia de obtener su descomposición LU, ver problemas 31 y 32 de Linear Algebra, D. Lay, 4th ed., p. 131 y las explicaciones que ahí se dan.)

\end{enumerate}
\end{document}